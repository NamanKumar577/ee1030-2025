\documentclass{article}
\usepackage{gvv-book}
\usepackage{gvv}
\usepackage{amsmath}
\usepackage{amsfonts}
\usepackage{tikz}
\usepackage{setspace}
\usepackage{gensymb}
\usepackage[cmex10]{amsmath}
\usepackage{amsthm}
\usepackage{mathrsfs}
\usepackage{txfonts}
\usepackage{stfloats}
\usepackage{bm}
\usepackage{cite}
\usepackage{cases}
\usepackage{subfig}
\usepackage{longtable}
\usepackage{multirow}
\usepackage{enumitem}
\usepackage{mathtools}
\usepackage{tikz}
\usepackage{circuitikz}
\usepackage{verbatim}
\usepackage[breaklinks=true]{hyperref}
\usepackage{tkz-euclide}
\usepackage{listings}
\usepackage{color}    
\usepackage{array}    
\usepackage{longtable}
\usepackage{calc}     
\usepackage{multirow} 
\usepackage{hhline}   
\usepackage{ifthen}   
\usepackage{lscape}     
\usepackage{chngcntr}
\usepackage{graphicx}
\usepackage{float}
\usepackage{multicol}
\usepackage[a4paper, left = 1.5cm, right = 1.5cm]{geometry}

\begin{document}

\begin{center}
\large
    \textbf{Samyak Gondane-AI25BTECH11029}
\end{center}
\date{}

\section*{Question}
If $\textbf{A}$ = $\myvec{-3 & 2 \\ 1 & -1}$ and $\textbf{I}$ = $\myvec{1 & 0 \\ 0 & 1}$, find the scalar $k$ so that $\textbf{A}^2 + \textbf{I} = k\textbf{A}$.

\section*{Solution}

\textbf{Given}:

\begin{align}
A = \myvec{-3 & 2 \\ 1 & -1}, \quad
I = \myvec{1 & 0 \\ 0 & 1}
\end{align}


\subsection*{Characteristic Polynomial of $A$}

The characteristic polynomial is obtained from:

\begin{align}
\det(A - \lambda I) = 0
\end{align}

\begin{align}
\det\myvec{-3 - \lambda & 2 \\ 1 & -1 - \lambda}
= (-3 - \lambda)(-1 - \lambda) - (2)(1)
\end{align}

\begin{align}
= (\lambda + 3)(\lambda + 1) - 2 = \lambda^2 + 4\lambda + 3 - 2 = \lambda^2 + 4\lambda + 1
\end{align}

So the characteristic equation is:

\begin{align}
\lambda^2 + 4\lambda + 1 = 0
\end{align}

By Cayley-Hamilton theorem, matrix $A$ satisfies its own characteristic equation:

\begin{align}
A^2 + 4A + I = 0
\end{align}


\subsection*{Rearranging the Equation}

From the Cayley-Hamilton result:

\begin{align}
A^2 + I = -4A
\end{align}

Comparing with the target equation $A^2 + I = kA$, we get:

\begin{align}
kA = -4A \Rightarrow \boxed{k = -4}
\end{align}


\subsection*{Final Answer}

\begin{align}
\boxed{k = -4}
\end{align}

\end{document}