\documentclass{article}
\usepackage{gvv-book}
\usepackage{gvv}
\usepackage{amsmath}
\usepackage{amsfonts}
\usepackage{tikz}
\usepackage{setspace}
\usepackage{gensymb}
\usepackage[cmex10]{amsmath}
\usepackage{amsthm}
\usepackage{mathrsfs}
\usepackage{txfonts}
\usepackage{stfloats}
\usepackage{bm}
\usepackage{cite}
\usepackage{cases}
\usepackage{subfig}
\usepackage{longtable}
\usepackage{multirow}
\usepackage{enumitem}
\usepackage{mathtools}
\usepackage{tikz}
\usepackage{circuitikz}
\usepackage{verbatim}
\usepackage[breaklinks=true]{hyperref}
\usepackage{tkz-euclide}
\usepackage{listings}
\usepackage{color}    
\usepackage{array}    
\usepackage{longtable}
\usepackage{calc}     
\usepackage{multirow} 
\usepackage{hhline}   
\usepackage{ifthen}   
\usepackage{lscape}     
\usepackage{chngcntr}
\usepackage{graphicx}
\usepackage{float}
\usepackage{multicol}
\usepackage[a4paper, left = 1.5cm, right = 1.5cm]{geometry}

\begin{document}

\begin{center}
\large
    \textbf{Samyak Gondane-AI25BTECH11029}
\end{center}
\date{}

\section*{Question}
If $\vec{a} = \hat{i} + \hat{j} + \hat{k}, \vec{b} = 4\hat{i} + 3\hat{j} + 4\hat{k}$ and $\vec{c} = \hat{i} + \alpha\hat{j} + \beta\hat{k}$ are linearly dependant vectors and $\abs{c} = \sqrt{3}$, then
\begin{multicols}{2}
\begin{enumerate}
    \item $\alpha = 1, \beta = -1$
    \item $\alpha = 1, \beta = \pm1$
    \item $\alpha = -1, \beta = -1$
    \item $\alpha = \pm1, \beta = 1$
\end{enumerate}
\end{multicols}

\section*{Solution}

Given three vectors in $\mathbb{R}^3$:


\begin{align}
\vec{a} = \myvec{1 \\ 1 \\ 1}, \quad
\vec{b} = \myvec{4 \\ 2 \\ 4}, \quad
\vec{c} = \myvec{1 \\ \alpha \\ \beta}
\end{align}



We are told:
\begin{itemize}
    \item The vectors are linearly dependent.
    \item The magnitude of $\vec{c}$ is $\sqrt{3}$.
\end{itemize}



\begin{align}
\norm{\vec{c}}^2 = 1^2 + \alpha^2 + \beta^2 = 3 \quad \Rightarrow \quad \alpha^2 + \beta^2 = 2 \tag{1}
\end{align}



Place the vectors as rows of a matrix:


\begin{align}
A = \myvec{
1 & 1 & 1 \\
4 & 2 & 4 \\
1 & \alpha & \beta
}
\end{align}



\subsection*{Apply row operations:}

\begin{align}
R_2 &\rightarrow R_2 - 4R_1 \
\myvec{
1 & 1 & 1 \\
0 & -2 & 0 \\
1 & \alpha & \beta
}\\
R_3 &\rightarrow R_3 - R_1 \
\myvec{
1 & 1 & 1 \\
0 & -2 & 0 \\
0 & \alpha - 1 & \beta - 1
}
\end{align}

Normalize second row:


\begin{align}
R_2 \rightarrow \frac{1}{-2} R_2 \
\myvec{
1 & 1 & 1 \\
0 & 1 & 0 \\
0 & \alpha - 1 & \beta - 1
}
\end{align}



Eliminate second column:
\begin{align}
R_1 &\rightarrow R_1 - R_2 \
\myvec{
1 & 0 & 1 \\
0 & -2 & 0 \\
0 & \alpha - 1 & \beta - 1
}\\
R_3 &\rightarrow R_3 - (\alpha - 1) R_2 \
\myvec{
1 & 1 & 1 \\
0 & -2 & 0 \\
0 & 0 & \beta - 1
}
\end{align}

Final RREF matrix:


\begin{align}
\myvec{
1 & 0 & 1 \\
0 & 1 & 0 \\
0 & 0 & \beta - 1
}
\end{align}



For the rows to be linearly dependent, the third row must be zero:


\begin{align}
\beta - 1 = 0 \quad \Rightarrow \quad \beta = 1 \tag{2}
\end{align}



Substitute Equation (2) into Equation (1):


\begin{align}
\alpha^2 + 1 = 2 \quad \Rightarrow \quad \alpha^2 = 1 \quad \Rightarrow \quad \alpha = \pm 1
\end{align}



\subsection*{Final Answer}


\begin{align}
\boxed{\alpha = \pm 1, \quad \beta = 1}
\end{align}

\end{document}