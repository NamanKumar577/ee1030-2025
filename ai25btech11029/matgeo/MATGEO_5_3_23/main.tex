\documentclass{article}
\usepackage{gvv-book}
\usepackage{gvv}
\usepackage{amsmath}
\usepackage{amsfonts}
\usepackage{tikz}
\usepackage{setspace}
\usepackage{gensymb}
\usepackage[cmex10]{amsmath}
\usepackage{amsthm}
\usepackage{mathrsfs}
\usepackage{txfonts}
\usepackage{stfloats}
\usepackage{bm}
\usepackage{cite}
\usepackage{cases}
\usepackage{subfig}
\usepackage{longtable}
\usepackage{multirow}
\usepackage{enumitem}
\usepackage{mathtools}
\usepackage{tikz}
\usepackage{circuitikz}
\usepackage{verbatim}
\usepackage[breaklinks=true]{hyperref}
\usepackage{tkz-euclide}
\usepackage{listings}
\usepackage{color}    
\usepackage{array}    
\usepackage{longtable}
\usepackage{calc}     
\usepackage{multirow} 
\usepackage{hhline}   
\usepackage{ifthen}   
\usepackage{lscape}     
\usepackage{chngcntr}
\usepackage{graphicx}
\usepackage{float}
\usepackage{multicol}
\usepackage[a4paper, left = 1.5cm, right = 1.5cm]{geometry}

\begin{document}

\begin{center}
\large
    \textbf{Samyak Gondane-AI25BTECH11029}
\end{center}
\date{}

\section*{Question}
For what value of $k$, will the following pain of equations have infinitly many solutions\\

$2x + 3y = 7$ and $(k + 2)x - 3(1 - k)y = 5k + 1$

\section*{Solution}

Given:

\begin{align}
2x + 3y &= 7 \\
(k + 2)x - 3(1 - k)y &= 5k + 1
\end{align}


\subsection*{Augmented Matrix}
Convert the system to an augmented matrix:

\begin{align}
\myvec{2 & 3 & \vert & 7 \\
k+2 & -3 + 3k & \vert & 5k + 1}
\end{align}

Let the second row be:
\begin{align}
R_2 = \myvec{a & b & \vert & c}
\quad \text{where} \quad
a = k+2,\quad b = -3 + 3k,\quad c = 5k + 1
\end{align}


\subsection*{Eliminate First Entry of Row 2}

Apply row operation:
\begin{align}
R_2 \rightarrow R_2 - \frac{a}{2} R_1
\end{align}

Compute each entry:
\begin{align}
\text{New second entry} &= b - \frac{a}{2} \cdot 3 \\
&= (-3 + 3k) - \frac{3(k + 2)}{2} \\
&= \frac{-6 + 6k - 3k - 6}{2} = \frac{3k - 12}{2}
\end{align}

\begin{align}
\text{New third entry} &= c - \frac{a}{2} \cdot 7 \\
&= (5k + 1) - \frac{7(k + 2)}{2} \\
&= \frac{10k + 2 - 7k - 14}{2} = \frac{3k - 12}{2}
\end{align}

So the matrix becomes:

\begin{align}
\myvec{2 & 3 & \vert & 7 \\
0 & \frac{3k - 12}{2} & \vert & \frac{3k - 12}{2}}
\end{align}


\subsection*{Condition for Infinitely Many Solutions}

For infinitely many solutions, the second row must reduce to:
\begin{align}
0x + 0y = 0
\Rightarrow \frac{3k - 12}{2} = 0
\Rightarrow 3k - 12 = 0
\Rightarrow k = 4
\end{align}


\subsection*{Final Answer}

\begin{align}
\boxed{k = 4}
\end{align}

\end{document}