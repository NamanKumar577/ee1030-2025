\documentclass{beamer}
\usepackage[latin1]{inputenc}

\usetheme{Madrid}
\usecolortheme{default}
\usepackage{amsmath}
\usepackage{amssymb,amsfonts,amsthm}
\usepackage{txfonts}
\usepackage{tkz-euclide}
\usepackage{listings}
\usepackage{adjustbox}
\usepackage{array}
\usepackage{tabularx}
\usepackage{gvv}
\usepackage{lmodern}
\usepackage{circuitikz}
\usepackage{tikz}
\usepackage{graphicx}
\usepackage{gensymb}
\usepackage{physics}

\setbeamertemplate{page number in head/foot}[totalframenumber]

\usepackage{tcolorbox}
\tcbuselibrary{minted,breakable,xparse,skins}



\definecolor{bg}{gray}{0.95}
\DeclareTCBListing{mintedbox}{O{}m!O{}}{%
  breakable=true,
  listing engine=minted,
  listing only,
  minted language=#2,
  minted style=default,
  minted options={%
    linenos,
    gobble=0,
    breaklines=true,
    breakafter=,,
    fontsize=\small,
    numbersep=8pt,
    #1},
  boxsep=0pt,
  left skip=0pt,
  right skip=0pt,
  left=25pt,
  right=0pt,
  top=3pt,
  bottom=3pt,
  arc=5pt,
  leftrule=0pt,
  rightrule=0pt,
  bottomrule=2pt,
  toprule=2pt,
  colback=bg,
  colframe=orange!70,
  enhanced,
  overlay={%
    \begin{tcbclipinterior}
    \fill[orange!20!white] (frame.south west) rectangle ([xshift=20pt]frame.north west);
    \end{tcbclipinterior}},
  #3,
}
\lstset{
    language=C,
    basicstyle=\ttfamily\small,
    keywordstyle=\color{blue},
    stringstyle=\color{orange},
    commentstyle=\color{green!60!black},
    numbers=left,
    numberstyle=\tiny\color{gray},
    breaklines=true,
    showstringspaces=false,
}
\title{5.4.20}
\date{1st october, 2025}
\author{Vishwambhar - EE25BTECH11025}

\begin{document}

\frame{\titlepage}
\begin{frame}{Question}
Using elementary transformations, find the inverse of the following matrix.
\begin{align}
    \myvec{3&-1\\-4&2}
\end{align}
\end{frame}

\begin{frame}{Given}
Given:
\begin{align}
    A=\myvec{3&-1\\-4&2}
\end{align}
Let $A^{-1}$ be the inverse of the given matrix $A$:
\begin{align}
    AA^{-1} = I
\end{align}
\end{frame}

\begin{frame}{Augmented Matrix}
The augmented matrix {$A|I$}:
\begin{align}
    \augvec{2}{2}{3&-1&1&0\\-4&2&0&1}R_1\rightarrow\frac{R_1}{3}\\
    \augvec{2}{2}{1&\frac{-1}{3}&\frac{1}{3}&0\\-4&2&0&1}R_2\rightarrowR_2+4R_1\\
    \augvec{2}{2}{1&\frac{-1}{3}&\frac{1}{3}&0\\0&\frac{2}{3}&\frac{4}{3}&1}R_2\rightarrow\frac{3}{2}R_2\\
    \augvec{2}{2}{1&\frac{-1}{3}&\frac{1}{3}&0\\0&1&2&\frac{3}{2}}R_1\rightarrow R_1+\frac{1}{3}R_2\\
    \augvec{2}{2}{1&0&1&\frac{1}{2}\\0&1&2&\frac{3}{2}}
\end{align}
\end{frame}

\begin{frame}{conclusion}
Therefore,
\begin{align}
    A^{-1}=\myvec{1&\frac{1}{2}\\2&\frac{3}{2}}
\end{align}
\end{frame}

\begin{frame}[fragile]
    \frametitle{C Code}
    \begin{lstlisting}
#include<stdio.h>

void define_matrix(double *out_data){
    out_data[0] = 3.0;
    out_data[1] = -1.0;
    out_data[2] = -4.0;
    out_data[3] = 2.0; 
}
    \end{lstlisting}
\end{frame}

\begin{frame}[fragile]
    \frametitle{Python Code 1}
    \begin{lstlisting}
import ctypes as ct
import numpy as np
lib = ct.CDLL("./problem.so")
entry = ct.c_double*4
lib.define_matrix.argtypes = [ct.POINTER(ct.c_double)]
data = entry()
lib.define_matrix(data)
A = np.array([[data[0],data[1]],
                  [data[2],data[3]]])    
Ainv = np.linalg.inv(A)
print("Inverse of given matrix is\n", Ainv)
    \end{lstlisting}
\end{frame}
\end{document}