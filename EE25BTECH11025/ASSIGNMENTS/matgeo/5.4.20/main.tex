\documentclass[journal]{IEEEtran}
\usepackage[a5paper, margin=10mm, onecolumn]{geometry}
\usepackage[cmex10]{amsmath}
\usepackage{amssymb,amsfonts,amsthm}
\usepackage{gvv-book}
\usepackage{gvv}
\usepackage{hyperref}
\usepackage{physics}
\usepackage{gauss}

\begin{document}
\title{5.4.20}
\author{EE25BTECH11025 - Ganachari Vishwambhar}
\maketitle

\textbf{Question}:\\
Using elementary transformations, find the inverse of the following matrix.
\begin{align}
    \myvec{3&-1\\-4&2}
\end{align}
\textbf{Solution: }\\
Given:
\begin{align}
    A=\myvec{3&-1\\-4&2}
\end{align}
Let $A^{-1}$ be the inverse of the given matrix $A$:
\begin{align}
    AA^{-1} = I
\end{align}

The augmented matrix {$A|I$}:
\begin{align}
    \augvec{2}{2}{3&-1&1&0\\-4&2&0&1}R_1\rightarrow\frac{R_1}{3}\\
    \augvec{2}{2}{1&\frac{-1}{3}&\frac{1}{3}&0\\-4&2&0&1}R_2\rightarrow R_2+4R_1\\
    \augvec{2}{2}{1&\frac{-1}{3}&\frac{1}{3}&0\\0&\frac{2}{3}&\frac{4}{3}&1}R_2\rightarrow\frac{3}{2}R_2\\
    \augvec{2}{2}{1&\frac{-1}{3}&\frac{1}{3}&0\\0&1&2&\frac{3}{2}}R_1\rightarrow R_1+\frac{1}{3}R_2\\
    \augvec{2}{2}{1&0&1&\frac{1}{2}\\0&1&2&\frac{3}{2}}
\end{align}

Therefore,
\begin{align}
    A^{-1}=\myvec{1&\frac{1}{2}\\2&\frac{3}{2}}
\end{align}
\end{document}
 
