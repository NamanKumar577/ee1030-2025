\documentclass[journal]{IEEEtran}
\usepackage[a5paper, margin=10mm, onecolumn]{geometry}
\usepackage[cmex10]{amsmath}
\usepackage{amssymb,amsfonts,amsthm}
\usepackage{gvv-book}
\usepackage{gvv}
\usepackage{hyperref}


\begin{document}
\title{5.13.51}
\author{EE25BTECH11025 - Ganachari Vishwambhar}
\maketitle

\textbf{Question}:\\
If $A=\myvec{\alpha&0\\1&1}$ and $B=\myvec{1&0\\5&1}$, then value of $\alpha$ for which $A^2=B$, is
\begin{enumerate}
\begin{multicols}{2}
    \item 1
    \item 4
    \item 2
    \item infinite
\end{multicols}
\end{enumerate}
\textbf{Solution: }\\
Given:
\begin{align}
    A = \myvec{\alpha&0\\1&1};
    B = \myvec{1&0\\5&1}
\end{align}

Using outer product,
\begin{align}
    \myvec{\alpha\\1}\myvec{\alpha&0} = \myvec{\alpha^2&0\\\alpha&0}\\
    \myvec{0\\1}\myvec{1&1} = \myvec{0&0\\1&1}
\end{align}

Adding (2) and (3):
\begin{align}
    \myvec{\alpha^2&0\\\alpha+1&1}
\end{align}

Equating (4) to $B$:
\begin{align}
    \alpha = \pm1; \alpha = 4
\end{align}

No finite $\alpha$ satisfies the above conditions. Hence $\alpha$ is infinite.

\end{document}
