\documentclass{beamer}
\usepackage[latin1]{inputenc}

\usetheme{Madrid}
\usecolortheme{default}
\usepackage{amsmath}
\usepackage{amssymb,amsfonts,amsthm}
\usepackage{txfonts}
\usepackage{tkz-euclide}
\usepackage{listings}
\usepackage{adjustbox}
\usepackage{array}
\usepackage{tabularx}
\usepackage{gvv}
\usepackage{lmodern}
\usepackage{circuitikz}
\usepackage{tikz}
\usepackage{graphicx}
\usepackage{gensymb}
\usepackage{physics}

\setbeamertemplate{page number in head/foot}[totalframenumber]

\usepackage{tcolorbox}
\tcbuselibrary{minted,breakable,xparse,skins}



\definecolor{bg}{gray}{0.95}
\DeclareTCBListing{mintedbox}{O{}m!O{}}{%
  breakable=true,
  listing engine=minted,
  listing only,
  minted language=#2,
  minted style=default,
  minted options={%
    linenos,
    gobble=0,
    breaklines=true,
    breakafter=,,
    fontsize=\small,
    numbersep=8pt,
    #1},
  boxsep=0pt,
  left skip=0pt,
  right skip=0pt,
  left=25pt,
  right=0pt,
  top=3pt,
  bottom=3pt,
  arc=5pt,
  leftrule=0pt,
  rightrule=0pt,
  bottomrule=2pt,
  toprule=2pt,
  colback=bg,
  colframe=orange!70,
  enhanced,
  overlay={%
    \begin{tcbclipinterior}
    \fill[orange!20!white] (frame.south west) rectangle ([xshift=20pt]frame.north west);
    \end{tcbclipinterior}},
  #3,
}
\lstset{
    language=C,
    basicstyle=\ttfamily\small,
    keywordstyle=\color{blue},
    stringstyle=\color{orange},
    commentstyle=\color{green!60!black},
    numbers=left,
    numberstyle=\tiny\color{gray},
    breaklines=true,
    showstringspaces=false,
}
\title{5.13.51}
\date{3rd october, 2025}
\author{Vishwambhar - EE25BTECH11025}

\begin{document}

\frame{\titlepage}
\begin{frame}{Question}
If $A=\myvec{\alpha&0\\1&1}$ and $B=\myvec{1&0\\5&1}$, then value of $\alpha$ for which $A^2=B$, is
\begin{enumerate}
\begin{multicols}{2}
    \item 1
    \item 4
    \item 2
    \item infinite
\end{multicols}
\end{enumerate}\end{frame}

\begin{frame}{Given}
Given:
\begin{align}
    A = \myvec{\alpha&0\\1&1};
    B = \myvec{1&0\\5&1}
\end{align}
\end{frame}

\begin{frame}{Outer product}
Using outer product,
\begin{align}
    \myvec{\alpha\\1}\myvec{\alpha&0} = \myvec{\alpha^2&0\\\alpha&0}\\
    \myvec{0\\1}\myvec{1&1} = \myvec{0&0\\1&1}
\end{align}

Adding (2) and (3):
\begin{align}
    \myvec{\alpha^2&0\\\alpha+1&1}
\end{align}
\end{frame}

\begin{frame}{conclusion}
Equating (4) to $B$:
\begin{align}
    \alpha = \pm1; \alpha = 4
\end{align}
No finite $\alpha$ satisfies the above conditions. Hence $\alpha$ is infinite.
\end{frame}

\begin{frame}[fragile]
    \frametitle{C Code}
    \begin{lstlisting}
#include<stdio.h>
int check(double input, int t){
    double matB[2][2] = {{1, 0}, {5, 1}};
    double matA2[2][2];
    matA2[0][0] = input*input;
    matA2[0][1] = 0;
    matA2[1][0] = input + 1;
    matA2[1][1] = 1;
    int k = 1;
    for(int i = 0; i<2; i++){
        for(int j = 0; j<2; j++){
            if(matA2[i][j]==matB[i][j]){
                continue;
            }
            else{
                k = 0;
                break;
            }
        }
    \end{lstlisting}
\end{frame}

\begin{frame}[fragile]
    \frametitle{C code}
    \begin{lstlisting}
        if(k==0){
            break;
        }
    }
    
    if(k==0){
        printf("Given alpha = %.2lf is not the solution\n", input);
        t++;
    }
    else if(k==1){
        printf("Given alpha = %.2lf is the solution\n", input);
    }
    return t;

}
    \end{lstlisting}
\end{frame}

\begin{frame}[fragile]
    \frametitle{C code}
    \begin{lstlisting}
int main(){
    double input[3] = {1, 2, 4};
    int t = 0;
    int k = 0;
    for(int i = 0; i<3; i++){
        t = k; 
    k = check(input[i], t);
    }
    if(t==2){
        printf("only solution for the given question is alpha = infinity");
    }
}
    \end{lstlisting}
\end{frame}
\end{document}