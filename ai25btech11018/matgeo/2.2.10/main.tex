\let\negmedspace\undefined
\let\negthickspace\undefined
\documentclass[journal]{IEEEtran}
\usepackage[a5paper, margin=10mm, onecolumn]{geometry}
%\usepackage{lmodern} % Ensure lmodern is loaded for pdflatex
\usepackage{tfrupee} % Include tfrupee package

\setlength{\headheight}{1cm} % Set the height of the header box
\setlength{\headsep}{0mm}     % Set the distance between the header box and the top of the text

\usepackage{gvv-book}
\usepackage{gvv}
\usepackage{cite}
\usepackage{amsmath,amssymb,amsfonts,amsthm}
\usepackage{algorithmic}
\usepackage{graphicx}
\usepackage{textcomp}
\usepackage{xcolor}
\usepackage{txfonts}
\usepackage{listings}
\usepackage{enumitem}
\usepackage{mathtools}
\usepackage{gensymb}
\usepackage{comment}
\usepackage[breaklinks=true]{hyperref}
\usepackage{tkz-euclide} 
\usepackage{listings}
% \usepackage{gvv}                                        
\def\inputGnumericTable{}                                 
\usepackage[latin1]{inputenc}                                
\usepackage{color}                                            
\usepackage{array}                                            
\usepackage{longtable}                                       
\usepackage{calc}                                             
\usepackage{multirow}                                         
\usepackage{hhline}                                           
\usepackage{ifthen}                                           
\usepackage{lscape}
\usepackage{circuitikz}
\tikzstyle{block} = [rectangle, draw, fill=blue!20, 
    text width=4em, text centered, rounded corners, minimum height=3em]
\tikzstyle{sum} = [draw, fill=blue!10, circle, minimum size=1cm, node distance=1.5cm]
\tikzstyle{input} = [coordinate]
\tikzstyle{output} = [coordinate]


\begin{document}

\bibliographystyle{IEEEtran}
\vspace{3cm}

\title{2.2.10}
\author{AI25BTECH11018-Hemanth Reddy}
 \maketitle
% \newpage
% \bigskip
{\let\newpage\relax\maketitle}

\renewcommand{\thefigure}{\theenumi}
\renewcommand{\thetable}{\theenumi}
\setlength{\intextsep}{10pt} % Space between text and floats


\numberwithin{equation}{enumi}
\numberwithin{figure}{enumi}
\renewcommand{\thetable}{\theenumi}

\textbf{Question:}\\
The vectors $\vec{A} = 3\hat{i} - 2\hat{j} + 2\hat{k}$ and $\vec{B} = \hat{i} - 2\hat{k}$ are the adjancent sides of a parallelogram. \\
The acute angle between its diagonals is \underline{\hspace{2cm}}.\\

\textbf{Solution:}\\
The diagonals of the parallelogram are given by\\
\begin{center}
\begin{align}
\vec{A} + \vec{B} =
\myvec{
4 \\
-2 \\
0
}
, \vec{A} - \vec{B} =
\myvec{
2 \\
-2 \\
4
}
\end{align}

The angle $\theta$ between them satisfies
$
\cos\theta 
= \frac{\vec{d}_1 ^{T}\vec{d}_2}{\|\vec{d}_1\| \, \|\vec{d}_2\|}
= \frac{(\vec{A}+\vec{B})^{T}(\vec{A}-\vec{B})}{\|\vec{A}+\vec{B}\| \, \|\vec{A}-\vec{B}\|}
= \frac{\|\vec{A}\|^2 - \|\vec{B}\|^2}{\|\vec{A}+\vec{B}\| \, \|\vec{A}-\vec{B}\|}.
$\\
\vspace{0.3cm}
Now compute:
\begin{align}
\|\vec{A}\|^2 = 3^2 + (-2)^2 + 2^2 = 17,
\qquad
\|\vec{B}\|^2 = 1^2 + 0^2 + (-2)^2 = 5
\end{align}
\\
\vspace{0.3cm}
\begin{align}
    \vec{A} + \vec{B} = \langle 4, -2, 0 \rangle, 
\quad \|\vec{A}+\vec{B}\| = \sqrt{20} = 2\sqrt{5},
\end{align}

\begin{align}
    \vec{A} - \vec{B} = \langle 2, -2, 4 \rangle, 
\quad \|\vec{A}-\vec{B}\| = \sqrt{24} = 2\sqrt{6}.
\end{align}

Hence
\begin{align}
    \cos\theta 
= \frac{17 - 5}{(2\sqrt{5})(2\sqrt{6})} 
= \frac{12}{4\sqrt{30}} 
= \frac{3}{\sqrt{30}}.
\end{align}



\vspace{0.3cm}
Therefore, the acute angle between the diagonals is
$
\theta = \cos^{-1}\!\left(\frac{3}{\sqrt{30}}\right) \approx 56.7^\circ.
$

\begin{figure}
    \centering
    \includegraphics[width=0.6\linewidth]{figs/parallelogram_diagonals.png}
    \caption{}
    \label{fig:placeholder}
\end{figure}








\end{center}
\end{document}
