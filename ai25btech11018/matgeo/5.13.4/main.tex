
\let\negmedspace\undefined
\let\negthickspace\undefined
\documentclass[journal]{IEEEtran}
\usepackage[a5paper, margin=10mm, onecolumn]{geometry}
%\usepackage{lmodern} % Ensure lmodern is loaded for pdflatex
\usepackage{tfrupee} % Include tfrupee package

\setlength{\headheight}{1cm} % Set the height of the header box
\setlength{\headsep}{0mm}     % Set the distance between the header box and the top of the text

\usepackage{gvv-book}
\usepackage{gvv}
\usepackage{cite}
\usepackage{amsmath,amssymb,amsfonts,amsthm}
\usepackage{amsmath}
\usepackage{algorithmic}
\usepackage{graphicx}
\usepackage{textcomp}
\usepackage{xcolor}
\usepackage{txfonts}
\usepackage{listings}
\usepackage{enumitem}
\usepackage{mathtools}
\usepackage{gensymb}
\usepackage{comment}
\usepackage[breaklinks=true]{hyperref}
\usepackage{tkz-euclide} 
\usepackage{listings}
% \usepackage{gvv}                                        
\def\inputGnumericTable{}                                 
\usepackage[latin1]{inputenc}                                
\usepackage{color}                                            
\usepackage{array}                                            
\usepackage{longtable}                                       
\usepackage{calc}                                             
\usepackage{multirow}                                         
\usepackage{hhline}                                           
\usepackage{ifthen}                                           
\usepackage{lscape}
\usepackage{circuitikz}
\tikzstyle{block} = [rectangle, draw, fill=blue!20, 
    text width=4em, text centered, rounded corners, minimum height=3em]
\tikzstyle{sum} = [draw, fill=blue!10, circle, minimum size=1cm, node distance=1.5cm]
\tikzstyle{input} = [coordinate]
\tikzstyle{output} = [coordinate]


\begin{document}

\bibliographystyle{IEEEtran}
\vspace{3cm}

\title{5.13.4}
\author{AI25BTECH11018-Hemanth Reddy}
 \maketitle
% \newpage
% \bigskip
{\let\newpage\relax\maketitle}

\renewcommand{\thefigure}{\theenumi}
\renewcommand{\thetable}{\theenumi}
\setlength{\intextsep}{10pt} % Space between text and floats


\numberwithin{equation}{enumi}
\numberwithin{figure}{enumi}
\renewcommand{\thetable}{\theenumi}

\textbf{Question:}\\


    Let $\vec{A}$ be a $2\times2$ matrix with non-zero entries and let $\vec{A}^2 = \vec{I}$, where $\vec{I}$ is $2\times2$ identity matrix. Define \\
    $Tr(\vec{A})$- sum of diagonal elements of $\vec{A}$ and\\
    $|\vec{A}|$- determinant of matrix $\vec{A}$.\\
    Statement - 1: $Tr(\vec{A}) = 0$.\\
    Statement - 2: $|\vec{A}| = 1$\\


\begin{enumerate}
    \item Statement - 1 is true, Statement - 2 is true; Statement - 2 is not a correct explanation for Statement-1.
    \item Statement - 1 is true, Statement - 2 is false.
   \item Statement - 1 is false, Statement - 2 is true.

 \item Statement - 1 is true, Statement - 2 is true; Statement - 2 is a correct explanation for Statement-1.

\end{enumerate}

\textbf{Solution:}\\
Given,\\
$\vec{A}$ is a $2\times2$ matrix with non-zero entries and $\vec{A}^2 = \vec{I}$\\
The Cayley-Hamilton Theorem states that every square matrix satisfies its own characteristic equation.\\

For a $2\times2$ matrix $\vec{A}$ , the characteristic equation is given by $\lambda^2$-Tr(A) $\lambda$+det(A)=0.\\
\begin{align}
    \text{By the theorem,}  \vec{A}^2 -Tr(A) \vec{A} +\text{det}(\vec{A}) \vec{I} =0
\end{align}



Substituting $\vec{A}^2 = \vec{I}$ into the equation:\\
\begin{align}
      \vec{I}  - Tr(A) \vec{A} + |\vec{A}|\vec{I} =0\\
   Tr(A) \vec{A} =det(A) \vec{I} +  \vec{I}  
\end{align}
   
   
Rearranging the terms:\\
\begin{align}
     \vec{A} = \vec{I}(\frac{1+det(A)}{Tr(A)}) 
\end{align}



If the trace, Tr($\vec{A}$), is not zero, we would have $\vec{A}$=$\vec{I}(\frac{1+det(A)}{Tr(A)})$. This would mean $\vec{A}$ is a scalar multiple of the identity matrix, which contradicts the problem statement that $\vec{A}$ has non-zero entries.\\

The only way for the equation to hold true for a general matrix $\vec{A}$ with non-zero entries is if the coefficient of $\vec{A}$ on the left side is zero(see eq. 4.3) , which means Tr($\vec{A}$)=0. In this case, the right side must also be zero, so 1+det($\vec{A}$)=0 \qquad 
\begin{align}
      det( \vec{A} )=-1.
\end{align}
  
    \begin{center}
    Statement - 1 is true, Statement - 2 is false.
\end{center}





\end{document}






If the trace, Tr(A), is not zero, we would have A=Tr(A)(1+det(A))​I. This would mean A is a scalar multiple of the identity matrix, which contradicts the problem statement that A has non-zero entries.

The only way for the equation to hold true for a general matrix A with non-zero entries is if the coefficient of A on the left side is zero, which means Tr(A)=0. In this case, the right side must also be zero, so 1+det(A)=0⟹det(A)=−1.