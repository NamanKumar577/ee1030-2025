\let\negmedspace\undefined
\let\negthickspace\undefined
\documentclass[journal]{IEEEtran}
\usepackage[a5paper, margin=10mm, onecolumn]{geometry}
%\usepackage{lmodern} % Ensure lmodern is loaded for pdflatex
\usepackage{tfrupee} % Include tfrupee package

\setlength{\headheight}{1cm} % Set the height of the header box
\setlength{\headsep}{0mm}     % Set the distance between the header box and the top of the text

\usepackage{gvv-book}
\usepackage{gvv}
\usepackage{cite}
\usepackage{amsmath,amssymb,amsfonts,amsthm}
\usepackage{algorithmic}
\usepackage{graphicx}
\usepackage{textcomp}
\usepackage{xcolor}
\usepackage{txfonts}
\usepackage{listings}
\usepackage{enumitem}
\usepackage{mathtools}
\usepackage{gensymb}
\usepackage{comment}
\usepackage[breaklinks=true]{hyperref}
\usepackage{tkz-euclide} 
\usepackage{listings}
% \usepackage{gvv}                                        
\def\inputGnumericTable{}                                 
\usepackage[latin1]{inputenc}                                
\usepackage{color}                                            
\usepackage{array}                                            
\usepackage{longtable}                                       
\usepackage{calc}                                             
\usepackage{multirow}                                         
\usepackage{hhline}                                           
\usepackage{ifthen}                                           
\usepackage{lscape}
\usepackage{circuitikz}
\tikzstyle{block} = [rectangle, draw, fill=blue!20, 
    text width=4em, text centered, rounded corners, minimum height=3em]
\tikzstyle{sum} = [draw, fill=blue!10, circle, minimum size=1cm, node distance=1.5cm]
\tikzstyle{input} = [coordinate]
\tikzstyle{output} = [coordinate]


\begin{document}

\bibliographystyle{IEEEtran}
\vspace{3cm}

\title{2.6.27}
\author{AI25BTECH11018-Hemanth Reddy}
 \maketitle
% \newpage
% \bigskip
{\let\newpage\relax\maketitle}

\renewcommand{\thefigure}{\theenumi}
\renewcommand{\thetable}{\theenumi}
\setlength{\intextsep}{10pt} % Space between text and floats


\numberwithin{equation}{enumi}
\numberwithin{figure}{enumi}
\renewcommand{\thetable}{\theenumi}

\textbf{Question:}\\
If A (-5,7),B(-4,-5),C(-1,-6) and D(4,5) are the vertices of a quadrilateral, find
 the area of quadrilateral ABCD.\\
\textbf{Solution:}\\

Area of quadrilateral ABCD = The area of triangle ABC + The area of triangle ACD\\Let 
$\vec{A}$ \myvec{-5\\
7},
$\vec{B}$ \myvec{-4\\
-5},
$\vec{C}$ \myvec{-1\\
-6},
$\vec{D}$ \myvec{4\\
5}
be vectors\\
\begin{align}
    \overrightarrow{AB} =   \vec{B}  -  \vec{A}  =  \myvec{1\\
-12}
\end{align}

\begin{align}
    \overrightarrow{AC}  =   \vec{C}   -   \vec{A}   =  \myvec{4\\
-13}
\end{align}
 
\begin{align}
    \overrightarrow{AD}  =   \vec{D}   -   \vec{A}   =  \myvec{9\\
-2}
\end{align}

\begin{align}
ar(ABC) &= \frac{1}{2} \, \|(\vec{B} - \vec{A}) \times (\vec{C} - \vec{A}) \|  =  17.5
\end{align}
\begin{align}
ar(ACD) &= \frac{1}{2} \, \|(\vec{C} - \vec{A}) \times (\vec{D} - \vec{A}) \|  =  54.5
\end{align}

Therefore area of quadrilateral ABCD = 17.5+54.5 = 72 sq. units


\begin{center}
    \textbf{OR}
\end{center}

Area of quadrilateral ABCD = $\frac{1}{2}|d_{1}\times d_{2}|$\\
where $d_{1}$ = $\overrightarrow{AC}$  and $d_{2}$ = $\overrightarrow{BD}$


\begin{align}
    \overrightarrow{AC}  =   \vec{C}   -   \vec{A}   =  \myvec{4\\
-13}
\end{align}

\begin{align}
    \overrightarrow{BD}  =   \vec{D}   -   \vec{B}   =  \myvec{8\\
10}
\end{align}

\begin{align}
ar(ABCD) &= \frac{1}{2} \, \|(\vec{D} - \vec{B}) \times (\vec{C} - \vec{A}) \|  =  72  sq. units
\end{align}

\begin{figure}
    \centering
    \includegraphics[width=0.5\linewidth]{figs/quadrilateral_ABCD.png}
    \caption{}
    \label{fig:placeholder}
\end{figure}




\end{document}