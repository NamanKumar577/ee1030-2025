
\documentclass{beamer}
\usepackage[utf8]{inputenc}

\usetheme{Madrid}
\usecolortheme{default}
\usepackage{amsmath,amssymb,amsfonts,amsthm}
\usepackage{txfonts}
\usepackage{tkz-euclide}
\usepackage{listings}
\usepackage{adjustbox}
\usepackage{array}
\usepackage{tabularx}
\usepackage{gvv}
\usepackage{lmodern}
\usepackage{circuitikz}
\usepackage{tikz}
\usepackage{graphicx}

\setbeamertemplate{page number in head/foot}[totalframenumber]

\usepackage{tcolorbox}
\tcbuselibrary{minted,breakable,xparse,skins}



\definecolor{bg}{gray}{0.95}
\DeclareTCBListing{mintedbox}{O{}m!O{}}{%
  breakable=true,
  listing engine=minted,
  listing only,
  minted language=#2,
  minted style=default,
  minted options={%
    linenos,
    gobble=0,
    breaklines=true,
    breakafter=,,
    fontsize=\small,
    numbersep=8pt,
    #1},
  boxsep=0pt,
  left skip=0pt,
  right skip=0pt,
  left=25pt,
  right=0pt,
  top=3pt,
  bottom=3pt,
  arc=5pt,
  leftrule=0pt,
  rightrule=0pt,
  bottomrule=2pt,
  toprule=2pt,
  colback=bg,
  colframe=orange!70,
  enhanced,
  overlay={%
    \begin{tcbclipinterior}
    \fill[orange!20!white] (frame.south west) rectangle ([xshift=20pt]frame.north west);
    \end{tcbclipinterior}},
  #3,
}
\lstset{
    language=C,
    basicstyle=\ttfamily\small,
    keywordstyle=\color{blue},
    stringstyle=\color{orange},
    commentstyle=\color{green!60!black},
    numbers=left,
    numberstyle=\tiny\color{gray},
    breaklines=true,
    showstringspaces=false,
}
\begin{document}

\title 
{5.10.3}
\date{September 28,2025}


\author 
{EE25BTECH11065-Yoshita J}






\frame{\titlepage}
\begin{frame}{Question}
Balance the following chemical equation.
\begin{align*}
    Fe + H_2O \to Fe_3O_4 + H_2
\end{align*}
\bigskip

\end{frame}


\begin{frame}{Theoretical Solution}
Let the balanced version of the equation be
\begin{align}
    x_1 Fe + x_2 H_2O \to x_3 Fe_3O_4 + x_4 H_2
\end{align}
which results in the following equations based on the conservation of each element:
\begin{align}
    \text{For Fe: } &x_1 - 3x_3 = 0 \\
    \text{For H: } &2x_2 - 2x_4 = 0 \implies x_2 - x_4 = 0 \\
    \text{For O: } &x_2 - 4x_3 = 0
\end{align}
This can be expressed as a homogeneous system of linear equations:
\begin{align}
    x_1 + 0x_2 - 3x_3 + 0x_4 &= 0 \\
    0x_1 + x_2 + 0x_3 - x_4 &= 0 \\
    0x_1 + x_2 - 4x_3 + 0x_4 &= 0
\end{align}

\end{frame}

\begin{frame}{Theoretical Solution}
This results in the matrix equation $A\mathbf{x} = \mathbf{0}$, where:
\begin{align}
    \myvec{
        1 & 0 & -3 & 0 \\
        0 & 1 & 0 & -1 \\
        0 & 1 & -4 & 0
    } \mathbf{x} = \mathbf{0},
    \quad \mathbf{x} = \myvec{ x_1 \\ x_2 \\ x_3 \\ x_4 }
\end{align}
The coefficient matrix can be reduced as follows using Gaussian elimination to find the null space:
\end{frame}

\begin{frame}{Theoretical Solution}
\begin{align}
\myvec{
1 & 0 & -3 & 0 \\
0 & 1 & 0 & -1 \\
0 & 1 & -4 & 0
}
\xrightarrow{R_3 \to R_3 - R_2}
\\
\myvec{
1 & 0 & -3 & 0 \\
0 & 1 & 0 & -1 \\
0 & 0 & -4 & 1
}
\xrightarrow{R_3 \to -\frac{1}{4}R_3}
\\
\myvec{
1 & 0 & -3 & 0 \\
0 & 1 & 0 & -1 \\
0 & 0 & 1 & -1/4
}
\xrightarrow{R_1 \to R_1 + 3R_3}
\\
\myvec{
1 & 0 & 0 & -3/4 \\
0 & 1 & 0 & -1 \\
0 & 0 & 1 & -1/4
}
\end{align}
\end{frame}


\begin{frame}{Theoretical Solution}

From the reduced row echelon form, we get the solutions in terms of the free variable $x_4$:
\begin{align}
    x_1 = \frac{3}{4}x_4, \quad x_2 = x_4, \quad x_3 = \frac{1}{4}x_4
\end{align}
\end{frame}


\begin{frame}{Theoretical Solution}
Thus,
\begin{align}
    \mathbf{x} = x_4 \myvec{ 3/4 \\ 1 \\ 1/4 \\ 1 }
\end{align}
By substituting $x_4 = 4$, the simplest integer solution is found. Hence,
\begin{align}
    \mathbf{x} = 4 \myvec{ 3/4 \\ 1 \\ 1/4 \\ 1 } = \myvec{ 3 \\ 4 \\ 1 \\ 4 }
\end{align}
This gives $x_1 = 3, x_2 = 4, x_3 = 1, \text{ and } x_4 = 4$.
\bigskip
Hence, the balanced equation finally becomes:
\begin{align}
    \boxed{3Fe + 4H_2O \to Fe_3O_4 + 4H_2}
\end{align}
\end{frame}


\begin{frame}[fragile]
    \frametitle{C Code}

    \begin{lstlisting}
#include <stdio.h>

void solve_and_print_balance() {
    int x1, x2, x3, x4;
    int found = 0;

    printf("Searching for the smallest integer coefficients...\n\n");

    for (x1 = 1; x1 <= 100 && !found; x1++) {
        for (x2 = 1; x2 <= 100 && !found; x2++) {
            for (x3 = 1; x3 <= 100 && !found; x3++) {
                for (x4 = 1; x4 <= 100 && !found; x4++) {
                    if ((x1 == 3 * x3) && (x2 == x4) && (x2 == 4 * x3)) {
                        printf("Solution found!\n");
                        printf("Coefficients are: x1=%d, x2=%d, 
    \end{lstlisting}
\end{frame}


\begin{frame}[fragile]
    \frametitle{C Code}
    \begin{lstlisting}
x3=%d, x4=%d\n\n", x1, x2, x3, x4);
                        printf("The balanced chemical equation is:\n");
                        printf("%dFe + %dH2O -> %dFe3O4 + %dH2\n", x1, x2, x3, x4);
                        found = 1;
                    }
                }
            }
        }
    }

    if (!found) {
        printf("No solution was found within the search range.\n");
    }
}


    \end{lstlisting}
\end{frame}

\end{document}
