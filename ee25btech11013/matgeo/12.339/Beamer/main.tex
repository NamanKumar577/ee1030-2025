\documentclass{beamer}
\usepackage[utf8]{inputenc}

\usetheme{Madrid}
\usecolortheme{default}
\usepackage{amsmath,amssymb,amsfonts,amsthm}
\DeclareMathOperator{\Tr}{Tr}
\usepackage{mathtools}
\usepackage{txfonts}
\usepackage{tkz-euclide}
\usepackage{listings}
\usepackage{adjustbox}
\usepackage{tfrupee}
\usepackage{array}
\usepackage{gensymb}
\usepackage{tabularx}
\usepackage{gvv}
\usepackage{lmodern}
\usepackage{circuitikz}
\usepackage{tikz}
\lstset{literate={·}{{$\cdot$}}1 {λ}{{$\lambda$}}1 {→}{{$\to$}}1}
\usepackage{graphicx}

\setbeamertemplate{page number in head/foot}[totalframenumber]

\usepackage{tcolorbox}
\tcbuselibrary{minted,breakable,xparse,skins}

\definecolor{bg}{gray}{0.95}
\DeclareTCBListing{mintedbox}{O{}m!O{}}{%
  breakable=true,
  listing engine=minted,
  listing only,
  minted language=#2,
  minted style=default,
  minted options={%
    linenos,
    gobble=0,
    breaklines=true,
    breakafter=,,
    fontsize=\small,
    numbersep=8pt,
    #1},
  boxsep=0pt,
  left skip=0pt,
  right skip=0pt,
  left=25pt,
  right=0pt,
  top=3pt,
  bottom=3pt,
  arc=5pt,
  leftrule=0pt,
  rightrule=0pt,
  bottomrule=2pt,
  toprule=2pt,
  colback=bg,
  colframe=orange!70,
  enhanced,
  overlay={%
    \begin{tcbclipinterior}
    \fill[orange!20!white] (frame.south west) rectangle ([xshift=20pt]frame.north west);
    \end{tcbclipinterior}},
  #3,
}
\lstset{
    language=C,
    basicstyle=\ttfamily\small,
    keywordstyle=\color{blue},
    stringstyle=\color{orange},
    commentstyle=\color{green!60!black},
    numbers=left,
    numberstyle=\tiny\color{gray},
    breaklines=true,
    showstringspaces=false,
}

\title{12.339}
\date{October 2, 2025}
\author{Bhargav - EE25BTECH11013}

\begin{document}

\frame{\titlepage}

\begin{frame}{Question}
\textbf{Question}: \\
If
\begin{align}
\vec{A} = \myvec{3 & -3 \\ -3 & 4}
\end{align}
then 
\begin{align}
\det{\brak{-\vec{A}^2 + 7\vec{A} - 3\vec{I}}}
\end{align}
is \\ \\
\end{frame}


\begin{frame}{Solution}
The characteristic equation of matrix $\vec{A}$ is
\begin{align}
f(\lambda) = \abs{\vec{A} - \lambda\vec{I}} = 0
\end{align}
\begin{align}
\abs{\myvec{3-\lambda & -3 \\ -3 & 4-\lambda}} = 0
\end{align}
\begin{align}
\implies \brak{3-\lambda}\brak{4-\lambda} - 9 = 0
\end{align}
\begin{align}
-\lambda^2 + 7\lambda -3 = 0
\end{align}

\end{frame}

\begin{frame}{Solution}

According to Cayley-Hamilton Theorem:
\begin{align}
f(\lambda) = f(\vec{A})
\end{align}
\begin{align}
\therefore -\vec{A}^2 + 7\vec{A} - 3\vec{I} = 0
\end{align}
\begin{align}
\therefore \abs{-\vec{A}^2 + 7\vec{A} - 3\vec{I}} = 0
\end{align}
\end{frame}

\begin{frame}[fragile]
    \frametitle{C Code}
    \begin{lstlisting}
#include <stdio.h>

double det(double *mat) {
    return mat[0]*mat[3] - mat[1]*mat[2];
}



    \end{lstlisting}
\end{frame}
\begin{frame}[fragile]
    \frametitle{Python + C Code}
    \begin{lstlisting}
import numpy as np
import ctypes
lib = ctypes.CDLL("./libcode.so")   
lib.det.argtypes = [ctypes.POINTER(ctypes.c_double)]
lib.det.restype = ctypes.c_double

A = np.array([[3, -3],
              [-3, 4]], dtype=np.float64)

I = np.eye(2, dtype=np.float64)
expr = -A @ A + 7*A - 3*I
mat_flat = expr.flatten()
det_value = lib.det(mat_flat.ctypes.data_as(ctypes.POINTER(ctypes.c_double)))
print("Matrix (-A^2 + 7A - 3I):\n", expr)
print("Determinant =", det_value)
    \end{lstlisting}
\end{frame}

\begin{frame}[fragile]
    \frametitle{Python Code}
    \begin{lstlisting}
import numpy as np
A = np.array([[3, -3],
              [-3, 4]])
I = np.eye(2)

expr = -A @ A + 7*A - 3*I

det_value = np.linalg.det(expr)

print("Matrix A:\n", A)
print("Matrix (-A^2 + 7A - 3I):\n", expr)
print("Determinant =", det_value)


    \end{lstlisting}
\end{frame}

\end{document}

