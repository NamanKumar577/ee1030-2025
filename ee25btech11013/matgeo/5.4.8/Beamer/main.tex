\documentclass{beamer}
\usepackage[utf8]{inputenc}

\usetheme{Madrid}
\usecolortheme{default}
\usepackage{amsmath,amssymb,amsfonts,amsthm}
\usepackage{mathtools}
\usepackage{txfonts}
\usepackage{tkz-euclide}
\usepackage{listings}
\usepackage{adjustbox}
\usepackage{array}
\usepackage{gensymb}
\usepackage{tabularx}
\usepackage{gvv}
\usepackage{lmodern}
\usepackage{circuitikz}
\usepackage{tikz}
\lstset{literate={·}{{$\cdot$}}1 {λ}{{$\lambda$}}1 {→}{{$\to$}}1}
\usepackage{graphicx}

\setbeamertemplate{page number in head/foot}[totalframenumber]

\usepackage{tcolorbox}
\tcbuselibrary{minted,breakable,xparse,skins}



\definecolor{bg}{gray}{0.95}
\DeclareTCBListing{mintedbox}{O{}m!O{}}{%
  breakable=true,
  listing engine=minted,
  listing only,
  minted language=#2,
  minted style=default,
  minted options={%
    linenos,
    gobble=0,
    breaklines=true,
    breakafter=,,
    fontsize=\small,
    numbersep=8pt,
    #1},
  boxsep=0pt,
  left skip=0pt,
  right skip=0pt,
  left=25pt,
  right=0pt,
  top=3pt,
  bottom=3pt,
  arc=5pt,
  leftrule=0pt,
  rightrule=0pt,
  bottomrule=2pt,
  toprule=2pt,
  colback=bg,
  colframe=orange!70,
  enhanced,
  overlay={%
    \begin{tcbclipinterior}
    \fill[orange!20!white] (frame.south west) rectangle ([xshift=20pt]frame.north west);
    \end{tcbclipinterior}},
  #3,
}
\lstset{
    language=C,
    basicstyle=\ttfamily\small,
    keywordstyle=\color{blue},
    stringstyle=\color{orange},
    commentstyle=\color{green!60!black},
    numbers=left,
    numberstyle=\tiny\color{gray},
    breaklines=true,
    showstringspaces=false,
}

\title{5.4.8}
\date{September 13, 2025}
\author{Bhargav - EE25BTECH11013}

\begin{document}

\frame{\titlepage}

\begin{frame}{Question}
Using elementary transformations, find the inverse of the following matrix
\begin{align}
\myvec{-1 & 5 \\ -3 & 2}
\end{align}
\end{frame}


\begin{frame}{Inverse}
We know that
\begin{align}
\vec{A}^{-1}\vec{A} = \vec{I}
\end{align}
where $\vec{I}$ is the $2 \times 2$ identity matrix.

So we form the augmented matrix:
\begin{align}
\augvec{2}{2}{-1 & 5 & 1 & 0 \\ -3 & 2 & 0 & 1}
\end{align}
\end{frame}

\begin{frame}{Row Transformations}
Applying row operations:
\begin{align}
\augvec{2}{2}{-1 & 5 & 1 & 0 \\ -3 & 2 & 0 & 1}
&\xleftrightarrow[R_2 \leftarrow R_2 + 3R_1]{R_1 \leftarrow -R_1}
\augvec{2}{2}{1 & -5 & -1 & 0 \\ 0 & -13 & -3 & 1}
\end{align}

\begin{align}
&\xleftrightarrow[R_1 \leftarrow R_1 + 5R_2]{R_2 \leftarrow -\frac{1}{13}R_2}
\augvec{2}{2}{1 & 0 & \frac{2}{13} & -\frac{5}{13} \\ 0 & 1 & \frac{3}{13} & -\frac{1}{13}}
\end{align}
\end{frame}


\begin{frame}{Final Inverse}
Therefore, the inverse is
\begin{align}
\vec{A}^{-1} = \frac{1}{13}\myvec{2 & -5 \\ 3 & -1}
\end{align}

This can be verified in code by showing
\begin{align}
\vec{A}^{-1}\vec{A} = \vec{I}.
\end{align}
\end{frame}

\begin{frame}[fragile]
    \frametitle{C Code}
    \begin{lstlisting}
#include <stdio.h>

void matmul(double* A, double* B, double* C) {
    for (int i = 0; i < 2; i++) {
        for (int j = 0; j < 2; j++) {
            double sum = 0.0;
            for (int k = 0; k < 2; k++) {
                sum += A[i*2 + k] * B[k*2 + j];
            }
            C[i*2 + j] = sum;
        }
    }
}


    \end{lstlisting}
\end{frame}

\begin{frame}[fragile]
    \frametitle{Python + C Code}
    \begin{lstlisting}
import ctypes
import numpy as np
lib = ctypes.CDLL("./libmatmul.so")
lib.matmul.argtypes = [
    np.ctypeslib.ndpointer(dtype=np.float64, flags="C_CONTIGUOUS"), 
    np.ctypeslib.ndpointer(dtype=np.float64, flags="C_CONTIGUOUS"),
    np.ctypeslib.ndpointer(dtype=np.float64, flags="C_CONTIGUOUS"), 
]
lib.matmul.restype = None
A = np.array([[-1, 5],
              [-3, 2]], dtype=np.float64)
B = (1/13.0) * np.array([[2, -5],
                         [3, -1]], dtype=np.float64)
C = np.zeros((2, 2), dtype=np.float64)
lib.matmul(A, B, C)
print("C = A*B =\n", C) 


    \end{lstlisting}
\end{frame}

\begin{frame}[fragile]
    \frametitle{Python Code}
    \begin{lstlisting}
import numpy as np


a = np.array([[-1, 5],[-3, 2]])
inverse_a = np.array([[2/13, -5/13], [3/13, -1/13]])

b = a@inverse_a
print(b)

    \end{lstlisting}
\end{frame}


\end{document}
