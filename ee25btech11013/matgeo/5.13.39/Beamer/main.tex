\documentclass{beamer}
\usepackage[utf8]{inputenc}

\usetheme{Madrid}
\usecolortheme{default}
\usepackage{amsmath,amssymb,amsfonts,amsthm}
\usepackage{mathtools}
\usepackage{txfonts}
\usepackage{tkz-euclide}
\usepackage{listings}
\usepackage{adjustbox}
\usepackage{tfrupee}
\usepackage{array}
\usepackage{gensymb}
\usepackage{tabularx}
\usepackage{gvv}
\usepackage{lmodern}
\usepackage{circuitikz}
\usepackage{tikz}
\lstset{literate={·}{{$\cdot$}}1 {λ}{{$\lambda$}}1 {→}{{$\to$}}1}
\usepackage{graphicx}

\setbeamertemplate{page number in head/foot}[totalframenumber]

\usepackage{tcolorbox}
\tcbuselibrary{minted,breakable,xparse,skins}



\definecolor{bg}{gray}{0.95}
\DeclareTCBListing{mintedbox}{O{}m!O{}}{%
  breakable=true,
  listing engine=minted,
  listing only,
  minted language=#2,
  minted style=default,
  minted options={%
    linenos,
    gobble=0,
    breaklines=true,
    breakafter=,,
    fontsize=\small,
    numbersep=8pt,
    #1},
  boxsep=0pt,
  left skip=0pt,
  right skip=0pt,
  left=25pt,
  right=0pt,
  top=3pt,
  bottom=3pt,
  arc=5pt,
  leftrule=0pt,
  rightrule=0pt,
  bottomrule=2pt,
  toprule=2pt,
  colback=bg,
  colframe=orange!70,
  enhanced,
  overlay={%
    \begin{tcbclipinterior}
    \fill[orange!20!white] (frame.south west) rectangle ([xshift=20pt]frame.north west);
    \end{tcbclipinterior}},
  #3,
}
\lstset{
    language=C,
    basicstyle=\ttfamily\small,
    keywordstyle=\color{blue},
    stringstyle=\color{orange},
    commentstyle=\color{green!60!black},
    numbers=left,
    numberstyle=\tiny\color{gray},
    breaklines=true,
    showstringspaces=false,
}

\title{5.13.39}
\date{September 19, 2025}
\author{Bhargav - EE25BTECH11013}

\begin{document}

\frame{\titlepage}
\begin{frame}{Question}
Let $\vec{P} = \myvec{3 & -1 & -2 \\ 2 & 0 & \alpha \\ 3 & -5 & 0}$ where $\alpha \in \mathbf{R}$. Suppose $\vec{Q}$ = $\myvec{q_{ij}}$ is a matrix such that $\vec{P}\vec{Q}$ = k$\vec{I}$, where k $\neq$ 0 and $\vec{I}$ is the identity of order 3. If $q_{23}$ = -$\frac{k}{8}$ and $\det{\vec{Q}} = \frac{k^2}{2}$, then 
\begin{enumerate}
\item a = 0, k = 8
\item 4a - k + 8 = 0
\item $\det{(\vec{P} adj(\vec{Q}))} = 2^9$
\item $\det{(\vec{Q} adj(\vec{P}))} = 2^{13}$
\end{enumerate}
\end{frame}

\begin{frame}{Solution}
It is given that
\begin{align}
\vec{P}\vec{Q} = k\vec{I}, \det{\vec{Q}} = \frac{k^2}{2}
\end{align}

Taking the determinant
\begin{align}
\brak{\det{\vec{P}}} \cdot \frac{k^2}{2} = k^3
\end{align}

\begin{align}
\abs{\myvec{3 & -1 & -2 \\ 2 & 0 & \alpha \\ 3 & -5 & 0}} = 2k
\end{align}    
\end{frame}

\begin{frame}{Solution}
\begin{align}
\myvec{3 & -1 & -2 \\ 2 & 0 & \alpha \\ 3 & -5 & 0} \xleftrightarrow[R_3 \leftarrow R_3 - R_1]{R_2 \leftarrow R_2 - \frac{2}{3}R_1}\myvec{3 & -1 & -2 \\ 0 & \frac{2}{3} & \alpha + \frac{4}{3} \\ 0 & -4 & 2}
\end{align}

From equation \brak{3} we get
\begin{align}
3 \times \brak{\frac{2}{3}\times 2 - \brak{-4} \times \brak{\alpha + \frac{4}{3}}} = 2k
\end{align}
\begin{align}
20 + 12\alpha = 2k
\end{align}
\end{frame}

\begin{frame}{Solution}
Using the relation $\vec{P}\vec{Q}=k\vec{I}$, we get the following augmented matrix

\begin{align}
\augvec{3}{3}{3 & -1 & -2 & 1 & 0 & 0\\ 2 & 0 &\alpha & 0 & 1 & 0 \\ 3 & -5 & 0&0&0&1} \xleftrightarrow[R_2 \leftarrow R_2 - 2R_1]{R_1 \leftarrow \frac{1}{3}R_1} \augvec{3}{3}{1 & -\frac{1}{3} & -\frac{2}{3} & \frac{1}{3} & 0 & 0 \\ 0 & \frac{2}{3} & \alpha + \frac{4}{3} & -\frac{2}{3} & 1 & 0 \\ 3 & -5 & 0 & 0 & 0 & 1}
\end{align}

\begin{align}
\xleftrightarrow[R_2 \leftarrow \frac{3}{2}R_2]{R_3 \leftarrow R_3 - 3R_1} \augvec{3}{3}{
1 & -\frac{1}{3} & -\frac{2}{3} & \frac{1}{3} & 0 & 0 \\
0 & 1 & \frac{3}{2}\alpha + 2 & -1 & \frac{3}{2} & 0 \\
0 & -4 & 2 & -1 & 0 & 1
} \xleftrightarrow[R_3 \leftarrow R_3 + 4R_2]{R_1 \leftarrow \frac{1}{3}R_2} 
\end{align}

\begin{align}
\augvec{3}{3}{
1 & 0 & \frac{1}{2}\alpha & 0 & \frac{1}{2} & 0 \\
0 & 1 & \frac{3}{2}\alpha + 2 & -1 & \frac{3}{2} & 0 \\
0 & 0 & 6\alpha + 10 & -5 & 6 & 1
}
\xleftrightarrow{R_3 \leftarrow \frac{1}{6\alpha+10}R_3} 
\end{align}    
\end{frame}

\begin{frame}{Solution}
\begin{align}
\augvec{3}{3}{
1 & 0 & \frac{1}{2}\alpha & 0 & \frac{1}{2} & 0 \\
0 & 1 & \frac{3}{2}\alpha + 2 & -1 & \frac{3}{2} & 0 \\
0 & 0 & 1 & -\frac{5}{6\alpha+10} & \frac{6}{6\alpha+10} & \frac{1}{6\alpha+10}
}
\end{align}
\begin{align}
\xleftrightarrow[R_2 \leftarrow R_2 - \Big(\frac{3}{2}\alpha + 2\Big) R_3]{R_1 \leftarrow R_1 - \frac{1}{2}\alpha R_3} \augvec{3}{3}{
1 & 0 & 0 & \frac{5\alpha}{12\alpha+20} & \frac{3\alpha+10}{6\alpha+20} & -\frac{\alpha}{12\alpha+20} \\
0 & 1 & 0 & -1+\frac{5(3\alpha+4)}{12\alpha+20} & \frac{3}{2}-\frac{6(3\alpha+4)}{12\alpha+20} & -\frac{3\alpha+4}{12\alpha+20} \\
0 & 0 & 1 & -\frac{5}{6\alpha+10} & \frac{6}{6\alpha+10} & \frac{1}{6\alpha+10}
}
\end{align}
\end{frame}

\begin{frame}{Solution}
From this augmented matrix,
\begin{align}
q_{23} = -k\frac{3\alpha+4}{12\alpha+20} = -\frac{k}{8} \brak{Given}
\end{align}

\begin{align}
\implies \alpha = -1
\end{align}


Substituting the value of $\alpha$ in equation \brak{6}, we get
\begin{align}
k = 4
\end{align}

n is the order of matrix B
\begin{align}
\abs{{\vec{A}\mathrm{adj}\brak{\vec{B}}}} = \abs{\vec{A}\cdot\vec{B}^{n-1}}
\end{align}
$\abs{\vec{P}} = 8$, $\abs{\vec{Q}} = 8$
    
\end{frame}

\begin{frame}{Answer}
\begin{align}
\abs{{(\vec{P} adj(\vec{Q}))}} = \abs{\vec{P}} \abs{\vec{Q}}^{2} = 8\times64 = 2^9
\end{align}

\begin{align}
\abs{{(\vec{Q} adj(\vec{P}))}} = \abs{\vec{Q}} \abs{\vec{P}}^{2} = 8\times64 = 2^9 
\end{align}

So options \brak{2} and \brak{3} are correct    
\end{frame}

\begin{frame}[fragile]
    \frametitle{C Code}
    \begin{lstlisting}
#include <stdio.h>


int determinant(int n, int mat[n][n]){
    int det = mat[0][0]*(mat[1][1]*mat[2][2] - mat[1][2]*mat[2][1])
            - mat[0][1]*(mat[1][0]*mat[2][2] - mat[1][2]*mat[2][0])
            + mat[0][2]*(mat[1][0]*mat[2][1] - mat[1][1]*mat[2][0]);
    return det;
}

int solution(int a1, int b1, int c1, int alpha){
    return ((c1-a1*alpha)/b1);
}



    \end{lstlisting}
\end{frame}
\begin{frame}[fragile]
    \frametitle{Python + C Code}
    \begin{lstlisting}
import ctypes
import numpy as np 
lib = ctypes.CDLL("./libcode.so")
array = ctypes.c_int * 3
matrix = array * 3
lib.determinant.argtypes = [matrix]
lib.determinant.restype = ctypes.c_int
lib.solution.argtypes = [ctypes.c_int, ctypes.c_int, ctypes.c_int, ctypes.c_int]
lib.solution.restype = ctypes.c_int
k = lib.solution(12 , -2, -20, -1)
P = np.array([[3, -1, -2], [2, 0, -1], [3, -5, 0]])
mat = matrix(*[ (ctypes.c_int * 3)(*row) for row in P ])
det_P = lib.determinant(mat)
det_Q = 2*k 
print("Determinant of P and Q = ", det_P)
print(det_Q*(det_P**2))
print(det_P*(det_Q**2))



    \end{lstlisting}
\end{frame}

\begin{frame}[fragile]
    \frametitle{Python Code}
    \begin{lstlisting}
import numpy as np
def determinant(mat):
    det = (mat[0][0]*(mat[1][1]*mat[2][2] - mat[1][2]*mat[2][1])
         - mat[0][1]*(mat[1][0]*mat[2][2] - mat[1][2]*mat[2][0])
         + mat[0][2]*(mat[1][0]*mat[2][1] - mat[1][1]*mat[2][0]))
    return det
def solution(a1, b1, c1, alpha):
    return (c1 - a1*alpha) / b1
k = solution(12, -2, -20, -1)
P = np.array([[3, -1, -2],
              [2,  0, -1],
              [3, -5,  0]])
det_P = determinant(P)
det_Q = 2 * k
print("Determinant of P =", det_P)
print("Determinant of Q =", det_Q)
print("det(Q) * det(P)^2 =", det_Q * (det_P**2))
print("det(P) * det(Q)^2 =", det_P * (det_Q**2))




    \end{lstlisting}
\end{frame}

\end{document}