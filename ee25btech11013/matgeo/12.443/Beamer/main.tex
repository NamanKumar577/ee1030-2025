\documentclass{beamer}
\usepackage[utf8]{inputenc}

\usetheme{Madrid}
\usecolortheme{default}
\usepackage{amsmath,amssymb,amsfonts,amsthm}
\DeclareMathOperator{\Tr}{Tr}
\usepackage{mathtools}
\usepackage{txfonts}
\usepackage{tkz-euclide}
\usepackage{listings}
\usepackage{adjustbox}
\usepackage{tfrupee}
\usepackage{array}
\usepackage{gensymb}
\usepackage{tabularx}
\usepackage{gvv}
\usepackage{lmodern}
\usepackage{circuitikz}
\usepackage{tikz}
\lstset{literate={·}{{$\cdot$}}1 {λ}{{$\lambda$}}1 {→}{{$\to$}}1}
\usepackage{graphicx}

\setbeamertemplate{page number in head/foot}[totalframenumber]

\usepackage{tcolorbox}
\tcbuselibrary{minted,breakable,xparse,skins}

\definecolor{bg}{gray}{0.95}
\DeclareTCBListing{mintedbox}{O{}m!O{}}{%
  breakable=true,
  listing engine=minted,
  listing only,
  minted language=#2,
  minted style=default,
  minted options={%
    linenos,
    gobble=0,
    breaklines=true,
    breakafter=,,
    fontsize=\small,
    numbersep=8pt,
    #1},
  boxsep=0pt,
  left skip=0pt,
  right skip=0pt,
  left=25pt,
  right=0pt,
  top=3pt,
  bottom=3pt,
  arc=5pt,
  leftrule=0pt,
  rightrule=0pt,
  bottomrule=2pt,
  toprule=2pt,
  colback=bg,
  colframe=orange!70,
  enhanced,
  overlay={%
    \begin{tcbclipinterior}
    \fill[orange!20!white] (frame.south west) rectangle ([xshift=20pt]frame.north west);
    \end{tcbclipinterior}},
  #3,
}
\lstset{
    language=C,
    basicstyle=\ttfamily\small,
    keywordstyle=\color{blue},
    stringstyle=\color{orange},
    commentstyle=\color{green!60!black},
    numbers=left,
    numberstyle=\tiny\color{gray},
    breaklines=true,
    showstringspaces=false,
}

\title{12.443}
\date{October 3, 2025}
\author{Bhargav - EE25BTECH11013}

\begin{document}

\frame{\titlepage}

\begin{frame}{Question}
\textbf{Question}: \\
The positive eigenvalue of $\myvec{2 & 1 \\ 5 & 2}$
is \\ \\
\end{frame}


\begin{frame}{Solution}
\begin{align}
\vec{A}\vec{x} = \lambda \vec{x} \implies \brak{\vec{A}-\lambda\vec{I}}\vec{x} = 0
\end{align}
\begin{align}
\abs{\vec{A} - \lambda\vec{I}} = 0    
\end{align}
 where $\lambda$ is the eigenvalue, $\vec{x}$ is the eigenvector, $\vec{I}$ is the identity matrix
\begin{align}
\abs{\myvec{2-\lambda & 1 \\ 5 & 2-\lambda}} = 0 
\end{align}
\begin{align}
\brak{2-\lambda}^2 - 5 = 0 \implies \lambda^2 - 4\lambda - 1 = 0
\end{align}
\end{frame}

\begin{frame}{Solution}
Using the quadratic formula, 
\begin{align}
\lambda = \frac{4 \pm \sqrt{16 + 4}}{2}
\end{align}
\begin{align}
\lambda = 2 \pm \sqrt{5}
\end{align}
The positive eigenvalue of $\myvec{2 & 1 \\ 5 & 2}$ is 2 + $\sqrt{5}$
\end{frame}

\begin{frame}[fragile]
    \frametitle{C Code}
    \begin{lstlisting}
#include <math.h>

double positiveeigenvalue(double a, double b, double c, double d) {
    double trace = a + d;
    double det = a*d - b*c;
    double discriminant = trace*trace - 4*det;
    double lambda1 = (trace + sqrt(discriminant)) / 2.0;
    double lambda2 = (trace - sqrt(discriminant)) / 2.0;

    if (lambda1 > 0) {
        return lambda1;
    }
    return lambda2;
}
    \end{lstlisting}
\end{frame}
\begin{frame}[fragile]
    \frametitle{Python + C Code}
    \begin{lstlisting}
import ctypes

lib = ctypes.CDLL("./code.so")

lib.positiveeigenvalue.argtypes = [ctypes.c_double, ctypes.c_double,
                                    ctypes.c_double, ctypes.c_double]
lib.positiveeigenvalue.restype = ctypes.c_double
a, b, c, d = 2.0, 1.0, 5.0, 2.0

pos_eig = lib.positiveeigenvalue(a, b, c, d)
print("The positive eigenvalue is:", pos_eig)


    \end{lstlisting}
\end{frame}

\begin{frame}[fragile]
    \frametitle{Python Code}
    \begin{lstlisting}
import numpy as np

a = np.array([[2,1], [5,2]])
x,y = np.linalg.eig(a)
if(x[0]>0):
    print("The positive eigen value: ", x[0])
else:
    print("The positive eigen value: ", x[1])


    \end{lstlisting}
\end{frame}

\end{document}

