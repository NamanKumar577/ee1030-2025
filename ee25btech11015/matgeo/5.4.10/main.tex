\let\negmedspace\undefined
\let\negthickspace\undefined
\documentclass[journal]{IEEEtran}
\usepackage[a5paper, margin=10mm, onecolumn]{geometry}
%\usepackage{lmodern} % Ensure lmodern is loaded for pdflatex
\usepackage{tfrupee} % Include tfrupee package

\setlength{\headheight}{1cm} % Set the height of the header box
\setlength{\headsep}{0mm}     % Set the distance between the header box and the top of the text

\usepackage{gvv-book}
\usepackage{gvv}
\usepackage{cite}
\usepackage{amsmath,amssymb,amsfonts,amsthm}
\usepackage{algorithmic}
\usepackage{graphicx}
\usepackage{textcomp}
\usepackage{xcolor}
\usepackage{txfonts}
\usepackage{listings}
\usepackage{enumitem}
\usepackage{mathtools}
\usepackage{gensymb}
\usepackage{comment}
\usepackage[breaklinks=true]{hyperref}
\usepackage{tkz-euclide} 
\usepackage{listings}
% \usepackage{gvv}                                        
\def\inputGnumericTable{}                                 
\usepackage[latin1]{inputenc}                                
\usepackage{color}                                            
\usepackage{array}                                            
\usepackage{longtable}                                       
\usepackage{calc}                                             
\usepackage{multirow}                                         
\usepackage{hhline}                                           
\usepackage{ifthen}                                           
\usepackage{lscape}

\begin{document}

\bibliographystyle{IEEEtran}
\vspace{3cm}

\title{5.4.10}
\author{EE25BTECH11015 - Bhoomika V}
% \maketitle
% \newpage
% \bigskip
{\let\newpage\relax\maketitle}

\renewcommand{\thefigure}{\theenumi}
\renewcommand{\thetable}{\theenumi}
\setlength{\intextsep}{10pt} % Space between text and floats


\numberwithin{equation}{enumi}
\numberwithin{figure}{enumi}
\renewcommand{\thetable}{\theenumi}
\parindent 0px 
{Question :-} \\ 
Using elementary transformations, find the inverse of the following matrix:
\[
A = \begin{bmatrix}
1 & -1 & 2 \\
2 & 3 & 5 \\
-2 & 0 & 1
\end{bmatrix}.
\]

\solution \\
\[
A = \begin{bmatrix}
1 & -1 & 2 \\
2 & 3 & 5 \\
-2 & 0 & 1
\end{bmatrix}.
\]

We form the augmented matrix $[A \mid I]$:

\[
\left[\begin{array}{ccc|ccc}
1 & -1 & 2 & 1 & 0 & 0\\
2 & 3 & 5 & 0 & 1 & 0\\
-2 & 0 & 1 & 0 & 0 & 1
\end{array}\right]
\]

\[
\overset{\substack{R_2 \to R_2 - 2R_1 \\ R_3 \to R_3 + 2R_1}}{\longrightarrow}
\left[\begin{array}{ccc|ccc}
1 & -1 & 2 & 1 & 0 & 0\\
0 & 5 & 1 & -2 & 1 & 0\\
0 & -2 & 5 & 2 & 0 & 1
\end{array}\right]
\]

\[
\overset{R_2 \to \tfrac{1}{5}R_2}{\longrightarrow}
\left[\begin{array}{ccc|ccc}
1 & -1 & 2 & 1 & 0 & 0\\
0 & 1 & \tfrac{1}{5} & -\tfrac{2}{5} & \tfrac{1}{5} & 0\\
0 & -2 & 5 & 2 & 0 & 1
\end{array}\right]
\]

\[
\overset{\substack{R_1 \to R_1 + R_2 \\ R_3 \to R_3 + 2R_2}}{\longrightarrow}
\left[\begin{array}{ccc|ccc}
1 & 0 & \tfrac{11}{5} & \tfrac{3}{5} & \tfrac{1}{5} & 0\\
0 & 1 & \tfrac{1}{5} & -\tfrac{2}{5} & \tfrac{1}{5} & 0\\
0 & 0 & \tfrac{27}{5} & \tfrac{6}{5} & \tfrac{2}{5} & 1
\end{array}\right]
\]

\[
\overset{R_3 \to \tfrac{5}{27}R_3}{\longrightarrow}
\left[\begin{array}{ccc|ccc}
1 & 0 & \tfrac{11}{5} & \tfrac{3}{5} & \tfrac{1}{5} & 0\\
0 & 1 & \tfrac{1}{5} & -\tfrac{2}{5} & \tfrac{1}{5} & 0\\
0 & 0 & 1 & \tfrac{2}{9} & \tfrac{2}{27} & \tfrac{5}{27}
\end{array}\right]
\]

\[
\overset{\substack{R_1 \to R_1 - \tfrac{11}{5}R_3 \\ R_2 \to R_2 - \tfrac{1}{5}R_3}}{\longrightarrow}
\left[\begin{array}{ccc|ccc}
1 & 0 & 0 & \tfrac{1}{9} & \tfrac{1}{27} & -\tfrac{11}{27}\\
0 & 1 & 0 & -\tfrac{4}{9} & \tfrac{5}{27} & -\tfrac{1}{27}\\
0 & 0 & 1 & \tfrac{2}{9} & \tfrac{2}{27} & \tfrac{5}{27}
\end{array}\right]
\]

Thus the inverse is

\[
A^{-1} =
\begin{bmatrix}
\dfrac{1}{9} & \dfrac{1}{27} & -\dfrac{11}{27}\\[6pt]
-\dfrac{4}{9} & \dfrac{5}{27} & -\dfrac{1}{27}\\[6pt]
\dfrac{2}{9} & \dfrac{2}{27} & \dfrac{5}{27}
\end{bmatrix}.
\]




\end{document}
