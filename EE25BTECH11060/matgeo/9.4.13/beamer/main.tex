\documentclass{beamer}
\usepackage[utf8]{inputenc}
\usetheme{Madrid}
\usecolortheme{default}
\usepackage{amsmath,amssymb,amsfonts,amsthm}
\usepackage{txfonts}
\usepackage{tkz-euclide}
\usepackage{listings}
\usepackage{adjustbox}
\usepackage{array}
\usepackage{tabularx}
\usepackage{gvv}
\usepackage{lmodern}
\usepackage{circuitikz}
\usepackage{tikz}
\usepackage{graphicx}
\setbeamertemplate{page number in head/foot}[totalframenumber]
\usepackage{tcolorbox}
\tcbuselibrary{minted,breakable,xparse,skins}
\definecolor{bg}{gray}{0.95}
\DeclareTCBListing{mintedbox}{O{}m!O{}}{%
  breakable=true,
  listing engine=minted,
  listing only,
  minted language=#2,
  minted style=default,
  minted options={%
    linenos,
    gobble=0,
    breaklines=true,
    breakafter=,,
    fontsize=\small,
    numbersep=8pt,
    #1},
  boxsep=0pt,
  left skip=0pt,
  right skip=0pt,
  left=25pt,
  right=0pt,
  top=3pt,
  bottom=3pt,
  arc=5pt,
  leftrule=0pt,
  rightrule=0pt,
  bottomrule=2pt,
  toprule=2pt,
  colback=bg,
  colframe=orange!70,
  enhanced,
  overlay={%
    \begin{tcbclipinterior}
    \fill[orange!20!white] (frame.south west) rectangle ([xshift=20pt]frame.north west);
    \end{tcbclipinterior}},
  #3,
}
\lstset{
    language=C,
    basicstyle=\ttfamily\small,
    keywordstyle=\color{blue},
    stringstyle=\color{orange},
    commentstyle=\color{green!60!black},
    numbers=left,
    numberstyle=\tiny\color{gray},
    breaklines=true,
    showstringspaces=false,
}
\begin{document}

\title 
{4.3.50}
\date{september 14,2025}


\author 
{Namaswi-EE25BTECH11060}
\frame{\titlepage}
\begin{frame}{Question}
Let $a, b, c$ be real numbers with $a^2 + b^2 + c^2 = 1$. Show that the equation
\[
\begin{vmatrix}
ax - by - c & bx + ay & cx + a \\
bx + ay & -ax + by - c & cy + b \\
cx + a & cy + b & -ax - by + c
\end{vmatrix} = 0
\]
represents a straight line.
\end{frame}
\begin{frame}{Solution}
Let us denote it as a $3 \times 3$ matrix:
\[
M = 
\begin{pmatrix}
ax - by - c & bx + ay & cx + a \\
bx + ay & -ax + by - c & cy + b \\
cx + a & cy + b & -ax - by + c
\end{pmatrix}
\]  
\end{frame}
\begin{frame}{Solution}
A determinant represents a plane if it depends quadratically on $x$ and $y$. Here, if we can reduce it to a determinant that is linear in $x$ and $y$, it will represent a straight line.\\
so,Replace
\[
R_3 \rightarrow R_3 - c R_1 - b R_2
\]

\[
\text{First element of new } R_3: \quad
\]
\begin{align}
(cx + a) - c(ax - by - c) - b(bx + ay)\\
= cx + a - cax + cby + c^2 - b^2 x - a b y\\
x(c - ca - b^2) + y(cb - ab) + (a + c^2)
\end{align}  
\end{frame}
\begin{frame}{Solution}
\[
\text{Second element of new } R_3: \quad
\]
\begin{align}
(cy + b) - c(bx + ay) - b(-ax + by - c) \\
= cy + b - cbx - cay + abx - b^2 y + bc \\
= x(-cb + ab) + y(c - ca - b^2) + (b + bc)
\end{align}
 \[
\text{Third element of new } R_3: \quad
\]
\begin{align}
(-ax - by + c) - c(cx + a) - b(cy + b) \\
= -ax - by + c - c^2 x - ac - bcy - b^2 \\
= x(-a - c^2) + y(-b - bc) + (c - ac - b^2)
\end{align}  
\end{frame}
\begin{frame}{Solution}
But since 
\begin{align}
a^2 + b^2 + c^2 = 1 \implies 1 - a^2 = b^2 + c^2,
\end{align}
all quadratic terms cancel.
Similarly, the 2nd and 3rd entries of the new $R_3$ become constants or linear in $x, y$.\\
The determinant now depends linearly on $x$ and $y$, so we can write:
\[
\det(M) = 0 \quad \implies \quad px + qy + r = 0,
\]
for some real constants $p, q, r$. \\
Hence,the determinant represents a \textbf{straight line}.  
\end{frame}
\begin{frame}[fragile]
\frametitle{C Code}
\begin{lstlisting}
    #include <stdio.h>
int main() {
    // Define constants a, b, c (such that a^2 + b^2 + c^2 = 1)
    double a, b, c;
    printf("Enter values for a, b, c (a^2 + b^2 + c^2 = 1): ");
    scanf("%lf %lf %lf", &a, &b, &c);

    // Variables x and y
    double x, y;
    printf("Enter values for x and y: ");
    scanf("%lf %lf", &x, &y);
\end{lstlisting}
\end{frame}
\begin{frame}[fragile]
\frametitle{C Code}
\begin{lstlisting}
     // Original matrix M
    double M[3][3];
    M[0][0] = a*x - b*y - c; M[0][1] = b*x + a*y;      M[0][2] = c*x + a;
    M[1][0] = b*x + a*y;     M[1][1] = -a*x + b*y - c; M[1][2] = c*y + b;
    M[2][0] = c*x + a;       M[2][1] = c*y + b;        M[2][2] = -a*x - b*y + c;

    // Apply row operation: R3 -> R3 - c*R1 - b*R2
    double newR3[3];
    newR3[0] = M[2][0] - c*M[0][0] - b*M[1][0];
    newR3[1] = M[2][1] - c*M[0][1] - b*M[1][1];
    newR3[2] = M[2][2] - c*M[0][2] - b*M[1][2];

\end{lstlisting}
\end{frame}
\begin{frame}[fragile]
\frametitle{C Code}
\begin{lstlisting}
     // Determinant using expansion along R3 (simplified after operation)
    double det = M[0][0]*(M[1][1]*newR3[2] - M[1][2]*newR3[1])
               - M[0][1]*(M[1][0]*newR3[2] - M[1][2]*newR3[0])
               + M[0][2]*(M[1][0]*newR3[1] - M[1][1]*newR3[0]);

    printf("Determinant after row operation: %lf\n", det);

    // Since quadratic terms cancel, det is linear in x and y
    printf("This determinant represents a straight line in x and y.\n");

    return 0;
}
\end{lstlisting}
\end{frame}
\end{document}