\documentclass{beamer}
\usepackage[utf8]{inputenc}

\usetheme{Madrid}
\usecolortheme{default}
\usepackage{amsmath,amssymb,amsfonts,amsthm}
\usepackage{txfonts}
\usepackage{tkz-euclide}
\usepackage{listings}
\usepackage{adjustbox}
\usepackage{array}
\usepackage{tabularx}
\usepackage{gvv}
\usepackage{lmodern}
\usepackage{circuitikz}
\usepackage{tikz}
\usepackage{graphicx}

\setbeamertemplate{page number in head/foot}[totalframenumber]

\usepackage{tcolorbox}
\tcbuselibrary{minted,breakable,xparse,skins}



\definecolor{bg}{gray}{0.95}
\DeclareTCBListing{mintedbox}{O{}m!O{}}{%
  breakable=true,
  listing engine=minted,
  listing only,
  minted language=#2,
  minted style=default,
  minted options={%
    linenos,
    gobble=0,
    breaklines=true,
    breakafter=,,
    fontsize=\small,
    numbersep=8pt,
    #1},
  boxsep=0pt,
  left skip=0pt,
  right skip=0pt,
  left=25pt,
  right=0pt,
  top=3pt,
  bottom=3pt,
  arc=5pt,
  leftrule=0pt,
  rightrule=0pt,
  bottomrule=2pt,
  toprule=2pt,
  colback=bg,
  colframe=orange!70,
  enhanced,
  overlay={%
    \begin{tcbclipinterior}
    \fill[orange!20!white] (frame.south west) rectangle ([xshift=20pt]frame.north west);
    \end{tcbclipinterior}},
  #3,
}
\lstset{
    language=C,
    basicstyle=\ttfamily\small,
    keywordstyle=\color{blue},
    stringstyle=\color{orange},
    commentstyle=\color{green!60!black},
    numbers=left,
    numberstyle=\tiny\color{gray},
    breaklines=true,
    showstringspaces=false,
}
\begin{document}
\title 
{2.9.11}
\date{september 10,2025}
\author 
{Namaswi-EE25BTECH11060}
\frame{\titlepage}
\begin{frame}{Question}
If $\Bar{a} \;$ and \;$ \Bar{b} $ are unit vectors and $\theta$ is angle between them then prove that $\sin\frac{\theta}{2}=\frac{1}{2}$ $| {\Bar{a}-\Bar{b}}|$
\end{frame}
\begin{frame}{solution}
  \begin{align*}
    \sin^2\frac{\theta}{2}=\frac{1}{4}| {\Bar{a}-\Bar{b}}|^2\\
    \end{align*}
  consider RHS,\\
\begin{align}
 \implies \frac{1}{4} \|a-b\|^2
 \end{align}
\end{frame}
\begin{frame}{solution}
  \begin{align}
       = \frac{1}{4}(a-b)^\top  (a-b)\\
 = \frac{1}{4} \brak{a ^\top a - 2 a ^\top b + b ^\top b}\\
  =\frac{1}{4} \brak{  1 - 2 a ^\top b + 1 }\\
  = \frac{1}{2} (1 - a ^\top b)\\
  =\frac{1}{2}\brak{1-\cos \theta}\\
  =\sin^2{\frac{\theta}{2}}\\
=LHS
\end{align}
Hence, $\sin\frac{\theta}{2}=\frac{1}{2}$  $| {\Bar{a}-\Bar{b}}|
$ 
\end{frame}
 
\begin{frame}[fragile]
\frametitle{C Code }
\begin{lstlisting}
#include <stdio.h>
#include <math.h>

double dot_product(double a[], double b[]) {
    return a[0]*b[0] + a[1]*b[1];
}

double norm(double v[]) {
    return sqrt(v[0]*v[0] + v[1]*v[1]);
}

void normalize(double v[]) {
    double n = norm(v);
    if (n != 0) {
        v[0] /= n;
        v[1] /= n;
    }
}
\end{lstlisting}
\end{frame}

\begin{frame}[fragile]
    \frametitle{C Code}
    \begin{lstlisting}
int main() {
    double a[2] = {1, 2};
    double b[2] = {2, 1};

    normalize(a);
    normalize(b);

    double cos_theta = dot_product(a, b);
    double theta = acos(cos_theta); 

    double diff[2] = {0.5 * (a[0] - b[0]), 0.5 * (a[1] - b[1])};
    double lhs = norm(diff);
\end{lstlisting}
\end{frame}

 

\begin{frame}[fragile]
\frametitle{C Code}
\begin{lstlisting}
double rhs = sin(theta / 2.0);

    printf("Angle θ (in degrees): %.6f\n", theta * (180.0 / M_PI));
    printf("||0.5(a - b)||       = %.6f\n", lhs);
    printf("sin(θ / 2)           = %.6f\n", rhs);
    printf("Difference           = %.6e\n", fabs(lhs - rhs));

    return 0;
}
\end{lstlisting}
\end{frame}
\begin{frame}[fragile]
\frametitle{Python Code}
\begin{lstlisting}
import numpy as np
import matplotlib.pyplot as plt
from matplotlib.patches import Arc

def plot_vectors_with_angle(a, b):
    a = np.array(a)
    b = np.array(b)

    # Calculate angle (in radians and degrees)
    dot_product = np.dot(a, b)
    norm_a = np.linalg.norm(a)
    norm_b = np.linalg.norm(b)
    cos_theta = dot_product / (norm_a * norm_b)
    theta_rad = np.arccos(np.clip(cos_theta, -1.0, 1.0))
    theta_deg = np.degrees(theta_rad)


    \end{lstlisting}
\end{frame}
\begin{frame}[fragile]
    \frametitle{Python Code}
    \begin{lstlisting}
 # Setup plot
    fig, ax = plt.subplots()
    ax.set_aspect('equal')
    ax.grid(True)

    # Calculate plot limits
    max_val = max(np.linalg.norm(a), np.linalg.norm(b)) + 1
    ax.set_xlim(-1, max_val)
    ax.set_ylim(-1, max_val)

    # Plot vectors
    origin = [0, 0]
    ax.quiver(*origin, *a, angles='xy', scale_units='xy', scale=1, color='r', label='a')
    ax.quiver(*origin, *b, angles='xy', scale_units='xy', scale=1, color='b', label='b')

    \end{lstlisting}
\end{frame}
 \begin{frame}[fragile]
    \frametitle{ Python Code}
    \begin{lstlisting}

    # Draw angle arc
    arc_radius = 0.5
    arc = Arc(origin, arc_radius*2, arc_radius*2, angle=0,
              theta1=0, theta2=theta_deg, color='green')
    ax.add_patch(arc)

    # Annotate angle
    mid_angle = theta_rad / 2
    label_radius = arc_radius * 1.4
    x_text = label_radius * np.cos(mid_angle)
    y_text = label_radius * np.sin(mid_angle)
    ax.text(x_text, y_text, f'θ = {theta_deg:.1f}°', fontsize=12, color='green')
\end{lstlisting}
\end{frame}
\begin{frame}[fragile]
\frametitle{Python Code}
    \begin{lstlisting}
         ax.add_patch(arc)

    # Annotate angle
    mid_angle = theta_rad / 2
    label_radius = arc_radius * 1.4
    x_text = label_radius * np.cos(mid_angle)
    y_text = label_radius * np.sin(mid_angle)
    ax.text(x_text, y_text, f'θ = {theta_deg:.1f}°', fontsize=12, color='green')
    \end{lstlisting}
\end{frame}
\begin{frame}[fragile]
\frametitle{C and Python Code}
    \begin{lstlisting}    
    # Labels and title
    ax.legend()
    ax.set_xlabel('X-axis')
    ax.set_ylabel('Y-axis')
    ax.set_title('Angle Between Vectors a and b')
    plt.show()

# Example usage with any two 2D vectors:
# Replace these with your own vectors
a = [3, 2]
b = [2, 4]

plot_vectors_with_angle(a, b)

    \end{lstlisting}
\end{frame}
\begin{frame}[fragile]
\frametitle{C and Python Code}
\begin{lstlisting}
import ctypes
import numpy as np
import platform
import os

# Load C shared library
if platform.system() == 'Windows':

\end{lstlisting} 
\end{frame}
\begin{frame}[fragile]
\frametitle{C and python Code}
\begin{lstlisting}
     lib = ctypes.CDLL('./vector_math.dll')
else:
    lib = ctypes.CDLL('./libvector.so')
# Define argtypes and restype
lib.half_diff_norm.argtypes = [ctypes.c_double]*4
lib.half_diff_norm.restype = ctypes.c_double
lib.sin_theta_over_2.argtypes = [ctypes.c_double]
lib.sin_theta_over_2.restype = ctypes.c_double
# Define two vectors (they will be normalized in Python)
a = np.array([1, 2], dtype=np.float64)
b = np.array([2, 1], dtype=np.float64)
# Normalize
a_hat = a / np.linalg.norm(a)
b_hat = b / np.linalg.norm(b)
\end{lstlisting}
\end{frame}
\begin{frame}[fragile]
\frametitle{C and Python Code}
\begin{lstlisting}
    # Compute dot product
dot_ab = np.dot(a_hat, b_hat)
# Call C functions
lhs = lib.half_diff_norm(a_hat[0], a_hat[1], b_hat[0], b_hat[1])
rhs = lib.sin_theta_over_2(dot_ab)
# Show results
print(f'||0.5(a - b)|| = {lhs:.6f}')
print(f'sin(θ / 2)    = {rhs:.6f}')
print(f'Difference    = {abs(lhs - rhs):.6e}')
import ctypes
import numpy as np
import platform
import os
# Load C shared library
if platform.system() == 'Windows':
    lib = ctypes.CDLL('./vector_math.dll')
else:
    lib = ctypes.CDLL('./libvector.so')
\end{lstlisting}
\end{frame}
\begin{frame}[fragile]
\frametitle{C and Python Code}
    \begin{lstlisting}
        
# Define argtypes and restype
lib.half_diff_norm.argtypes = [ctypes.c_double]*4
lib.half_diff_norm.restype = ctypes.c_double

lib.sin_theta_over_2.argtypes = [ctypes.c_double]
lib.sin_theta_over_2.restype = ctypes.c_double

# Define two vectors (they will be normalized in Python)
a = np.array([1, 2], dtype=np.float64)
b = np.array([2, 1], dtype=np.float64)

# Normalize
a_hat = a / np.linalg.norm(a)
b_hat = b / np.linalg.norm(b)

    \end{lstlisting}
\end{frame}
\begin{frame}[fragile]
\frametitle{C and Python Code}
    \begin{lstlisting}
# Compute dot product
dot_ab = np.dot(a_hat, b_hat)

# Call C functions
lhs = lib.half_diff_norm(a_hat[0], a_hat[1], b_hat[0], b_hat[1])
rhs = lib.sin_theta_over_2(dot_ab)

# Show results
print(f'||0.5(a - b)|| = {lhs:.6f}')
print(f'sin(θ / 2)    = {rhs:.6f}')
print(f'Difference    = {abs(lhs - rhs):.6e}')

\end{lstlisting}
\end{frame}
\end{document}