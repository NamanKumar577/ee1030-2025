\documentclass[12pt]{article}
\usepackage{graphicx}
\usepackage{enumitem}
\usepackage{amsmath}
\usepackage{gvv-book}
\usepackage{gvv}

\title{\textbf{Inverse of a Matrix Using Elementary Transformations}}
\author{\textbf{EE25BTECH11008 - Anirudh M Abhilash}}
\date{October 4, 2025}

\begin{document}

\maketitle

\section*{Question}

Find the inverse of the matrix 
\[
A = \myvec{2 & 3 \\ 1 & 4}
\]
using elementary transformations.

\section*{Solution}
\[
A A^{-1} = I,
\]

We write the augmented matrix of $A$ with the identity matrix:
\[
[A | I] = \augvec{2}{2}{2 & 3 & 1 & 0 \\ 1 & 4 & 0 & 1}.
\]

\textbf{Step 1:}
\[
R_1 \to \frac{1}{2} R_1
\]
\[
\augvec{2}{2}{1 & 3/2 & 1/2 & 0 \\ 1 & 4 & 0 & 1}.
\]

\textbf{Step 2:}
\[
R_2 \to R_2 - R_1
\]
\[
\augvec{2}{2}{1 & 3/2 & 1/2 & 0 \\ 0 & 5/2 & -1/2 & 1}.
\]

\textbf{Step 3:}
\[
R_2 \to \frac{2}{5} R_2
\]
\[
\augvec{2}{2}{1 & 3/2 & 1/2 & 0 \\ 0 & 1 & -1/5 & 2/5}.
\]

\textbf{Step 4:}
\[
R_1 \to R_1 - \frac{3}{2} R_2
\]
\[
\augvec{2}{2}{1 & 0 & 4/5 & -3/5 \\ 0 & 1 & -1/5 & 2/5}.
\]

Hence, the inverse of $A$ is
\[
A^{-1} = \myvec{4/5 & -3/5 \\ -1/5 & 2/5}.
\]

\end{document}
