\let\negmedspace\undefined
\let\negthickspace\undefined
\documentclass[journal]{IEEEtran}
\usepackage[a5paper, margin=10mm, onecolumn]{geometry}
%\usepackage{lmodern} % Ensure lmodern is loaded for pdflatex
\usepackage{tfrupee} % Include tfrupee package

\setlength{\headheight}{1cm} % Set the height of the header box
\setlength{\headsep}{0mm}     % Set the distance between the header box and the top of the text

\usepackage{gvv-book}
\usepackage{gvv}
\usepackage{cite}
\usepackage{amsmath,amssymb,amsfonts,amsthm}
\usepackage{algorithmic}
\usepackage{graphicx}
\usepackage{textcomp}
\usepackage{xcolor}
\usepackage{txfonts}
\usepackage{listings}
\usepackage{enumitem}
\usepackage{mathtools}
\usepackage{gensymb}
\usepackage{comment}
\usepackage[breaklinks=true]{hyperref}
\usepackage{tkz-euclide} 
\usepackage{listings}
% \usepackage{gvv}                                        
\def\inputGnumericTable{}                                 
\usepackage[latin1]{inputenc}                                
\usepackage{color}                                            
\usepackage{array}                                            
\usepackage{longtable}                                       
\usepackage{calc}                                             
\usepackage{multirow}                                         
\usepackage{hhline}                                           
\usepackage{ifthen}                                           
\usepackage{lscape}
\begin{document}

\bibliographystyle{IEEEtran}

\title{5.4.13}
\author{EE25BTECH11018- Darisy Sreetej}
% \maketitle
% \newpage
% \bigskip
\maketitle \vspace{-1cm}
\renewcommand{\thefigure}{\theenumi}
\renewcommand{\thetable}{\theenumi}
\setlength{\intextsep}{10pt} % Space between text and floats
\numberwithin{align}{enumi}
\numberwithin{figure}{enumi}
\renewcommand{\thetable}{\theenumi}

\textbf{Question:}  \\
Using elementary transformations, find the inverse of the following matrix. 
\begin{align*}
\myvec{1&3\\2&7}
\end{align*}

\textbf{Solution:}  
Given  
\begin{align}
\vec{M}=\myvec{1&3\\2&7}
\end{align}
Let $\vec{M}^{-1}$ be the inverse of $\vec{M}$.Then
\begin{align}
    \vec{M}\vec{M}^{-1}=\vec{I}
\end{align}
Augmented matrix of $\augvec{1}{1}{\vec{M} & \vec{I}}$ is given by
\begin{align}
    \augvec{2}{2}{1& 3 & 1 & 0 \\ 2 & 7 & 0 & 1} 
    \xrightarrow{R_2 \rightarrow R_2 - 2R_1}
    \augvec{2}{2}{1 & 3 & 1 & 0 \\ 0 & 1 & -2 & 1}
    \xrightarrow{R_1 \rightarrow R_1 - 3R_2}
    \augvec{2}{2}{1 & 0 & 7 & -3 \\ 0 & 1 & -2 & 1} 
    \end{align}
Hence the inverse of the matrix $\myvec{1&3\\2&7}$ is \myvec{7&-3\\-2&1}
 
\end{document}  
