\documentclass[journal]{IEEEtran}
\usepackage[a5paper, margin=10mm, onecolumn]{geometry}
\usepackage{lmodern}

\setlength{\headheight}{1cm}
\setlength{\headsep}{0mm}

\usepackage{gvv-book}
\usepackage{gvv}
\usepackage{cite}
\usepackage{amsmath,amssymb,amsfonts,amsthm}
\usepackage{graphicx}
\graphicspath{{./figs/}}
\usepackage{xcolor}
\usepackage{txfonts}
\usepackage{enumitem}
\usepackage{mathtools}
\usepackage{hyperref}
\usepackage{tikz}
\usepackage{tkz-euclide}

\begin{document}

\bibliographystyle{IEEEtran}
\vspace{3cm}

\title{5.4.31}
\author{EE25BTECH11036 - M Chanakya Srinivas}
\maketitle

\renewcommand{\thetable}{\theenumi}
\setlength{\intextsep}{10pt}
\renewcommand\theequation{\arabic{equation}}

\textbf{Question 5.4.31:}  
Using elementary row transformations, find the inverse of  
\[
A = \myvec{1 & 2 \\ 4 & 2}.
\]

\textbf{Method:}  
The inverse of a non-singular matrix \(A\) can be found using the augmented form
\[
\augvec{n}{n}{A \; I} \;\xrightarrow{\text{row operations}}\; \augvec{n}{n}{I \; A^{-1}}.
\]
This is known as the \textbf{Gauss–Jordan elimination method}.  

\textbf{Solution:}  

\begin{align}
\augvec{2}{2}{1 & 2 & 1 & 0 \\ 4 & 2 & 0 & 1}
& \quad \text{Initial augmented matrix} \label{eq:init}
\\[12pt]
R_2 \leftarrow R_2 - 4R_1: &\quad
(4-4\cdot1=0,\; 2-4\cdot2=-6,\; 0-4\cdot1=-4,\; 1-4\cdot0=1)
\\[-2pt]
\augvec{2}{2}{1 & 2 & 1 & 0 \\ 0 & -6 & -4 & 1}
& \label{eq:step1}
\\[12pt]
R_2 \leftarrow -\tfrac{1}{6}R_2: &\quad
(0,-6,-4,1)\cdot\left(-\tfrac{1}{6}\right)=(0,1,\tfrac{2}{3},-\tfrac{1}{6})
\\[-2pt]
\augvec{2}{2}{1 & 2 & 1 & 0 \\ 0 & 1 & \tfrac{2}{3} & -\tfrac{1}{6}}
& \label{eq:step2}
\\[12pt]
R_1 \leftarrow R_1 - 2R_2: &\quad
(1,2,1,0)-(0,2,\tfrac{4}{3},-\tfrac{1}{3})=(1,0,-\tfrac{1}{3},\tfrac{1}{3})
\\[-2pt]
\augvec{2}{2}{1 & 0 & -\tfrac{1}{3} & \tfrac{1}{3} \\ 0 & 1 & \tfrac{2}{3} & -\tfrac{1}{6}}
& \label{eq:step3}
\end{align}

\medskip
From \eqref{eq:step3}, the left block is \(I\), hence the right block is the inverse:
\[
A^{-1} = \myvec{-\tfrac{1}{3} & \tfrac{1}{3} \\[6pt] \tfrac{2}{3} & -\tfrac{1}{6}}.
\]



\end{document}
