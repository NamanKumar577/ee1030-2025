\let\negmedspace\undefined
\let\negthickspace\undefined
\documentclass[journal,12pt,onecolumn]{IEEEtran}
\usepackage{cite}
\usepackage{amsmath,amssymb,amsfonts,amsthm}
\usepackage{algorithmic}
\usepackage{graphicx}
\graphicspath{{./figs/}}
\usepackage{textcomp}
\usepackage{xcolor}
\usepackage{txfonts}
\usepackage{listings}
\usepackage{enumitem}
\usepackage{mathtools}
\usepackage{gensymb}
\usepackage{comment}
\usepackage{caption}
\usepackage[breaklinks=true]{hyperref}
\usepackage{tkz-euclide} 
\usepackage{listings}
\usepackage{gvv}                                        
%\def\inputGnumericTable{}                                 
\usepackage[latin1]{inputenc}     
\usepackage{xparse}
\usepackage{color}                                            
\usepackage{array}
\usepackage{longtable}                                       
\usepackage{calc}                                             
\usepackage{multirow}
\usepackage{multicol}
\usepackage{hhline}                                           
\usepackage{ifthen}                                           
\usepackage{lscape}
\usepackage{tabularx}
\usepackage{array}
\usepackage{float}
\newtheorem{theorem}{Theorem}[section]
\newtheorem{problem}{Problem}
\newtheorem{proposition}{Proposition}[section]
\newtheorem{lemma}{Lemma}[section]
\newtheorem{corollary}[theorem]{Corollary}
\newtheorem{example}{Example}[section]
\newtheorem{definition}[problem]{Definition}
\newcommand{\BEQA}{\begin{eqnarray}}
\newcommand{\EEQA}{\end{eqnarray}}
\newcommand{\define}{\stackrel{\triangle}{=}}
\theoremstyle{remark}
\newtheorem{rem}{Remark}

\begin{document}

\title{12.485}
\author{ee25btech11056 - Suraj.N}
\maketitle
\renewcommand{\thefigure}{\theenumi}
\renewcommand{\thetable}{\theenumi}

\begin{document}

\textbf{Question :} Let 

\begin{align*}
  \vec{M} = \myvec{0 & 1\\0 & 1}
\end{align*}

Which of the following is correct 
\begin{enumerate}
  \item Rank of $\vec{M}$ is 1 and $\vec{M}$ is diagonalizable
  \item Rank of $\vec{M}$ is 2 and $\vec{M}$ is diagonalizable
  \item 1 is the only eigenvalue and $\vec{M}$ is diagonalizable
  \item 1 is the only eigenvalue and $\vec{M}$ is not diagonalizable
\end{enumerate}

\textbf{Solution :}

\begin{table}[h!]
  \centering
  \begin{tabular}{|c|c|}
\hline
\textbf{Name} & \textbf{Value} \\ \hline
$\vec{A}$ & $\myvec{2 & 1 \\0 & 3}$ \\ \hline
\end{tabular}

  \caption*{Table : Matrix}
  \label{12.485}
\end{table}

First convert $\vec{M}$ into echelon form by applying row reduction

\begin{align}
\myvec{0 & 1 \\ 0 & 1}
\xleftrightarrow{\;R_2 \to R_2 - R_1}
\myvec{0 & 1 \\ 0 & 0}
\end{align}

From the echelon form we see that there is one nonzero row, hence

\begin{align}
\text{rank}(\vec{M}) = 1
\end{align}

Next find the eigenvalues. Because $\vec{M}$ is upper triangular, its eigenvalues are the diagonal entries:

\begin{align}
  \lambda_1 &= 0 & \lambda_2 &= 1 
\end{align}

Now find eigenvectors by solving 
\begin{align}
  (\vec{M}-\lambda \vec{I})\vec{v} = \vec{0}
\end{align}

For $\lambda = 0$ solve 
\begin{align}
\vec{M}\vec{v}=\vec{0}\\
\myvec{0 & 1 \\ 0 & 1}\vec{v} = \myvec{0\\0}\\
\myvec{0 & 1\\0 & 1}\myvec{x\\y} = \myvec{0\\0}
\end{align}

\pagebreak

This gives 

\begin{align}
y = 0
\end{align}

And x can be anything\\

Thus an eigenvector for $\lambda=0$ is 
\begin{align} 
\vec{v}_1 = \myvec{1\\0}
\end{align}

For $\lambda = 1$ solve 

\begin{align}
(\vec{M}-\vec{I})\vec{v}=\vec{0}\\
\myvec{-1 & 1 \\ 0 & 0}\vec{v} = \myvec{0\\0}\\
\myvec{-1 & 1 \\ 0 & 0}\myvec{x\\y} = \myvec{0\\0}
\end{align}
This gives 
\begin{align}
-x + y = 0 \\
y = x
\end{align}
Thus an eigenvector for $\lambda=1$ is
\begin{align}
\vec{v}_2 = \myvec{1\\1}
\end{align}

Since the eigenvalues $\lambda_1$ and $\lambda_2$ are distinct, the matrix $\vec{M}$ is diagonalizable.

Form the matrix $\vec{P}$ with eigenvectors as columns and the diagonal matrix $\vec{D}$ of eigenvalues:
\begin{align}
\vec{P} = \myvec{1 & 1 \\ 0 & 1}\\
\vec{D} = \myvec{0 & 0 \\ 0 & 1}
\end{align}

Compute $\vec{P}^{-1}$ 

\begin{align}
\myaugvec{2}{1 & 1 & 1 & 0\\0 & 1 & 0 & 1}
\xleftrightarrow{\;R_1 \to R_1 - R_2}
\myaugvec{2}{1 & 0 & 1 & -1\\0 & 1 & 0 & 1}
\end{align}

The right block gives
\begin{align}
\vec{P}^{-1} = \myvec{1 & -1 \\ 0 & 1}
\end{align}

Finally, the diagonalization:
\begin{align}
  \vec{M} &= \vec{P}\vec{D}\vec{P}^{-1} \\
  \vec{M} &= \myvec{1 & 1 \\ 0 & 1}\myvec{0 & 0 \\ 0 & 1}\myvec{1 & -1 \\ 0 & 1}
\end{align}

\textbf{Conclusion :} The matrix $\vec{M}$ has $rank(\vec{M}) = 1$ and is \textbf{diagonizable}.\\
Therefore the correct option is (1).

\end{document}

