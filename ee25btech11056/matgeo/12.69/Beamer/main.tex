\documentclass{beamer}
\mode<presentation>
\usepackage{amsmath,amssymb,mathtools}
\usepackage{textcomp}
\usepackage{gensymb}
\usepackage{adjustbox}
\usepackage{subcaption}
\usepackage{enumitem}
\usepackage{multicol}
\usepackage{listings}
\usepackage{url}
\usepackage{graphicx} % <-- needed for images
\def\UrlBreaks{\do\/\do-}

\usetheme{Boadilla}
\usecolortheme{lily}
\setbeamertemplate{footline}{
  \leavevmode%
  \hbox{%
  \begin{beamercolorbox}[wd=\paperwidth,ht=2ex,dp=1ex,right]{author in head/foot}%
    \insertframenumber{} / \inserttotalframenumber\hspace*{2ex}
  \end{beamercolorbox}}%
  \vskip0pt%
}
\setbeamertemplate{navigation symbols}{}

\lstset{
  frame=single,
  breaklines=true,
  columns=fullflexible,
  basicstyle=\ttfamily\tiny   % tiny font so code fits
}

\numberwithin{equation}{section}

% ---- your macros ----
\providecommand{\nCr}[2]{\,^{#1}C_{#2}}
\providecommand{\nPr}[2]{\,^{#1}P_{#2}}
\providecommand{\mbf}{\mathbf}
\providecommand{\pr}[1]{\ensuremath{\Pr\left(#1\right)}}
\providecommand{\qfunc}[1]{\ensuremath{Q\left(#1\right)}}
\providecommand{\sbrak}[1]{\ensuremath{{}\left[#1\right]}}
\providecommand{\lsbrak}[1]{\ensuremath{{}\left[#1\right.}}
\providecommand{\rsbrak}[1]{\ensuremath{\left.#1\right]}}
\providecommand{\brak}[1]{\ensuremath{\left(#1\right)}}
\providecommand{\lbrak}[1]{\ensuremath{\left(#1\right.}}
\providecommand{\rbrak}[1]{\ensuremath{\left.#1\right)}}
\providecommand{\cbrak}[1]{\ensuremath{\left\{#1\right\}}}
\providecommand{\lcbrak}[1]{\ensuremath{\left\{#1\right.}}
\providecommand{\rcbrak}[1]{\ensuremath{\left.#1\right\}}}
\theoremstyle{remark}
\newtheorem{rem}{Remark}
\newcommand{\sgn}{\mathop{\mathrm{sgn}}}
\providecommand{\abs}[1]{\left\vert#1\right\vert}
\providecommand{\res}[1]{\Res\displaylimits_{#1}}
\providecommand{\norm}[1]{\lVert#1\rVert}
\providecommand{\mtx}[1]{\mathbf{#1}}
\providecommand{\mean}[1]{E\left[ #1 \right]}
\providecommand{\fourier}{\overset{\mathcal{F}}{ \rightleftharpoons}}
\providecommand{\system}{\overset{\mathcal{H}}{ \longleftrightarrow}}
\providecommand{\dec}[2]{\ensuremath{\overset{#1}{\underset{#2}{\gtrless}}}}
\newcommand{\myvec}[1]{\ensuremath{\begin{pmatrix}#1\end{pmatrix}}}
\newcommand{\mydet}[1]{\ensuremath{\begin{vmatrix}#1\end{vmatrix}}}

\newenvironment{amatrix}[1]{%
  \left(\begin{array}{@{}*{#1}{c}|*{#1}{c}@{}}
}{%
  \end{array}\right)
}

\newcommand{\myaugvec}[2]{\ensuremath{\begin{amatrix}{#1}#2\end{amatrix}}}
\let\vec\mathbf
% ---------------------

\title{Matgeo Presentation - Problem 12.69}
\author{ee25btech11056 - Suraj.N}

\begin{document}

\begin{frame}
  \titlepage
\end{frame}

\begin{frame}{Problem Statement}

Find the \textbf{condition number} for the matrix
\[
\vec{A} = \myvec{2 & 1 \\ 0 & 3}
\]

\end{frame}

\begin{frame}{Data}

\begin{table}[h!]
  \centering
  \begin{tabular}{|c|c|}
\hline
\textbf{Name} & \textbf{Value} \\ \hline
$\vec{A}$ & $\myvec{2 & 1 \\0 & 3}$ \\ \hline
\end{tabular}

  \caption*{Table : Matrix}
  \label{12.69}
\end{table}

The \textbf{condition number} of a matrix measures how sensitive the solution of a linear system involving that matrix is to small changes or errors in the input data. More precisely, it is the ratio of the largest singular value of the matrix to the smallest singular value

\begin{align}
\kappa(\vec{A}) = \frac{\sigma_{\max}(\vec{A})}{\sigma_{\min}(\vec{A})} 
\end{align}

\end{frame}

\begin{frame}{Solution}

\textbf{SVD / singular-value method}

Calculate $\vec{A}^\top\vec{A}$

\begin{align}
\vec{A}^\top &= \myvec{2 & 0 \\ 1 & 3} \\
\vec{A}^\top\vec{A} &= \myvec{2 & 0 \\ 1 & 3}\myvec{2 & 1 \\ 0 & 3}
= \myvec{4 & 2 \\ 2 & 10}
\end{align}

Then , find the eigen values of $\vec{A}^\top\vec{A}$
\begin{align}
\mydet{\vec{A}^\top\vec{A}-\lambda\vec{I}} = 0 
\end{align}
\begin{align}
\mydet{4-\lambda & 2 \\ 2 & 10-\lambda} = 0 
\end{align}
\begin{align}
\mydet{4-\lambda & 2 \\ 2 & 10-\lambda} 
\xleftrightarrow{R_2 \to R_2 -\tfrac{2}{(4-\lambda)}R_1}
\mydet{4-\lambda & 2 \\ 0 & \frac{(4-\lambda)(10-\lambda)-4}{(4-\lambda)}} = 0
\end{align}

\end{frame}

\begin{frame}{Solution}

By calculating the determinant 

\begin{align}
(4-\lambda)(10-\lambda)-4 = 0 \\
\lambda^2 - 14\lambda + 36 = 0
\end{align}

The eigenvalues are

\begin{align}
\lambda_i &= \frac{14 \pm \sqrt{196 - 144}}{2}
= \frac{14 \pm \sqrt{52}}{2}
= 7 \pm \sqrt{13}
\end{align}

So,

\begin{align}
\lambda_1 &= 7+\sqrt{13} & \lambda_2 &= 7-\sqrt{13}
\end{align}

The singular values are

\begin{align}
\sigma_{\max} &= \sqrt{7+\sqrt{13}} & \sigma_{\min} = \sqrt{7-\sqrt{13}}
\end{align}

\end{frame}

\begin{frame}{Solution}

  Finally, the \textbf{condition number} is

\begin{align}
\kappa(\vec{A}) = \frac{\sigma_{\max}(\vec{A})}{\sigma_{\min}(\vec{A})} = \sqrt{\frac{7+\sqrt{13}}{7-\sqrt{13}}} = 1.768
\end{align}

The \textbf{condition number} of $\vec{A}$ is 

\begin{align}
\kappa(\vec{A}) = 1.768
\end{align}

\end{frame}

\end{document}
