\let\negmedspace\undefined
\let\negthickspace\undefined
\documentclass[journal,12pt,onecolumn]{IEEEtran}
\usepackage{cite}
\usepackage{amsmath,amssymb,amsfonts,amsthm}
\usepackage{algorithmic}
\usepackage{graphicx}
\graphicspath{{./figs/}}
\usepackage{textcomp}
\usepackage{xcolor}
\usepackage{txfonts}
\usepackage{listings}
\usepackage{enumitem}
\usepackage{mathtools}
\usepackage{gensymb}
\usepackage{comment}
\usepackage{caption}
\usepackage[breaklinks=true]{hyperref}
\usepackage{tkz-euclide} 
\usepackage{listings}
\usepackage{gvv}                                        
%\def\inputGnumericTable{}                                 
\usepackage[latin1]{inputenc}     
\usepackage{xparse}
\usepackage{color}                                            
\usepackage{array}
\usepackage{longtable}                                       
\usepackage{calc}                                             
\usepackage{multirow}
\usepackage{multicol}
\usepackage{hhline}                                           
\usepackage{ifthen}                                           
\usepackage{lscape}
\usepackage{tabularx}
\usepackage{array}
\usepackage{float}
\newtheorem{theorem}{Theorem}[section]
\newtheorem{problem}{Problem}
\newtheorem{proposition}{Proposition}[section]
\newtheorem{lemma}{Lemma}[section]
\newtheorem{corollary}[theorem]{Corollary}
\newtheorem{example}{Example}[section]
\newtheorem{definition}[problem]{Definition}
\newcommand{\BEQA}{\begin{eqnarray}}
\newcommand{\EEQA}{\end{eqnarray}}
\newcommand{\define}{\stackrel{\triangle}{=}}
\theoremstyle{remark}
\newtheorem{rem}{Remark}

\begin{document}

\title{5.13.81}
\author{ee25btech11056 - Suraj.N}
\maketitle
\renewcommand{\thefigure}{\theenumi}
\renewcommand{\thetable}{\theenumi}

\begin{document}

\textbf{Question :} Let \(S=\{\vec{A}=\myvec{0 & 1 & c\\[2pt] 1 & a & d\\[2pt] 1 & b & e} : a,b,c,d,e\in\{0,1\}\text{ and }|\vec{A}|\in\{-1,1\}\}\).
Find the number of elements in \(S\).

\textbf{Solution :} 

\begin{table}[h!]
  \centering
  \begin{tabular}{|c|c|}
\hline
\textbf{Name} & \textbf{Value} \\ \hline
$\vec{A}$ & $\myvec{2 & 1 \\0 & 3}$ \\ \hline
\end{tabular}

  \caption*{Table : Matrix}
  \label{5.13.81}
\end{table}

Rearranging the rows of $\vec{A}$

\begin{align}
\myvec{0 & 1 & c\\[2pt] 1 & a & d\\[2pt] 1 & b & e}
\xleftrightarrow{R_2 \leftrightarrow R_3}
\myvec{0 & 1 & c\\1 & b & e\\1 & a & d}
\xleftrightarrow{R_1 \leftrightarrow R_2}
\myvec{1 & b & e\\0 & 1 & c\\1 & a & d}
\end{align}

Applyting row operation to $\vec{A}$ to reduce it into Echelon form

\begin{align}
\myvec{1 & b & e\\0 & 1 & c\\1 & a & d}
\xleftrightarrow{R_3 \to R_3 - R_1}
\myvec{1 & b & e\\0 & 1 & c\\0 & a-b & d-e}
\xleftrightarrow{R_3 \to R_3 - (a-b)R_2}
\myvec{
  \makebox[1cm][c]{1} & \makebox[1cm][c]{b} & \makebox[2cm][c]{e} \\
  \makebox[1cm][c]{0} & \makebox[1cm][c]{1} & \makebox[2cm][c]{c} \\
  \makebox[1cm][c]{0} & \makebox[1cm][c]{0} & \makebox[2.6cm][c]{$d-e-c(a-b)$}
}
\end{align}

Finding the determinant by the first column 

\begin{align}
  \mydet{\vec{A}} &= d - e - c(a - b)
\end{align}

Taking cases to find the possibilities of matrix $\vec{A}$

Case 1 : $\mydet{\vec{A}}=1$ \\

if c = 0 \\
the value of b and a can be 0 or 1.

\begin{align}
d - e &= 1
\end{align}

So,

\begin{align}
d &= 1 \\  
e &= 0
\end{align}

By permutation we get ,\\

\begin{align}
2 \times 2 \times 1 \times 1 = 4
\end{align}

\pagebreak

if c = 1, we get 4 possibilities

\begin{align}
d - e -(a - b) &= 1
\end{align}

So,

\begin{align}
  d &= 1 & e&=0 \\   
  b&=a=1 & b&=a=0
\end{align}

\begin{align}
  a &= 0 & b&=1 \\   
  d&=e=1 & d&=e=0
\end{align}

Case 2 : $\mydet{\vec{A}}= -1$ \\

if c = 0 \\
the value of b and a can be 0 or 1.

\begin{align}
d - e &= -1
\end{align}

So,

\begin{align}
d &= 0 \\  
e &= 1
\end{align}

By permutation we get ,\\

\begin{align}
2 \times 2 \times 1 \times 1 = 4
\end{align}

if c = 1, we get 4 possibilities

\begin{align}
d - e -(a - b) &= -1
\end{align}

So,

\begin{align}
  d &= 0 & e&=1 \\   
  b&=a=1 & b&=a=0
\end{align}

\begin{align}
  a &= 1 & b&=0 \\   
  d&=e=1 & d&=e=0
\end{align}

\pagebreak 

By adding all the possibilities , we get 

\begin{align}
  4 + 4 + 4 + 4 &= 16
\end{align}

Therefore, the number of elements in \(S=16\).

\end{document}


