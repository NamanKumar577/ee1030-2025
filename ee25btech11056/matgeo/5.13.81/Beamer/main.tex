\documentclass{beamer}
\mode<presentation>
\usepackage{amsmath,amssymb,mathtools}
\usepackage{textcomp}
\usepackage{gensymb}
\usepackage{adjustbox}
\usepackage{subcaption}
\usepackage{enumitem}
\usepackage{multicol}
\usepackage{listings}
\usepackage{url}
\usepackage{graphicx} % <-- needed for images
\def\UrlBreaks{\do\/\do-}

\usetheme{Boadilla}
\usecolortheme{lily}
\setbeamertemplate{footline}{
  \leavevmode%
  \hbox{%
  \begin{beamercolorbox}[wd=\paperwidth,ht=2ex,dp=1ex,right]{author in head/foot}%
    \insertframenumber{} / \inserttotalframenumber\hspace*{2ex}
  \end{beamercolorbox}}%
  \vskip0pt%
}
\setbeamertemplate{navigation symbols}{}

\lstset{
  frame=single,
  breaklines=true,
  columns=fullflexible,
  basicstyle=\ttfamily\tiny   % tiny font so code fits
}

\numberwithin{equation}{section}

% ---- your macros ----
\providecommand{\nCr}[2]{\,^{#1}C_{#2}}
\providecommand{\nPr}[2]{\,^{#1}P_{#2}}
\providecommand{\mbf}{\mathbf}
\providecommand{\pr}[1]{\ensuremath{\Pr\left(#1\right)}}
\providecommand{\qfunc}[1]{\ensuremath{Q\left(#1\right)}}
\providecommand{\sbrak}[1]{\ensuremath{{}\left[#1\right]}}
\providecommand{\lsbrak}[1]{\ensuremath{{}\left[#1\right.}}
\providecommand{\rsbrak}[1]{\ensuremath{\left.#1\right]}}
\providecommand{\brak}[1]{\ensuremath{\left(#1\right)}}
\providecommand{\lbrak}[1]{\ensuremath{\left(#1\right.}}
\providecommand{\rbrak}[1]{\ensuremath{\left.#1\right)}}
\providecommand{\cbrak}[1]{\ensuremath{\left\{#1\right\}}}
\providecommand{\lcbrak}[1]{\ensuremath{\left\{#1\right.}}
\providecommand{\rcbrak}[1]{\ensuremath{\left.#1\right\}}}
\theoremstyle{remark}
\newtheorem{rem}{Remark}
\newcommand{\sgn}{\mathop{\mathrm{sgn}}}
\providecommand{\abs}[1]{\left\vert#1\right\vert}
\providecommand{\res}[1]{\Res\displaylimits_{#1}}
\providecommand{\norm}[1]{\lVert#1\rVert}
\providecommand{\mtx}[1]{\mathbf{#1}}
\providecommand{\mean}[1]{E\left[ #1 \right]}
\providecommand{\fourier}{\overset{\mathcal{F}}{ \rightleftharpoons}}
\providecommand{\system}{\overset{\mathcal{H}}{ \longleftrightarrow}}
\providecommand{\dec}[2]{\ensuremath{\overset{#1}{\underset{#2}{\gtrless}}}}
\newcommand{\myvec}[1]{\ensuremath{\begin{pmatrix}#1\end{pmatrix}}}
\newcommand{\mydet}[1]{\ensuremath{\begin{vmatrix}#1\end{vmatrix}}}

\newenvironment{amatrix}[1]{%
  \left(\begin{array}{@{}*{#1}{c}|*{#1}{c}@{}}
}{%
  \end{array}\right)
}

\newcommand{\myaugvec}[2]{\ensuremath{\begin{amatrix}{#1}#2\end{amatrix}}}
\let\vec\mathbf
% ---------------------

\title{Matgeo Presentation - Problem 5.13.81}
\author{ee25btech11056 - Suraj.N}

\begin{document}

\begin{frame}
  \titlepage
\end{frame}

\begin{frame}{Problem Statement}

Let \(S=\{\vec{A}=\myvec{0 & 1 & c\\[2pt] 1 & a & d\\[2pt] 1 & b & e} : a,b,c,d,e\in\{0,1\}\text{ and }|\vec{A}|\in\{-1,1\}\}\).
Find the number of elements in \(S\).

\end{frame}


\begin{frame}{Data}

\begin{table}[h!]
  \centering
  \begin{tabular}{|c|c|}
\hline
\textbf{Name} & \textbf{Value} \\ \hline
$\vec{A}$ & $\myvec{2 & 1 \\0 & 3}$ \\ \hline
\end{tabular}

  \caption*{Table : Matrix}
  \label{5.13.81}
\end{table}

\end{frame}

\begin{frame}{Solution}

Rearranging the rows of $\vec{A}$

\begin{align}
\myvec{0 & 1 & c\\[2pt] 1 & a & d\\[2pt] 1 & b & e}
\xleftrightarrow{R_2 \leftrightarrow R_3}
\myvec{0 & 1 & c\\1 & b & e\\1 & a & d}
\xleftrightarrow{R_1 \leftrightarrow R_2}
\myvec{1 & b & e\\0 & 1 & c\\1 & a & d}
\end{align}

Applyting row operation to $\vec{A}$ to reduce it into Echelon form
\begin{align}
\myvec{1 & b & e\\0 & 1 & c\\1 & a & d}
\xleftrightarrow{R_3 \to R_3 - R_1}
\myvec{1 & b & e\\0 & 1 & c\\0 & a-b & d-e}
\xleftrightarrow{R_3 \to R_3 - (a-b)R_2}\\
\myvec{
  \makebox[1cm][c]{1} & \makebox[1cm][c]{b} & \makebox[2cm][c]{e} \\
  \makebox[1cm][c]{0} & \makebox[1cm][c]{1} & \makebox[2cm][c]{c} \\
  \makebox[1cm][c]{0} & \makebox[1cm][c]{0} & \makebox[2.6cm][c]{$d-e-c(a-b)$}
}
\end{align}

Finding the determinant by the first column 
\begin{align}
  \mydet{\vec{A}} &= d - e - c(a - b)
\end{align}
Taking cases to find the possibilities of matrix $\vec{A}$

\end{frame}

\begin{frame}{Solution}

Case 1 : $\mydet{\vec{A}}=1$ \\

if c = 0 \\
the value of b and a can be 0 or 1.

\begin{align}
d - e &= 1
\end{align}

So,

\begin{align}
d &= 1 \\  
e &= 0
\end{align}

By permutation we get ,\\

\begin{align}
2 \times 2 \times 1 \times 1 = 4
\end{align}

\end{frame}

\begin{frame}{Solution}

if c = 1, we get 4 possibilities

\begin{align}
d - e -(a - b) &= 1
\end{align}

So,

\begin{align}
  d &= 1 & e&=0 \\   
  b&=a=1 & b&=a=0
\end{align}

\begin{align}
  a &= 0 & b&=1 \\   
  d&=e=1 & d&=e=0
\end{align}

\end{frame}

\begin{frame}{Solution}

Case 2 : $\mydet{\vec{A}}= -1$ \\

if c = 0 \\
the value of b and a can be 0 or 1.

\begin{align}
d - e &= -1
\end{align}

So,

\begin{align}
d &= 0 \\  
e &= 1
\end{align}

By permutation we get ,\\

\begin{align}
2 \times 2 \times 1 \times 1 = 4
\end{align}

\end{frame}

\begin{frame}{Solution}

if c = 1, we get 4 possibilities

\begin{align}
d - e -(a - b) &= -1
\end{align}

So,

\begin{align}
  d &= 0 & e&=1 \\   
  b&=a=1 & b&=a=0
\end{align}

\begin{align}
  a &= 1 & b&=0 \\   
  d&=e=1 & d&=e=0
\end{align}

By adding all the possibilities , we get 

\begin{align}
  4 + 4 + 4 + 4 &= 16
\end{align}

Therefore, the number of elements in \(S=16\).

\end{frame}

\end{document}
