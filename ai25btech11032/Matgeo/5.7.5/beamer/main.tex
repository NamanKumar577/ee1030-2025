\documentclass{beamer}
\mode<presentation>
\usepackage{amsmath}
\usepackage{amssymb}
%\usepackage{advdate}
\usepackage{adjustbox}
\usepackage{subcaption}
\usepackage{enumitem}
\usepackage{multicol}
\usepackage{mathtools}
\usepackage{listings}
\usepackage{float}
\usepackage{graphicx}
\usepackage{url}
\def\UrlBreaks{\do\/\do-}
\usetheme{Boadilla}
\usecolortheme{lily}
\setbeamertemplate{footline}
{
  \leavevmode%
  \hbox{%
  \begin{beamercolorbox}[wd=\paperwidth,ht=2.25ex,dp=1ex,right]{author in head/foot}%
    \insertframenumber{} / \inserttotalframenumber\hspace*{2ex} 
  \end{beamercolorbox}}%
  \vskip0pt%
}
\setbeamertemplate{navigation symbols}{}

\providecommand{\nCr}[2]{\,^{#1}C_{#2}} % nCr
\providecommand{\nPr}[2]{\,^{#1}P_{#2}} % nPr
\providecommand{\mbf}{\mathbf}
\providecommand{\pr}[1]{\ensuremath{\Pr\left(#1\right)}}
\providecommand{\qfunc}[1]{\ensuremath{Q\left(#1\right)}}
\providecommand{\sbrak}[1]{\ensuremath{{}\left[#1\right]}}
\providecommand{\lsbrak}[1]{\ensuremath{{}\left[#1\right.}}
\providecommand{\rsbrak}[1]{\ensuremath{{}\left.#1\right]}}
\providecommand{\brak}[1]{\ensuremath{\left(#1\right)}}
\providecommand{\lbrak}[1]{\ensuremath{\left(#1\right.}}
\providecommand{\rbrak}[1]{\ensuremath{\left.#1\right)}}
\providecommand{\cbrak}[1]{\ensuremath{\left\{#1\right\}}}
\providecommand{\lcbrak}[1]{\ensuremath{\left\{#1\right.}}
\providecommand{\rcbrak}[1]{\ensuremath{\left.#1\right\}}}
\theoremstyle{remark}
\newtheorem{rem}{Remark}
\newcommand{\sgn}{\mathop{\mathrm{sgn}}}
\providecommand{\abs}[1]{\left\vert#1\right\vert}
\providecommand{\res}[1]{\Res\displaylimits_{#1}} 
\providecommand{\norm}[1]{\lVert#1\rVert}
\providecommand{\mtx}[1]{\mathbf{#1}}
\providecommand{\mean}[1]{E\left[ #1 \right]}
\providecommand{\fourier}{\overset{\mathcal{F}}{ \rightleftharpoons}}
%\providecommand{\hilbert}{\overset{\mathcal{H}}{ \rightleftharpoons}}
\providecommand{\system}{\overset{\mathcal{H}}{ \longleftrightarrow}}
	%\newcommand{\solution}[2]{\textbf{Solution:}{#1}}
%\newcommand{\solution}{\noindent \textbf{Solution: }}
\providecommand{\dec}[2]{\ensuremath{\overset{#1}{\underset{#2}{\gtrless}}}}
\newcommand{\myvec}[1]{\ensuremath{\begin{pmatrix}#1\end{pmatrix}}}
\let\vec\mathbf

\lstset{
language=C,
frame=single, 
breaklines=true,
columns=fullflexible
}

\numberwithin{equation}{section}

\title{Presentation - Matgeo}
\author{Aryansingh Sonaye \\
AI25BTECH11032 \\
EE1030 - Matrix Theory}

\date{\today} 
\begin{document}

\begin{frame}
\titlepage
\end{frame}

\section{Problem}
\begin{frame}
\frametitle{Problem Statement}
\textbf{Problem 5.7.5}\\
If 
\begin{align}
A = \myvec{1 & -1 \\ -1 & 1},
\end{align}
find $A^2$.
\end{frame}

\section{Solution}
\subsection{Description of Variables used}
\begin{frame}
\frametitle{Description of Variables used}
\begin{table}[H]
\centering
\begin{tabular}{|c|c|}
\hline
\textbf{Variable} & \textbf{Value} \\
\hline
$A$ & $\myvec{1 & -1 \\ -1 & 1}$ \\
\hline
$\vec{u}$ & $\myvec{1 \\ -1}$ \\
\hline
\end{tabular}
\caption{Input variables}
\end{table}


\end{frame}

\subsection{Theoretical Solution }
\begin{frame}
\frametitle{Theoretical Solution}
\textbf{Method 1: Direct Matrix Multiplication}
\begin{align}
A^2 &= A \cdot A \\
&= \myvec{1 & -1 \\ -1 & 1}\myvec{1 & -1 \\ -1 & 1} \\
&= \myvec{1\cdot 1 + (-1)(-1) & 1\cdot (-1) + (-1)(1) \\ (-1)(1) + 1(-1) & (-1)(-1) + 1\cdot 1} \\
&= \myvec{2 & -2 \\ -2 & 2}
\end{align}

\end{frame}

\begin{frame}
\frametitle{Theoretical Solution}
\textbf{Method 2: Vector--Matrix Representation}
\begin{align}
A &= \vec{u}\vec{u}^T \\
A^2 &= (\vec{u}\vec{u}^T)(\vec{u}\vec{u}^T) \\
&= \vec{u}(\vec{u}^T \vec{u})\vec{u}^T \\
\vec{u}^T \vec{u} &= 1^2 + (-1)^2 = 2 \\
\implies A^2 &= 2\vec{u}\vec{u}^T \\
&= 2A \\
&= \myvec{2 & -2 \\ -2 & 2}
\end{align}

\textbf{Final Answer}
\begin{align}
A^2 = \myvec{2 & -2 \\ -2 & 2}
\end{align}


\end{frame}






\begin{frame}[fragile]
    \frametitle{Code - C}
    \begin{lstlisting}
// All matrices are 2x2, flattened in row-major order:
// [a11, a12,
//  a21, a22]

void mat2_mul(const double *A, const double *B, double *C) {
    // C = A * B
    double a11 = A[0], a12 = A[1], a21 = A[2], a22 = A[3];
    double b11 = B[0], b12 = B[1], b21 = B[2], b22 = B[3];

    C[0] = a11*b11 + a12*b21; // c11
    C[1] = a11*b12 + a12*b22; // c12
    C[2] = a21*b11 + a22*b21; // c21
    C[3] = a21*b12 + a22*b22; // c22
}


    \end{lstlisting}
    \end{frame}

    \begin{frame}[fragile]
    \frametitle{Code - C}
    \begin{lstlisting}
void mat2_square(const double *A, double *C) {
    // C = A^2
    mat2_mul(A, A, C);
}

\end{lstlisting}
\end{frame}

\begin{frame}[fragile]
    \frametitle{Code - Python(with shared C code)}
    The code to obtain the required plot is
    \begin{lstlisting}

import ctypes
import numpy as np

# Load the shared library
lib = ctypes.CDLL("./libmat2.so")

# Define function prototypes
Nd = np.ctypeslib.ndpointer(dtype=np.double, ndim=1, flags="C_CONTIGUOUS")
lib.mat2_square.argtypes = [Nd, Nd]
lib.mat2_square.restype  = None

# Define matrix A (flattened row-major)
A = np.array([1.0, -1.0,
              -1.0,  1.0], dtype=np.double)


\end{lstlisting}
\end{frame}
\begin{frame}[fragile]
\frametitle{Code - Python(with shared C code)}
\begin{lstlisting}
A2 = np.empty(4, dtype=np.double)

# Call C function: A2 = A^2
lib.mat2_square(A, A2)

# Reshape for printing
A_mat  = A.reshape(2, 2)
A2_mat = A2.reshape(2, 2)

print("A =\n", A_mat)
print("\nA^2 (from C) =\n", A2_mat)


\end{lstlisting}
\end{frame}




\begin{frame}[fragile]
\frametitle{Code - Python only}
\begin{lstlisting}
import numpy as np

# Define matrix A
A = np.array([[1, -1],
              [-1, 1]], dtype=float)

# Compute A^2
A2 = A @ A   # or np.dot(A, A)

print("A =\n", A)
print("\nA^2 =\n", A2)




\end{lstlisting}
\end{frame}
\end{document}
