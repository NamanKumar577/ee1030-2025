\let\negmedspace\undefined
\let\negthickspace\undefined
\documentclass[journal]{IEEEtran}
\usepackage[a5paper, margin=10mm, onecolumn]{geometry}
\usepackage{tfrupee} 
\setlength{\headheight}{1cm} 
\setlength{\headsep}{0mm}     

\usepackage{gvv-book}
\usepackage{gvv}
\usepackage{cite}
\usepackage{amsmath,amssymb,amsfonts,amsthm}
\usepackage{algorithmic}
\usepackage{graphicx}
\usepackage{textcomp}
\usepackage{xcolor}
\usepackage{txfonts}
\usepackage{listings}
\usepackage{enumitem}
\usepackage{mathtools}
\usepackage{gensymb}
%\usepackage{wasysym}
\usepackage{comment}
\usepackage[breaklinks=true]{hyperref}
\usepackage{tkz-euclide} 
\usepackage{listings}
\def\inputGnumericTable{}                                 
\usepackage[latin1]{inputenc}                                
\usepackage{color}                                            
\usepackage{array}                                            
\usepackage{longtable}                                       
\usepackage{calc}                                             
\usepackage{multirow}                                         
\usepackage{hhline}                                           
\usepackage{ifthen}                                           
\usepackage{lscape}
\usepackage{circuitikz}
\tikzstyle{block} = [rectangle, draw, fill=blue!20, 
    text width=4em, text centered, rounded corners, minimum height=3em]
\tikzstyle{sum} = [draw, fill=blue!10, circle, minimum size=1cm, node distance=1.5cm]
\tikzstyle{input} = [coordinate]
\tikzstyle{output} = [coordinate]
\renewcommand{\thefigure}{\theenumi}
\renewcommand{\thetable}{\theenumi}
\setlength{\intextsep}{10pt} % Space between text and floats
\numberwithin{equation}{enumi}
\numberwithin{figure}{enumi}
\renewcommand{\thetable}{\theenumi}

\begin{document}

\bibliographystyle{IEEEtran}
\vspace{3cm}

\title{12.670}
\author{EE25BTECH11032 - Kartik Lahoti}
\maketitle

\subsection*{Question: } 
Consider the linear transformation $\vec{T} \colon \mathbb{C}^3 \xrightarrow{} \mathbb{C}^3$ defined by 
\begin{align*}
    \vec{T}\brak{x,y,z} = \brak{x,\frac{\sqrt{3}}{2}y - \frac{1}{2}z , \frac{1}{2}y+\frac{\sqrt{3}}{2}z}
\end{align*}

where $\mathbb{C}$ is the set of all complex numbers and $\mathbb{C}^3 = \mathbb{C}\times\mathbb{C}\times\mathbb{C} $. Which of the following statements is TRUE?

\begin{enumerate}
        \item There exists a non-zero vector $\vec{X}$ such that $\vec{T}\brak{\vec{X}} = -\vec{X}$
        \item There exists a non-zero vector $\vec{Y}$ and a real number  $\lambda \neq 1$  such that $\vec{T}\brak{\vec{Y}} = \lambda\vec{Y}$
        \item $\vec{T}$ is diagonalizable
        \item $\vec{T}^2 = \vec{I}_3$ , where $\vec{I}_3$ us the $3\times 3$ identity matrix
    \end{enumerate}
\textbf{Solution}:\\

Let this function be written as 

\begin{align}
    \vec{y} = \vec{T}\vec{x}
\end{align}
where , $\vec{x} $ and $\vec{y}$ are complex vector in $\mathbb{C}^3$

Now, 
\begin{align}
    \vec{T}\vec{e_1} = \myvec{1\\ 0 \\ 0} 
\end{align}
\begin{align}
    \vec{T}\vec{e_2}  = \myvec{0\\ \frac{\sqrt{3}}{2} \\ \frac{1}{2}} 
\end{align}
\begin{align}
    \vec{T}\vec{e_3}  = \myvec{0\\ \frac{-1}{2} \\ \frac{\sqrt{3}}{2}} 
\end{align}

\begin{align}
    \therefore \vec{T} = \myvec{1&0&0\\0&\frac{\sqrt{3}}{2}&\frac{-1}{2}\\ 0&\frac{1}{2}&\frac{\sqrt{3}}{2}}
\end{align}

Now, 

\begin{align}
    \mydet{\lambda\vec{I}-\vec{T}} &= \mydet{\lambda-1 & 0 & 0 \\ 0 & \lambda-\frac{\sqrt{3}}{2}& \frac{1}{2} \\ 0 & \frac{-1}{2} & \lambda-\frac{\sqrt{3}}{2}}\\
    &= \brak{\lambda-1}\brak{\brak{\lambda-\frac{\sqrt{3}}{2}}^2 + \frac{1}{4}}
\end{align}
This gives eigen values as
\begin{align}
    \lambda_1 = 1 , \lambda_2 = \frac{\sqrt{3}}{2} + i\frac{1}{2} , \lambda_3 =\frac{\sqrt{3}}{2} - i\frac{1}{2}
\end{align}

Option $1$ : INCORRECT . $\because$ No Eigen Value $= -1$

Option $2$ : INCORRECT . $\because$ No Eigen Value other than $1$ is real. 

Option $3$ : CORRECT.    $\because$ All eigen values are distinct! 

Hence , Answer : Option $\brak{3}$

\end{document}


