\let\negmedspace\undefined
\let\negthickspace\undefined
\documentclass[journal]{IEEEtran}
\usepackage[a5paper, margin=10mm, onecolumn]{geometry}
\usepackage{tfrupee} 
\setlength{\headheight}{1cm} 
\setlength{\headsep}{0mm}     

\usepackage{gvv-book}
\usepackage{gvv}
\usepackage{cite}
\usepackage{amsmath,amssymb,amsfonts,amsthm}
\usepackage{algorithmic}
\usepackage{graphicx}
\usepackage{textcomp}
\usepackage{xcolor}
\usepackage{txfonts}
\usepackage{listings}
\usepackage{enumitem}
\usepackage{mathtools}
\usepackage{gensymb}
%\usepackage{wasysym}
\usepackage{comment}
\usepackage[breaklinks=true]{hyperref}
\usepackage{tkz-euclide} 
\usepackage{listings}
\def\inputGnumericTable{}                                 
\usepackage[latin1]{inputenc}                                
\usepackage{color}                                            
\usepackage{array}                                            
\usepackage{longtable}                                       
\usepackage{calc}                                             
\usepackage{multirow}                                         
\usepackage{hhline}                                           
\usepackage{ifthen}                                           
\usepackage{lscape}
\usepackage{circuitikz}
\tikzstyle{block} = [rectangle, draw, fill=blue!20, 
    text width=4em, text centered, rounded corners, minimum height=3em]
\tikzstyle{sum} = [draw, fill=blue!10, circle, minimum size=1cm, node distance=1.5cm]
\tikzstyle{input} = [coordinate]
\tikzstyle{output} = [coordinate]
\renewcommand{\thefigure}{\theenumi}
\renewcommand{\thetable}{\theenumi}
\setlength{\intextsep}{10pt} % Space between text and floats
\numberwithin{equation}{enumi}
\numberwithin{figure}{enumi}
\renewcommand{\thetable}{\theenumi}

\begin{document}

\bibliographystyle{IEEEtran}
\vspace{3cm}

\title{5.4.27}
\author{EE25BTECH11032 - Kartik Lahoti}
\maketitle

\subsection*{Question: } 
Using elementary transformations, find the inverse of the following matrix.
\begin{align*}
    \myvec{2 & 0 & -1 \\ 5 & 1 & 0 \\ 0 & 1 & 3}
\end{align*}

\textbf{Solution}:\\
Given the matrix,
\begin{align}
    \vec{A} = \myvec{2 & 0 & -1 \\ 5 & 1 & 0 \\ 0 & 1 & 3}
\end{align}

Let $\vec{A}^{-1}$ be the inverse of the matrix $\vec{A}$

We know that,

\begin{align}
\vec{A}\vec{A}^{-1} = \vec{I}
\end{align}

The augmented matrix of $\augvec{1}{1}{\vec{A} & \vec{I}}$ is given by , 

\begin{align}
    \augvec{3}{3}{2 & 0 & -1 & 1 & 0 & 0\\ 5 & 1 & 0 & 0 & 1 & 0 \\ 0 & 1 & 3 & 0 & 0 & 1}
\end{align}

\begin{align}
    \augvec{3}{3}{2 & 0 & -1 & 1 & 0 & 0\\ 5 & 1 & 0 & 0 & 1 & 0 \\ 0 & 1 & 3 & 0 & 0 & 1}\xleftrightarrow[]{R_1\rightarrow\frac{1}{2}R_1}\augvec{3}{3}{1 & 0 & \frac{-1}{2}& \frac{1}{2} & 0 & 0\\ 5 & 1 & 0 & 0 & 1 & 0 \\ 0 & 1 & 3 & 0 & 0 & 1}
\end{align}
\begin{align}
    \augvec{3}{3}{1 & 0 & \frac{-1}{2}& \frac{1}{2} & 0 & 0\\ 5 & 1 & 0 & 0 & 1 & 0 \\ 0 & 1 & 3 & 0 & 0 & 1}\xleftrightarrow[]{R_2\rightarrow R_2 - 5R_1}\augvec{3}{3}{1 & 0 & \frac{-1}{2}& \frac{1}{2} & 0 & 0\\ 0 & 1 & \frac{5}{2} & \frac{-5}{2} & 1 & 0 \\ 0 & 1 & 3 & 0 & 0 & 1}
\end{align}
\begin{align}
    \augvec{3}{3}{1 & 0 & \tfrac{-1}{2}& \tfrac{1}{2} & 0 & 0\\ 0 & 1 & \tfrac{5}{2} & \tfrac{-5}{2} & 1 & 0 \\ 0 & 1 & 3 & 0 & 0 & 1}\xleftrightarrow[]{R_3\rightarrow R_3 - R_2 }\augvec{3}{3}{1 & 0 & \tfrac{-1}{2}& \tfrac{1}{2} & 0 & 0\\ 0 & 1 & \tfrac{5}{2} & \tfrac{-5}{2} & 1 & 0 \\ 0 & 0 & \tfrac{1}{2} & \tfrac{5}{2} & -1 & 1}
\end{align}
\begin{align}
    \augvec{3}{3}{1 & 0 & \tfrac{-1}{2}& \tfrac{1}{2} & 0 & 0\\ 0 & 1 & \tfrac{5}{2} & \tfrac{-5}{2} & 1 & 0 \\ 0 & 0 & \tfrac{1}{2} & \tfrac{5}{2} & -1 & 1}\xleftrightarrow[]{R_3\rightarrow 2R_3}\augvec{3}{3}{1 & 0 & \tfrac{-1}{2}& \tfrac{1}{2} & 0 & 0\\ 0 & 1 & \tfrac{5}{2} & \tfrac{-5}{2} & 1 & 0 \\ 0 & 0 & 1 & 5 & -2 & 2}
\end{align}
\begin{align}
    \augvec{3}{3}{1 & 0 & \tfrac{-1}{2}& \tfrac{1}{2} & 0 & 0\\ 0 & 1 & \tfrac{5}{2} & \tfrac{-5}{2} & 1 & 0 \\ 0 & 0 & 1 & 5 & -2 & 2}\xleftrightarrow[]{R_2\rightarrow R_2 - \frac{5}{2}R_3 }\augvec{3}{3}{1 & 0 & \tfrac{-1}{2}& \tfrac{1}{2} & 0 & 0\\ 0 & 1 & 0 & -15 & 6 & -5 \\ 0 & 0 & 1 & 5 & -2 & 2}
\end{align}
\begin{align}
    \augvec{3}{3}{1 & 0 & \tfrac{-1}{2}& \tfrac{1}{2} & 0 & 0\\ 0 & 1 & 0 & -15 & 6 & -5 \\ 0 & 0 & 1 & 5 & -2 & 2}\xleftrightarrow[]{R_1\rightarrow R_1 + \frac{1}{2}R_3}\augvec{3}{3}{1 & 0 & 0 & 3 & -1 & 1\\ 0 & 1 & 0 & -15 & 6 & -5 \\ 0 & 0 & 1 & 5 & -2 & 2}
\end{align}
Hence ,
\begin{align}
    \vec{A}^{-1} = \myvec{3 & -1 & 1 \\ -15 & 6 & -5 \\ 5 & -2 & 2}
\end{align}
\end{document}


