\let\negmedspace\undefined
\let\negthickspace\undefined
\documentclass[journal]{IEEEtran}
\usepackage[a5paper, margin=10mm, onecolumn]{geometry}
\usepackage{tfrupee} 
\setlength{\headheight}{1cm} 
\setlength{\headsep}{0mm}     

\usepackage{gvv-book}
\usepackage{gvv}
\usepackage{cite}
\usepackage{amsmath,amssymb,amsfonts,amsthm}
\usepackage{algorithmic}
\usepackage{graphicx}
\usepackage{textcomp}
\usepackage{xcolor}
\usepackage{txfonts}
\usepackage{listings}
\usepackage{enumitem}
\usepackage{mathtools}
\usepackage{gensymb}
%\usepackage{wasysym}
\usepackage{comment}
\usepackage[breaklinks=true]{hyperref}
\usepackage{tkz-euclide} 
\usepackage{listings}
\def\inputGnumericTable{}                                 
\usepackage[latin1]{inputenc}                                
\usepackage{color}                                            
\usepackage{array}                                            
\usepackage{longtable}                                       
\usepackage{calc}                                             
\usepackage{multirow}                                         
\usepackage{hhline}                                           
\usepackage{ifthen}                                           
\usepackage{lscape}
\usepackage{circuitikz}
\tikzstyle{block} = [rectangle, draw, fill=blue!20, 
    text width=4em, text centered, rounded corners, minimum height=3em]
\tikzstyle{sum} = [draw, fill=blue!10, circle, minimum size=1cm, node distance=1.5cm]
\tikzstyle{input} = [coordinate]
\tikzstyle{output} = [coordinate]
\renewcommand{\thefigure}{\theenumi}
\renewcommand{\thetable}{\theenumi}
\setlength{\intextsep}{10pt} % Space between text and floats
%\numberwithin{equation}{enumi}
\numberwithin{figure}{enumi}
\renewcommand{\thetable}{\theenumi}

\begin{document}

\bibliographystyle{IEEEtran}
\vspace{3cm}

\title{12.358}
\author{EE25BTECH11032 - Kartik Lahoti}
\maketitle

\subsection*{Question: } 
The inverse of the $2\times 2 $ matrix.
\begin{align*}
    \myvec{1 & 2 \\ 5 & 7}
\end{align*}

\begin{enumerate}
    \begin{multicols}{4}
        \item $
                \frac{1}{3}\myvec{-7 & 2 \\ 5 & -1}
               $
        \item $
                \frac{1}{3}\myvec{7 & 2 \\ 5 & 1}
               $
        \item $
                \frac{1}{3}\myvec{7 & -2 \\ -5 & 1}
                $
        \item $
                \frac{1}{3}\myvec{-7 & -2 \\ -5 & -1}
                $
        
    \end{multicols}
\end{enumerate}

\textbf{Solution}:\\
Given the matrix,
\begin{align}
    \vec{A} = \myvec{1 & 2 \\ 5 & 7}
\end{align}

Let $\vec{A}^{-1}$ be the inverse of the matrix $\vec{A}$

We know that,

\begin{align}
\vec{A}\vec{A}^{-1} = \vec{I}
\end{align}

The augmented matrix of $\augvec{1}{1}{\mathbf{A} & \mathbf{I}}$ is given by
\begin{align}
\augvec{2}{2}{1 & 2 & 1 & 0\\ 5 & 7 & 0 & 1}
\end{align}

\begin{align}
\augvec{2}{2}{1 & 2 & 1 & 0\\ 5 & 7 & 0 & 1} \xleftrightarrow[]{R_2 \rightarrow R_2 - 5R_1} 
\augvec{2}{2}{1 & 2 & 1 & 0\\ 0 & -3 & -5 & 1}
\end{align}

\begin{align}
\augvec{2}{2}{1 & 2 & 1 & 0\\ 0 & -3 & -5 & 1} \xleftrightarrow[]{R_2 \rightarrow -\frac{1}{3} R_2} 
\augvec{2}{2}{1 & 2 & 1 & 0\\ 0 & 1 & \frac{5}{3} & -\frac{1}{3}}
\end{align}

\begin{align}
\augvec{2}{2}{1 & 2 & 1 & 0\\ 0 & 1 & \frac{5}{3} & -\frac{1}{3}} \xleftrightarrow[]{R_1 \rightarrow R_1 - 2 R_2} 
\augvec{2}{2}{1 & 0 & -\frac{7}{3} & \frac{2}{3}\\ 0 & 1 & \frac{5}{3} & -\frac{1}{3}}
\end{align}

Hence,
\begin{align}
\mathbf{A}^{-1} = \frac{1}{3}\myvec{-7 & 2 \\ 5 & -1}
\end{align}

Answer: Option $\brak{2}$

\end{document}


