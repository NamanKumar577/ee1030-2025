\documentclass{beamer}
\usepackage{amsmath}
\usepackage{gvv}

\title{Question 5.13.26}
\author{AI25BTECH11040 - Vivaan Parashar}
\date{\today}

\begin{document}

\frame{\titlepage}

\begin{frame}
    \frametitle{Question: }
    If $\vec{A}$ and $\vec{B}$ are square matrices of size $n \times n$ such that $\vec{A}^2 - \vec{B}^2 = (\vec{A} - \vec{B}) (\vec{A} + \vec{B})$, then which of the following will be always true?
    \begin{enumerate}[label=(\alph*)]
        \item $\vec{A} = \vec{B}$
        \item $\vec{AB} = \vec{BA}$
        \item either of $\vec{A}$ or $\vec{B}$ is a zero matrix
        \item either $\vec{A}$ or $\vec{B}$ is an identity matrix
    \end{enumerate}
\end{frame}

\begin{frame}
    \frametitle{Solution: }
    We know that for any two square matrices $\vec{A}$ and $\vec{B}$ of size $n \times n$, the following is true:
    \begin{align}
        (\vec{A} - \vec{B}) (\vec{A} + \vec{B}) & = (\vec{A} - \vec{B})\vec{A} + (\vec{A} - \vec{B})\vec{B} \\
                                                & = \vec{A}^2 - \vec{BA} + \vec{AB} - \vec{B}^2 \label{eq1}
    \end{align}
    Given that $\vec{A}^2 - \vec{B}^2 = (\vec{A} - \vec{B}) (\vec{A} + \vec{B})$, we can use equation \ref{eq1}:
    \begin{align}
        \vec{A}^2 - \vec{B}^2 & = \vec{A}^2 - \vec{BA} + \vec{AB} - \vec{B}^2 \\
        \implies 0            & = - \vec{BA} + \vec{AB}                       \\
        \implies \vec{BA}     & = \vec{AB}
    \end{align}
    Thus, option (b) is always true.
\end{frame}

\end{document}
