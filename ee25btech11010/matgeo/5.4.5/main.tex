\let\negmedspace\undefined
\let\negthickspace\undefined
\documentclass[journal]{IEEEtran}
\usepackage[a5paper, margin=10mm, onecolumn]{geometry}
%\usepackage{lmodern} % Ensure lmodern is loaded for pdflatex
\usepackage{tfrupee} % Include tfrupee package

\setlength{\headheight}{1cm} % Set the height of the header box
\setlength{\headsep}{0mm}     % Set the distance between the header box and the top of the text

\usepackage{gvv-book}
\usepackage{gvv}
\usepackage{cite}
\usepackage{amsmath,amssymb,amsfonts,amsthm}
\usepackage{algorithmic}
\usepackage{graphicx}
\usepackage{textcomp}
\usepackage{xcolor}
%\usepackage{txfonts}
\usepackage{listings}
\usepackage{enumitem}
\usepackage{mathtools}
\usepackage{gensymb}
\usepackage{comment}
\usepackage[breaklinks=true]{hyperref}
\usepackage{tkz-euclide} 
\usepackage{listings}
% \usepackage{gvv}                                        
\def\inputGnumericTable{}                                 
\usepackage[latin1]{inputenc}                                
\usepackage{color}                                            
\usepackage{array}                                            
\usepackage{longtable}                                       
\usepackage{calc}                                             
\usepackage{multirow}                                         
\usepackage{hhline}                                           
\usepackage{ifthen}                                           
\usepackage{lscape}
\usepackage{circuitikz}
\tikzstyle{block} = [rectangle, draw, fill=blue!20, 
    text width=4em, text centered, rounded corners, minimum height=3em]
\tikzstyle{sum} = [draw, fill=blue!10, circle, minimum size=1cm, node distance=1.5cm]
\tikzstyle{input} = [coordinate]
\tikzstyle{output} = [coordinate]


\begin{document}

\bibliographystyle{IEEEtran}
\vspace{3cm}

\title{5.4.5}
\author{EE25BTECH11010 - Arsh Dhoke}
\maketitle
{\let\newpage\relax\maketitle}

\renewcommand{\thefigure}{\theenumi}
\renewcommand{\thetable}{\theenumi}
\setlength{\intextsep}{10pt} % Space between text and floats

\numberwithin{equation}{enumi}
\numberwithin{figure}{enumi}
\renewcommand{\thetable}{\theenumi}

\textbf{Question}:\\
Using elementary transformations, find the inverse of  the following matrix: 

\myvec{1 & 2 \\ 2 & 1}. \\

\solution \\
We know
\begin{align}
\vec{A}^{-1}\vec{A} = \vec{I} 
\end{align}
where $\vec{I}$ is the identity matrix  $\vec{I}_2$ \\ 

The augmented matrix for the given matrix will be
\begin{align}
\augvec{2}{4}{2 & 1 & 1 & 0 \\ 1 & 2 & 0 & 1}
&\xleftrightarrow[R_2\to R_2-2R_1]{R_1\leftrightarrow R_2}
\augvec{2}{4}{1 & 2 & 0 & 1 \\ 0 & -3 & 1 & -2} \\
&\xleftrightarrow[R_1\to R_1-2R_2]{R_2\to -\frac{1}{3}R_2}
\augvec{2}{4}{1 & 0 & \frac{2}{3} & -\frac{1}{3} \\ 0 & 1 & -\frac{1}{3} & \frac{2}{3}}
\end{align}


\begin{align}
\therefore \quad 
\vec{A}^{-1} &= \myvec{\frac{2}{3} & -\frac{1}{3} \\ -\frac{1}{3} & \frac{2}{3}} \\
\vec{A}^{-1}&= \frac{1}{3}\myvec{2 & -1 \\ -1 & 2}
\end{align}

We can verify the computed inverse using python code by showing 
$\vec{A}^{-1}\vec{A} = \vec{I}$. 
\end{document}