\documentclass{beamer}
\usepackage[utf8]{inputenc}

\usetheme{Madrid}
\usecolortheme{default}
\usepackage{amsmath,amssymb,amsfonts,amsthm}
\usepackage{txfonts}
\usepackage{tkz-euclide}
\usepackage{listings}
\usepackage{adjustbox}
\usepackage{array}
\usepackage{tabularx}
\usepackage{gvv}
\usepackage{lmodern}
\usepackage{circuitikz}
\usepackage{tikz}
\usepackage{graphicx}
\usepackage{amsmath}

\setbeamertemplate{page number in head/foot}[totalframenumber]

\usepackage{tcolorbox}
\tcbuselibrary{minted,breakable,xparse,skins}



\definecolor{bg}{gray}{0.95}
\DeclareTCBListing{mintedbox}{O{}m!O{}}{%
  breakable=true,
  listing engine=minted,
  listing only,
  minted language=#2,
  minted style=default,
  minted options={%
    linenos,
    gobble=0,
    breaklines=true,
    breakafter=,,
    fontsize=\small,
    numbersep=8pt,
    #1},
  boxsep=0pt,
  left skip=0pt,
  right skip=0pt,
  left=25pt,
  right=0pt,
  top=3pt,
  bottom=3pt,
  arc=5pt,
  leftrule=0pt,
  rightrule=0pt,
  bottomrule=2pt,
  toprule=2pt,
  colback=bg,
  colframe=orange!70,
  enhanced,
  overlay={%
    \begin{tcbclipinterior}
    \fill[orange!20!white] (frame.south west) rectangle ([xshift=20pt]frame.north west);
    \end{tcbclipinterior}},
  #3,
}
\lstset{
    language=C,
    basicstyle=\ttfamily\small,
    keywordstyle=\color{blue},
    stringstyle=\color{orange},
    commentstyle=\color{green!60!black},
    numbers=left,
    numberstyle=\tiny\color{gray},
    breaklines=true,
    showstringspaces=false,
}


\title 
{5.6.7}
\date{September 28,2025}


\author 
{Abhiram Reddy-AI25BTECH11021}



\begin{document}

\frame{\titlepage}


\begin{frame}
\frametitle{Question}
Let
\[
A=\begin{pmatrix}
2 & -1 & 1\\
-1 & 2 & -1\\
1 & -1 & 2
\end{pmatrix}.
\]
Verify that
\begin{equation}\label{eq:CH}
A^{3}-6A^{2}+9A-4I=0,
\end{equation}
by using the Cayley--Hamilton theorem, and hence find \(A^{-1}\).
\end{frame}

\begin{frame}
\frametitle{Step 1: Characteristic Polynomial}
\begin{align}
\chi_A(\lambda)&=\det(\lambda I - A)\notag\\
&=\det\begin{pmatrix}
\lambda-2 & 1 & -1\\
1 & \lambda-2 & 1\\
-1 & 1 & \lambda-2
\end{pmatrix}. \label{eq:char-det}
\end{align}
Expanding gives
\begin{equation}\label{eq:char-factor}
\chi_A(\lambda)=(\lambda-4)(\lambda-1)^{2}
=\lambda^{3}-6\lambda^{2}+9\lambda-4.
\end{equation}
\end{frame}

\begin{frame}
\frametitle{Step 2: Cayley--Hamilton Theorem}
By Cayley--Hamilton, \(A\) satisfies its characteristic equation:
\begin{equation}\label{eq:CH-matrix}
A^{3}-6A^{2}+9A-4I=0,
\end{equation}
which proves the required identity.
\end{frame}

\begin{frame}
\frametitle{Step 3: Formula for \(A^{-1}\)}
Multiplying \eqref{eq:CH-matrix} on the right by \(A^{-1}\):
\begin{equation}\label{eq:pre-inv}
A^{2}-6A+9I-4A^{-1}=0.
\end{equation}
Thus,
\begin{equation}\label{eq:inv-formula}
A^{-1}=\tfrac{1}{4}\bigl(A^{2}-6A+9I\bigr).
\end{equation}
\end{frame}

\begin{frame}
\frametitle{Step 4: Determinant and Adjugate}
From \eqref{eq:char-factor}, eigenvalues are \(4,1,1\). Hence
\begin{equation}\label{eq:det}
\det(A)=4.
\end{equation}
The adjugate matrix:
\[
\operatorname{adj}(A)=\det(A)\,A^{-1}=4A^{-1}.
\]
So,
\begin{equation}\label{eq:inv-final}
A^{-1}=\frac{1}{4}
\begin{pmatrix}
3 & 1 & -1\\
1 & 3 & 1\\
-1 & 1 & 3
\end{pmatrix}.
\end{equation}
\end{frame}

% ---------------- C CODE -----------------
\begin{frame}[fragile]
\frametitle{C Code (Part 1)}
\begin{lstlisting}[language=C]
#include <stdio.h>

int main() {
    int i, j, k;
    int A[3][3] = {
        {2, -1, 1},
        {-1, 2, -1},
        {1, -1, 2}
    };
    
    int A2[3][3] = {0};
    int temp[3][3] = {0};
    float Ainv[3][3];

    for(i = 0; i < 3; i++) {
        for(j = 0; j < 3; j++) {
            for(k = 0; k < 3; k++) {
                A2[i][j] += A[i][k] * A[k][j];
            }
        }
    }
\end{lstlisting}
\end{frame}

\begin{frame}[fragile]
\frametitle{C Code (Part 2)}
\begin{lstlisting}[language=C]
    for(i = 0; i < 3; i++) {
        for(j = 0; j < 3; j++) {
            temp[i][j] = A2[i][j] - 6*A[i][j];
            if(i == j) temp[i][j] += 9;
        }
    }

    for(i = 0; i < 3; i++) {
        for(j = 0; j < 3; j++) {
            Ainv[i][j] = temp[i][j] / 4.0;
        }
    }

    printf("The inverse matrix A^{-1} is:\n");
    for(i = 0; i < 3; i++) {
        for(j = 0; j < 3; j++) {
            printf("%6.2f ", Ainv[i][j]);
        }
        printf("\n");
    }
    return 0;
}
\end{lstlisting}
\end{frame}

% ---------------- PYTHON CODE -----------------
\begin{frame}[fragile]
\frametitle{Python Code (Part 1)}
\begin{lstlisting}[language=Python]
import numpy as np

A = np.array([
    [2, -1, 1],
    [-1, 2, -1],
    [1, -1, 2]
], dtype=float)

# Compute A^2
A2 = np.dot(A, A)

# Compute A^2 - 6A + 9I
temp = A2 - 6*A + 9*np.eye(3)

# Divide by determinant (4) to get inverse
Ainv = temp / 4.0

print("The inverse matrix A^{-1} is:")
print(Ainv)

# Verify by multiplying A * A^{-1}
print("Check A * A^{-1} = I:")
print(np.dot(A, Ainv))
\end{lstlisting}
\end{frame}


\end{document}