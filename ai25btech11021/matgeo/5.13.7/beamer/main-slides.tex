\documentclass{beamer}
\usepackage[utf8]{inputenc}

\usetheme{Madrid}
\usecolortheme{default}
\usepackage{amsmath,amssymb,amsfonts,amsthm}
\usepackage{txfonts}
\usepackage{tkz-euclide}
\usepackage{listings}
\usepackage{adjustbox}
\usepackage{array}
\usepackage{tabularx}
\usepackage{gvv}
\usepackage{lmodern}
\usepackage{circuitikz}
\usepackage{tikz}
\usepackage{graphicx}
\usepackage{amsmath}

\setbeamertemplate{page number in head/foot}[totalframenumber]

\usepackage{tcolorbox}
\tcbuselibrary{minted,breakable,xparse,skins}



\definecolor{bg}{gray}{0.95}
\DeclareTCBListing{mintedbox}{O{}m!O{}}{%
  breakable=true,
  listing engine=minted,
  listing only,
  minted language=#2,
  minted style=default,
  minted options={%
    linenos,
    gobble=0,
    breaklines=true,
    breakafter=,,
    fontsize=\small,
    numbersep=8pt,
    #1},
  boxsep=0pt,
  left skip=0pt,
  right skip=0pt,
  left=25pt,
  right=0pt,
  top=3pt,
  bottom=3pt,
  arc=5pt,
  leftrule=0pt,
  rightrule=0pt,
  bottomrule=2pt,
  toprule=2pt,
  colback=bg,
  colframe=orange!70,
  enhanced,
  overlay={%
    \begin{tcbclipinterior}
    \fill[orange!20!white] (frame.south west) rectangle ([xshift=20pt]frame.north west);
    \end{tcbclipinterior}},
  #3,
}
\lstset{
    language=C,
    basicstyle=\ttfamily\small,
    keywordstyle=\color{blue},
    stringstyle=\color{orange},
    commentstyle=\color{green!60!black},
    numbers=left,
    numberstyle=\tiny\color{gray},
    breaklines=true,
    showstringspaces=false,
}


\title 
{5.13.7}
\date{September 28,2025}


\author 
{Abhiram Reddy-AI25BTECH11021}



\begin{document}


\frame{\titlepage}

\begin{frame}
\frametitle{Question 5.13.7}
Let $A$ and $B$ be two symmetric matrices of order $3$.  

\textbf{Statement 1:} $A(BA)$ and $(AB)A$ are symmetric matrices.  

\textbf{Statement 2:} $AB$ is symmetric if $A$ and $B$ commute ($AB=BA$).  

Determine the correct option.
\end{frame}

\begin{frame}
\frametitle{Step 1: Symmetry property}
A matrix $M$ is symmetric if
\begin{equation}
M^T = M
\end{equation}
\end{frame}

\begin{frame}
\frametitle{Step 2: Check $A(BA)$}
Let 
\[
M = A(BA)
\]  
Then,
\begin{align}
M^T &= (A(BA))^T \\
&= (BA)^T A^T \\
&= A^T B^T A^T \\
&= A B A \quad \text{(since $A^T=A$, $B^T=B$)} \\
&= A(BA) = M
\end{align}
Hence, $A(BA)$ is symmetric.
\end{frame}

\begin{frame}
\frametitle{Step 3: Check $(AB)A$}
Let 
\[
N = (AB)A
\]  
Then,
\begin{align}
N^T &= ((AB)A)^T \\
&= A^T (AB)^T \\
&= A^T B^T A^T \\
&= ABA \\
&= (AB)A = N
\end{align}
Hence, $(AB)A$ is symmetric.
\end{frame}

\begin{frame}
\frametitle{Step 4: Condition for $AB$ to be symmetric}
\begin{align}
(AB)^T &= B^T A^T \\
&= BA \quad \text{(since $A^T=A$, $B^T=B$)}
\end{align}
Thus,
\begin{equation}
AB \text{ is symmetric } \iff AB = BA
\end{equation}

Conclusion: Statement 1 is true, Statement 2 is true but not the correct explanation for Statement 1.
\[
\boxed{\text{Correct option: (a)}}
\]
\end{frame}

% ---------------- C CODE -----------------
\begin{frame}[fragile]
\frametitle{C Code (Part 1)}
\begin{lstlisting}[language=C]
#include <stdio.h>

int main() {
    int A[3][3] = {{1,2,3},{4,5,6},{7,8,9}};
    int B[3][3] = {{9,8,7},{6,5,4},{3,2,1}};
    int AB[3][3], ABt[3][3], BtAt[3][3], i, j, k;

    for(i=0;i<3;i++){
        for(j=0;j<3;j++){
            AB[i][j]=0;
            for(k=0;k<3;k++){
                AB[i][j]+=A[i][k]*B[k][j];
            }
        }
    }

    for(i=0;i<3;i++){
        for(j=0;j<3;j++){
            ABt[i][j]=AB[j][i];
        }
    }
\end{lstlisting}
\end{frame}

\begin{frame}[fragile]
\frametitle{C Code (Part 2)}
\begin{lstlisting}[language=C]
    int At[3][3], Bt[3][3];
    for(i=0;i<3;i++){
        for(j=0;j<3;j++){
            At[i][j]=A[j][i];
            Bt[i][j]=B[j][i];
        }
    }

    for(i=0;i<3;i++){
        for(j=0;j<3;j++){
            BtAt[i][j]=0;
            for(k=0;k<3;k++){
                BtAt[i][j]+=Bt[i][k]*At[k][j];
            }
        }
    }

    printf("Transpose of AB:\n");
    for(i=0;i<3;i++){
        for(j=0;j<3;j++){
            printf("%d ",ABt[i][j]);
        }
        printf("\n");
    }

    printf("Product B^T A^T:\n");
    for(i=0;i<3;i++){
        for(j=0;j<3;j++){
            printf("%d ",BtAt[i][j]);
        }
        printf("\n");
    }
    return 0;
}
\end{lstlisting}
\end{frame}

% ---------------- PYTHON CODE -----------------
\begin{frame}[fragile]
\frametitle{Python Code}
\begin{lstlisting}[language=Python]
import numpy as np

A = np.array([[1,2,3],[4,5,6],[7,8,9]])
B = np.array([[9,8,7],[6,5,4],[3,2,1]])

AB = np.dot(A, B)
ABt = AB.T
BtAt = np.dot(B.T, A.T)

print("Transpose of AB:")
print(ABt)
print("Product B^T A^T:")
print(BtAt)
\end{lstlisting}
\end{frame}

\end{document}