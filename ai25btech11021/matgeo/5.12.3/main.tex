\let\negmedspace\undefined
\let\negthickspace\undefined
\documentclass[journal]{IEEEtran}
\usepackage[a5paper, margin=10mm, onecolumn]{geometry}
\usepackage{tfrupee}
\setlength{\headheight}{1cm}
\setlength{\headsep}{0mm}
\usepackage{gvv-book}
\usepackage{gvv}
\usepackage{cite}
\usepackage{amsmath,amssymb,amsfonts,amsthm}
\usepackage{algorithmic}
\usepackage{graphicx}
\usepackage{textcomp}
\usepackage{xcolor}
\usepackage{txfonts}
\usepackage{listings}
\usepackage{enumitem}
\usepackage{mathtools}
\usepackage{gensymb}
\usepackage{comment}
\usepackage[breaklinks=true]{hyperref}
\usepackage{multicol}

\begin{document}

\bibliographystyle{IEEEtran}
\vspace{2cm}

\title{5.12.3}
\author{AI25BTECH11008 - Chiruvella Harshith Sharan}
{\let\newpage\relax\maketitle}

\textbf{Question:} \\

Write the number of all possible matrices of order \(2\times 2\) with each entry \(1,2\) or \(3\).

\vspace{0.3cm}
\textbf{Solution:}

\subsection*{Step 1: Understand the problem}
A \(2\times2\) matrix has four entries. Each entry may independently be chosen from the set \(\{1,2,3\}\). We are asked to count all possible such matrices (order matters — different entries or positions give different matrices).

\subsection*{Step 2: Count choices per entry}
Each of the four positions (row 1 col 1, row 1 col 2, row 2 col 1, row 2 col 2) has exactly \(3\) possible values.

\subsection*{Step 3: Use the product rule}
By the rule of product (each position chosen independently),
\[
\text{number of matrices} = 3\times 3\times 3\times 3 = 3^4.
\]

\subsection*{Step 4: Compute}
\[
3^2 = 9,\qquad 3^3 = 27,\qquad 3^4 = 81.
\]

\subsection*{Final Answer:}
\[
\boxed{81}
\]


\end{document}
