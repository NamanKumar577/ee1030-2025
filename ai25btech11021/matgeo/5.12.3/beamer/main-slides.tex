\documentclass{beamer}
\usepackage[utf8]{inputenc}

\usetheme{Madrid}
\usecolortheme{default}
\usepackage{amsmath,amssymb,amsthm}
\usepackage{txfonts}
\usepackage{tikz}
\usepackage{listings}
\usepackage{graphicx}
\usepackage{lmodern}

\setbeamertemplate{page number in head/foot}[totalframenumber]

\lstset{
    language=C,
    basicstyle=\ttfamily\small,
    keywordstyle=\color{blue},
    stringstyle=\color{orange},
    commentstyle=\color{green!60!black},
    numbers=left,
    numberstyle=\tiny\color{gray},
    breaklines=true,
    showstringspaces=false,
}

\title{5.12.3}
\author{Chiruvella Harshith Sharan - AI25BTECH11008}

\begin{document}

\frame{\titlepage}

%------------------------------------

\begin{frame}{Question}
Write the number of all possible matrices of order \(2\times 2\) with each entry \(1,2\) or \(3\).
\end{frame}

%------------------------------------

\begin{frame}{Step 1: Matrix Structure}
A \(2\times 2\) matrix has four entries:
\[
A = \begin{bmatrix}
a & b \\
c & d
\end{bmatrix}.
\]
Each entry \(a,b,c,d\) can independently be chosen from the set \(\{1,2,3\}\).
\end{frame}

%------------------------------------

\begin{frame}{Step 2: Counting Choices}
\begin{itemize}
    \item Each entry has 3 possible values.
    \item Since there are 4 entries, total number of matrices is
    \[
    3 \times 3 \times 3 \times 3 = 3^4.
    \]
\end{itemize}
\end{frame}

%------------------------------------

\begin{frame}{Step 3: Final Answer}
\[
3^4 = 81
\]

\[
\boxed{\text{Total number of such matrices} = 81}
\]
\end{frame}

%------------------------------------

\begin{frame}[fragile]{C Code}
\begin{lstlisting}[language=C]
#include <stdio.h>

int main() {
    int a, b, c, d;
    int count = 0;

    for (a = 1; a <= 3; a++) {
        for (b = 1; b <= 3; b++) {
            for (c = 1; c <= 3; c++) {
                for (d = 1; d <= 3; d++) {
                    count++;
                    printf("Matrix %d:\n", count);
                    printf("%d %d\n", a, b);
                    printf("%d %d\n\n", c, d);
\end{lstlisting}
\end{frame}

\begin{frame}[fragile]{C Code}
\begin{lstlisting}[language=C]

                }
            }
        }
    }

    printf("Total matrices = %d\n", count);
    return 0;
}
\end{lstlisting}
\end{frame}

%------------------------------------

\begin{frame}[fragile]{Python Code}
\begin{lstlisting}[language=Python]
count = 0

for a in [1,2,3]:
    for b in [1,2,3]:
        for c in [1,2,3]:
            for d in [1,2,3]:
                count += 1
                print(f"Matrix {count}:")
                print(f"{a} {b}")
                print(f"{c} {d}\n")

print("Total matrices =", count)
\end{lstlisting}
\end{frame}

\end{document}
