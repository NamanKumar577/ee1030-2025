\documentclass{beamer}
\usepackage[utf8]{inputenc}

\usetheme{Madrid}
\usecolortheme{default}
\usepackage{amsmath,amssymb,amsfonts,amsthm}
\usepackage{txfonts}
\usepackage{tkz-euclide}
\usepackage{listings}
\usepackage{adjustbox}
\usepackage{array}
\usepackage{tabularx}
\usepackage{gvv}
\usepackage{lmodern}
\usepackage{circuitikz}
\usepackage{tikz}
\usepackage{graphicx}

\setbeamertemplate{page number in head/foot}[totalframenumber]

\usepackage{tcolorbox}
\tcbuselibrary{minted,breakable,xparse,skins}



\definecolor{bg}{gray}{0.95}
\DeclareTCBListing{mintedbox}{O{}m!O{}}{%
  breakable=true,
  listing engine=minted,
  listing only,
  minted language=#2,
  minted style=default,
  minted options={%
    linenos,
    gobble=0,
    breaklines=true,
    breakafter=,,
    fontsize=\small,
    numbersep=8pt,
    #1},
  boxsep=0pt,
  left skip=0pt,
  right skip=0pt,
  left=25pt,
  right=0pt,
  top=3pt,
  bottom=3pt,
  arc=5pt,
  leftrule=0pt,
  rightrule=0pt,
  bottomrule=2pt,
  toprule=2pt,
  colback=bg,
  colframe=orange!70,
  enhanced,
  overlay={%
    \begin{tcbclipinterior}
    \fill[orange!20!white] (frame.south west) rectangle ([xshift=20pt]frame.north west);
    \end{tcbclipinterior}},
  #3,
}
\lstset{
    language=C,
    basicstyle=\ttfamily\small,
    keywordstyle=\color{blue},
    stringstyle=\color{orange},
    commentstyle=\color{green!60!black},
    numbers=left,
    numberstyle=\tiny\color{gray},
    breaklines=true,
    showstringspaces=false,
}
%------------------------------------------------------------

\title
{12.393}
\date{September 29, 2025}
\author 
{AI25BTECH11003 - Bhavesh Gaikwad}



\begin{document}


\frame{\titlepage}
\begin{frame}{Question}
Values of a, b, c which render the matrix $$\vec{Q} = \myvec{\frac{1}{\sqrt{3}} & \frac{1}{\sqrt{2}} & a \\ 
\frac{1}{\sqrt{3}} & 0 & b \\
\frac{1}{\sqrt{3}} & \frac{-1}{\sqrt{2}} & c}$$

orthonormal are, respectively,

\hfill{(AE 2013)}
\begin{itemize}
    \item[a)]$\frac{1}{\sqrt{2}}$, $\frac{1}{\sqrt{2}}$, 0

    \item[b)]$\frac{1}{\sqrt{6}}$, $\frac{-2}{\sqrt{6}}$, $\frac{1}{\sqrt{6}}$

    \item[c)]$\frac{1}{\sqrt{3}}$, $\frac{1}{\sqrt{3}}$, $\frac{1}{\sqrt{3}}$

    \item[d)]$\frac{-1}{\sqrt{6}}$, $\frac{2}{\sqrt{6}}$, $\frac{-1}{\sqrt{6}}$
\end{itemize}
\end{frame}


\begin{frame}[fragile]
    \frametitle{Theoretical Solution}
Given: $\vec{Q}$ is an orthogonal matrix.\\
$\qquad$ $\vec{Q} = \dfrac{1}{\sqrt{6}}\myvec{\sqrt{2} & \sqrt{3} & \sqrt{6}a \\ \sqrt{2} & 0 & \sqrt{6}b \\ \sqrt{2} & -\sqrt{3} & \sqrt{6}c}$\\\\

Condition for orthogonality,
\begin{equation}
    \vec{A}^\top\vec{A} = \vec{I}
\end{equation}

Substituting $\vec{Q}$ in Equation 0.1,
\begin{align}
\vec{Q}^\top\vec{Q} = \dfrac{1}{6} \myvec{6 & 0 & 2\sqrt{3}(a+b+c) \\ 0 & 6 & 3\sqrt{2}(a-c) \\ 2\sqrt{3}(a+b+c) & 3\sqrt{2}(a-c) & 6(a^2 + b^2 + c^2)} 
\end{align}

\begin{equation}
    \dfrac{1}{6} \myvec{6 & 0 & 2\sqrt{3}(a+b+c) \\ 0 & 6 & 3\sqrt{2}(a-c) \\ 2\sqrt{3}(a+b+c) & 3\sqrt{2}(a-c) & 6(a^2 + b^2 + c^2)} = \myvec{1 & 0 & 0 \\ 0 & 1 & 0 \\ 0 & 0 & 1}
\end{equation}
\end{frame}

\begin{frame}[fragile]
    \frametitle{Theoretical Solution}
On Comparing we get,\\
Let $\vec{P} = \myvec{a \\ b \\c}$
\begin{align}
\vec{P}^\top\myvec{1 \\ 1 \\ 1} &= 0\\
\vec{P}^\top\myvec{1 \\ 0 \\ -1} &= 0\\
\vec{P}^\top\vec{P} = \norm{\vec{P}}^2 &=  1
\end{align}
\end{frame}

\begin{frame}[fragile]
    \frametitle{Theoretical Solution}
From Equation 4 and 5,
\begin{equation}
    \myvec{1 & 1 & 1 \\ 1 & 0 & -1}\vec{P} = \myvec{0 \\ 0}
\end{equation}

Row Transformation-1: $R_2 \rightarrow R_2 - R_1$
\begin{equation}
    \myvec{1 & 1 & 1 \\ 0 & -1 & -2}\vec{P} = \myvec{0 \\ 0}
\end{equation}

Row Transformation-2: $R_1 \rightarrow R_1 + R_2$
\begin{equation}
    \myvec{1 & 0 & -1 \\ 0 & -1 & -2}\vec{P} = \myvec{0 \\ 0}
\end{equation}

From this,
\begin{equation}
    a = c \; \& \; b = -2a
\end{equation}
\end{frame}

\begin{frame}[fragile]
    \frametitle{Theoretical Solution}
\begin{equation}
  \therefore \;  \vec{P} = a\myvec{1 \\ -2 \\ 1}
\end{equation}

From Equations 6 and 11
\begin{align}
   a^2 \myvec{1 & -2 & 1} \myvec{1 \\ -2 \\ 1} &= 1 \\ 
   a^2\left[ 1 + 4 + 1 \right] = 6a^2 &= 1\\
   \boxed{a = \pm \dfrac{1}{\sqrt{6}}}
\end{align}
\end{frame}

\begin{frame}[fragile]
    \frametitle{Theoretical Solution}
    Substituting value of a in Equation-11
\begin{align}
    \therefore \; \vec{P} = \myvec{\frac{1}{\sqrt{6}} \\ \frac{-2}{\sqrt{6}} \\ \frac{1}{\sqrt{6}}} 
     \qquad OR \qquad
    \vec{P} = \myvec{\frac{-1}{\sqrt{6}} \\ \frac{2}{\sqrt{6}} \\ \frac{-1}{\sqrt{6}}}
    \end{align}

\begin{align*}
    \boxed{\text{Thus, Option B and D are correct.}}
\end{align*}
\end{frame}
\end{document}