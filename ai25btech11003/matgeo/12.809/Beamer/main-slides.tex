\documentclass{beamer}
\usepackage[utf8]{inputenc}

\usetheme{Madrid}
\usecolortheme{default}
\usepackage{amsmath,amssymb,amsfonts,amsthm}
\usepackage{txfonts}
\usepackage{tkz-euclide}
\usepackage{listings}
\usepackage{adjustbox}
\usepackage{array}
\usepackage{tabularx}
\usepackage{gvv}
\usepackage{lmodern}
\usepackage{circuitikz}
\usepackage{tikz}
\usepackage{graphicx}

\setbeamertemplate{page number in head/foot}[totalframenumber]

\usepackage{tcolorbox}
\tcbuselibrary{minted,breakable,xparse,skins}



\definecolor{bg}{gray}{0.95}
\DeclareTCBListing{mintedbox}{O{}m!O{}}{%
  breakable=true,
  listing engine=minted,
  listing only,
  minted language=#2,
  minted style=default,
  minted options={%
    linenos,
    gobble=0,
    breaklines=true,
    breakafter=,,
    fontsize=\small,
    numbersep=8pt,
    #1},
  boxsep=0pt,
  left skip=0pt,
  right skip=0pt,
  left=25pt,
  right=0pt,
  top=3pt,
  bottom=3pt,
  arc=5pt,
  leftrule=0pt,
  rightrule=0pt,
  bottomrule=2pt,
  toprule=2pt,
  colback=bg,
  colframe=orange!70,
  enhanced,
  overlay={%
    \begin{tcbclipinterior}
    \fill[orange!20!white] (frame.south west) rectangle ([xshift=20pt]frame.north west);
    \end{tcbclipinterior}},
  #3,
}
\lstset{
    language=C,
    basicstyle=\ttfamily\small,
    keywordstyle=\color{blue},
    stringstyle=\color{orange},
    commentstyle=\color{green!60!black},
    numbers=left,
    numberstyle=\tiny\color{gray},
    breaklines=true,
    showstringspaces=false,
}
%------------------------------------------------------------

\title
{12.809}
\date{October 11, 2025}
\author 
{AI25BTECH11003 - Bhavesh Gaikwad}



\begin{document}


\frame{\titlepage}
\begin{frame}{Question}
If $\vec{A} = \myvec{1 & -1 \\ 2 & -2}$, the eigenvalues of $\vec{A}$ are 

\hfill{(BM 2024)}

\begin{itemize}
    \item[a)]-1 and 0
    \item[b)]-1 and +1
    \item[c)]-1 and -1
    \item[d)]+1 and 0
\end{itemize}

\end{frame}


\begin{frame}[fragile]
    \frametitle{Theoretical Solution}
The eigenvalues of $\vec{A}$ can be obtained by solving
\begin{equation}
    \det(\vec{A} - \lambda\vec{I}) = 0 
\end{equation}

\begin{align}
 \vec{A} - \lambda\vec{I} &= \myvec{1 & -1 \\ 2 & -2} - \lambda\myvec{1 & 0 \\ 0 & 1}\\
 \vec{A} - \lambda\vec{I} &= \myvec{1-\lambda & -1 \\ 2 & -2-\lambda}
\end{align}

\begin{equation}
    \det(\vec{A} - \lambda\vec{I}) = (\lambda - 1)(\lambda + 2) - (-1)(2) = 0 
\end{equation}

\begin{align}
    \lambda^2 + \lambda = 0 \\
    \lambda = 0 \quad OR \quad \lambda = -1
\end{align}
    \begin{align*}
        \boxed{\text{Option-A is correct.}}
    \end{align*}
\end{frame}

\end{document}