\documentclass{beamer}
\usepackage[utf8]{inputenc}

\usetheme{Madrid}
\usecolortheme{default}
\usepackage{amsmath,amssymb,amsfonts,amsthm}
\usepackage{txfonts}
\usepackage{tkz-euclide}
\usepackage{listings}
\usepackage{adjustbox}
\usepackage{array}
\usepackage{tabularx}
\usepackage{gvv}
\usepackage{lmodern}
\usepackage{circuitikz}
\usepackage{tikz}
\usepackage{graphicx}

\setbeamertemplate{page number in head/foot}[totalframenumber]

\usepackage{tcolorbox}
\tcbuselibrary{minted,breakable,xparse,skins}



\definecolor{bg}{gray}{0.95}
\DeclareTCBListing{mintedbox}{O{}m!O{}}{%
  breakable=true,
  listing engine=minted,
  listing only,
  minted language=#2,
  minted style=default,
  minted options={%
    linenos,
    gobble=0,
    breaklines=true,
    breakafter=,,
    fontsize=\small,
    numbersep=8pt,
    #1},
  boxsep=0pt,
  left skip=0pt,
  right skip=0pt,
  left=25pt,
  right=0pt,
  top=3pt,
  bottom=3pt,
  arc=5pt,
  leftrule=0pt,
  rightrule=0pt,
  bottomrule=2pt,
  toprule=2pt,
  colback=bg,
  colframe=orange!70,
  enhanced,
  overlay={%
    \begin{tcbclipinterior}
    \fill[orange!20!white] (frame.south west) rectangle ([xshift=20pt]frame.north west);
    \end{tcbclipinterior}},
  #3,
}
\lstset{
    language=C,
    basicstyle=\ttfamily\small,
    keywordstyle=\color{blue},
    stringstyle=\color{orange},
    commentstyle=\color{green!60!black},
    numbers=left,
    numberstyle=\tiny\color{gray},
    breaklines=true,
    showstringspaces=false,
}
%------------------------------------------------------------

\title
{12.497}
\date{September 30, 2025}
\author 
{AI25BTECH11003 - Bhavesh Gaikwad}



\begin{document}


\frame{\titlepage}
\begin{frame}{Question}
Consider the following simultaneous equations (with $c_1$ and $c_2$ being constants)
$$3x_1 + 2x_2 = c_1$$
$$4x_1 + x_2 = c_2$$
The characteristic equation for these simultaneous equations is

\hfill{(CE 2017)}
\begin{itemize}
    \item[a)]$\lambda^2 - 4\lambda - 5 =0$
    \item[b)] $\lambda^2 - 4\lambda + 5 =0$
    \item[c)]$\lambda^2 + 4\lambda - 5 =0$
    \item[d)]$\lambda^2 + 4\lambda + 5 =0$
\end{itemize}
\end{frame}


\begin{frame}[fragile]
    \frametitle{Theoretical Solution}
Given:
\begin{align}
   3x_1 + 2x_2 &= c_1\\
4x_1 + x_2 &= c_2 
\end{align}

Let $\vec{A} = \myvec{3 & 2 \\ 4 & 1}$, $\vec{x} = \myvec{x_1 \\ x_2}$
and $\vec{c} = \myvec{c_1 \\ c_2}$

From Equation 0.1 and 0.2,
\begin{equation}
    \vec{A}\vec{x} = \vec{c}
\end{equation}

The characteristic equation is given by,
\begin{align}
|\vec{A} - \lambda\vec{I}| &= 0 \\
\left| \myvec{3-\lambda & 2 \\ 4 & 1-\lambda} \right| &= 0 \\
\end{align}
\end{frame}

\begin{frame}[fragile]
    \frametitle{Theoretical Solution}
\begin{equation}
    \boxed{\lambda^2 - 4\lambda - 5 =0}
\end{equation}

\begin{align*}
    \boxed{\text{Thus, Option-A is correct.}}
\end{align*}
\end{frame}

\end{document}