\let\negmedspace\undefined
\let\negthickspace\undefined
\documentclass[journal]{IEEEtran}
\usepackage[a5paper, margin=10mm, onecolumn]{geometry}
%\usepackage{lmodern} % Ensure lmodern is loaded for pdflatex
\usepackage{tfrupee} % Include tfrupee package

\setlength{\headheight}{1cm} % Set the height of the header box
\setlength{\headsep}{0mm}     % Set the distance between the header box and the top of the text

\usepackage{gvv-book}
\usepackage{gvv}
\usepackage{cite}
\usepackage{amsmath,amssymb,amsfonts,amsthm}
\usepackage{algorithmic}
\usepackage{graphicx}
\usepackage{textcomp}
\usepackage{xcolor}
\usepackage{txfonts}
\usepackage{listings}
\usepackage{enumitem}
\usepackage{mathtools}
\usepackage{gensymb}
\usepackage{comment}
\usepackage[breaklinks=true]{hyperref}
\usepackage{tkz-euclide} 
\usepackage{listings}
\def\inputGnumericTable{}                                 
\usepackage[latin1]{inputenc}                                
\usepackage{color}                                            
\usepackage{array}                                            
\usepackage{longtable}                                       
\usepackage{calc}                                             
\usepackage{multirow}                                         
\usepackage{hhline}                                           
\usepackage{ifthen}                                           
\usepackage{lscape}
\begin{document}

\bibliographystyle{IEEEtran}


\title{4.13.81}
\author{AI25BTECH11012 - GARIGE UNNATHI}
% \maketitle
% \newpage
% \bigskip
{\let\newpage\relax\maketitle}


\renewcommand{\thefigure}{\theenumi}
\renewcommand{\thetable}{\theenumi}
\setlength{\intextsep}{10pt} % Space between text and floats


\numberwithin{equation}{enumi}
\numberwithin{figure}{enumi}

\vspace{-1cm}

\textbf{Question}:\\
Let $P_1$ : 2x + y - z = 3 and $P_2$ : x + 2y + z = 2 be two planes . Then, which of the following statements is/are TRUE ?
\begin{enumerate}
    \item The line of intersection of P1 and P2 has direction ratios 1,2,-1
    \item The line $\frac{3x-4}{9}$ = $\frac{1-3y}{9} $= $\frac{z}{3}$ is perpendicular to the line of intersection of P1 and P2
    \item The acute angle between $P_1$ and $P_2$ is 60\degree
    \item If $P_3$ is the plane passing through the point (4,2,-2) and perpendicular to the line of intersection of $P_1$ and $P_2$ ,then the distance of the point (2,1,1) from the plane $P_3$ is $\frac{2}{\sqrt{3}}$
\end{enumerate}
\textbf{Solution: }
\begin{enumerate}

 \item  Let 
\begin{align}
    P_1 = \myvec{2\\1\\-1}^{T}\vec{X} = 3 \\
    P_2 = \myvec{1\\2\\1}^{T}\vec{X} = 2\\
    \myvec{2 &1 &-1\\1&2&1}\vec{X} = \myvec{3\\2}
\end{align}

Combining both equations and solving by row reduction we get :

\begin{align}
  \vec{X} = \myvec{0\\ \frac{5}{3} \\ -\frac{4}{3}} + \lambda\myvec{1\\-1\\1}
\end{align}

Hence , the direction ratios of the line of intersection are (1,-1,1) . So option 1 is false

\item 
simplifing the line equation we get the line equation to be :


\begin{align}
 \frac{x-\frac{4}{3}}{3} = \frac{y - \frac{1}{3}}{-3} = \frac{z}{3}\\
 \vec{X} = \myvec{-\frac{4}{3}\\- \frac{1}{3}\\0} + \mu\myvec{3\\-3\\3}
\end{align}


solving the equation by row reduction we get  direction ratios of the line to be (3,-3,3) \\
For two lines to be perpendicular :
\begin{align}
   n_1^{T}n_2 = 0
\end{align}
For the given lines :

\begin{align}
  \myvec{1\\-1\\1}^{T} \myvec{3 \\-3\\3} = 9 
\end{align}
Hence. the lines are not perpendicular . So option 2 is also false 

\item 
We find the angle between two planes by the formula :
\begin{align}
    \cos\theta = \frac{\lvert n_1^{T}n_2 \rvert}{\lVert n_1\rVert\lVert n_2\rVert}
\end{align}
By solving using above equation we get :
\begin{align}
    \cos\theta = \frac{1}{2}
\end{align}
Hence the angle $\theta$ = 60\degree . So option 3 is true 

\item 
The plane perpendicular to a line has normal or direction ratios equal to the direction ratios of the line that is (1,-1,1)\\
Hence the plane equation can be written as :
\begin{align}
    \myvec{1\\-1\\1}^T\vec{X} = c
\end{align}
To find c we can substitute the point (4,2,-2) in the plane equation :
\begin{align}
    \myvec{1\\-1\\1}^T\myvec{4\\2\\-2} = 0
\end{align}
Hence the plane equation is :
\begin{align}
    \myvec{1\\-1\\1}^T\vec{X} = 0
\end{align}
The distance of a point from a plane is given by the equation :
\begin{align}
    \frac{\lvert n^{T}\vec{P} - c \rvert}{\lVert n \rVert}
\end{align}
Solving using above equation for the point $\vec{P} $= \myvec{2\\1\\1} we get :
 \begin{align}
      \frac{\lvert 2 - 1 +1 \rvert}{\sqrt{3}} = \frac{2}{\sqrt{3}}
 \end{align}

 Hence , option 4 is also true .\\
 \end{enumerate}
 Thus options 3 and 4 are true 
\end{document}


