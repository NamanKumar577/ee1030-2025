\let\negmedspace\undefined
\let\negthickspace\undefined
\documentclass[journal]{IEEEtran}
\usepackage[a5paper, margin=10mm, onecolumn]{geometry}
%\usepackage{lmodern} % Ensure lmodern is loaded for pdflatex
\usepackage{tfrupee} % Include tfrupee package

\setlength{\headheight}{1cm} % Set the height of the header box
\setlength{\headsep}{0mm}     % Set the distance between the header box and the top of the text

\usepackage{gvv-book}
\usepackage{gvv}
\usepackage{cite}
\usepackage{amsmath,amssymb,amsfonts,amsthm}
\usepackage{algorithmic}
\usepackage{graphicx}
\usepackage{textcomp}
\usepackage{xcolor}
\usepackage{txfonts}
\usepackage{listings}
\usepackage{enumitem}
\usepackage{mathtools}
\usepackage{gensymb}
\usepackage{comment}
\usepackage[breaklinks=true]{hyperref}
\usepackage{tkz-euclide} 
\usepackage{listings}
\def\inputGnumericTable{}                                 
\usepackage[latin1]{inputenc}                                
\usepackage{color}                                            
\usepackage{array}                                            
\usepackage{longtable}                                       
\usepackage{calc}                                             
\usepackage{multirow}                                         
\usepackage{hhline}                                           
\usepackage{ifthen}                                           
\usepackage{lscape}
\begin{document}

\bibliographystyle{IEEEtran}


\title{2.10.1}
\author{AI25BTECH11012 - GARIGE UNNATHI}
% \maketitle
% \newpage
% \bigskip
{\let\newpage\relax\maketitle}


\renewcommand{\thefigure}{\theenumi}
\renewcommand{\thetable}{\theenumi}
\setlength{\intextsep}{10pt} % Space between text and floats


\numberwithin{equation}{enumi}
\numberwithin{figure}{enumi}

\vspace{-1cm}

\textbf{Question}:\\
Consider 3 points :
 \begin{align*}
     \vec{P} = (-\sin(\beta - \alpha),-\cos\beta) , \vec{Q} = (\cos(\beta - \alpha),\sin\beta)
 \end{align*}
\begin{align*}
    \vec{R} = (\cos(\beta - \alpha + \theta),\sin(\beta - \theta))
\end{align*}
where $\theta < \alpha,\beta,\theta < \frac{\pi}{4}$ Then,
\begin{enumerate}
\item  P lies on the line segment RQ
\item  Q lies on the line segment PR
\item  R lies on the line segment QP
\item  P,Q,R are non-collinear
\end{enumerate}

\textbf{Solution}:\\
First we have to check if points can be collinear for the values satisfing the given conditions :\\

The eqution for coliinearity of the given points are :\\
\begin{align}
   Rank\myvec{\vec{P} - \vec{Q}\\
               \vec{R} - \vec{Q} } = 1
\end{align}


\begin{align}
    Rank\myvec{-\sin(\beta - \alpha) - \cos(\beta - \alpha) & -\cos\beta - \sin\beta \\
               \cos(\beta - \alpha + \theta) - \cos(\beta - \alpha) & \sin(\beta - \theta) - \sin\beta } = 1\\
\end{align} 
\begin{align}
     R_2 = R_2 - R_1
\end{align}
\begin{align}
Rank\myvec{-\sin(\beta - \alpha) - \cos(\beta - \alpha) & -\cos\beta - \sin\beta \\
               \cos(\beta - \alpha + \theta) - \sin(\beta - \alpha) & \sin(\beta - \theta) - \cos\beta } = 1
\end{align}

For the rank to be 1 $R_2$ must be zero :

\begin{align}
   \cos(\beta - \alpha + \theta) - \sin(\beta - \alpha) = 0\\
\end{align}
This will only be satisfied if  :
\begin{align}
    \theta = \frac{\pi}{2} + 2\pi K \quad or  \quad \frac{\pi}{2} - 2\pi K
\end{align}
But ,given that :
\begin{align}
    0 < \theta < \frac{\pi}{4}
\end{align}
Which is contradictory :\\
Hence the points P,Q,R are not collinear .



\end{document}



