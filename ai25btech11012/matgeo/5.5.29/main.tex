\let\negmedspace\undefined
\let\negthickspace\undefined
\documentclass[journal]{IEEEtran}
\usepackage[a5paper, margin=10mm, onecolumn]{geometry}
%\usepackage{lmodern} % Ensure lmodern is loaded for pdflatex
\usepackage{tfrupee} % Include tfrupee package

\setlength{\headheight}{1cm} % Set the height of the header box
\setlength{\headsep}{0mm}     % Set the distance between the header box and the top of the text

\usepackage{gvv-book}
\usepackage{gvv}
\usepackage{cite}
\usepackage{amsmath,amssymb,amsfonts,amsthm}
\usepackage{algorithmic}
\usepackage{graphicx}
\usepackage{textcomp}
\usepackage{xcolor}
\usepackage{txfonts}
\usepackage{listings}
\usepackage{enumitem}
\usepackage{mathtools}
\usepackage{gensymb}
\usepackage{comment}
\usepackage[breaklinks=true]{hyperref}
\usepackage{tkz-euclide} 
\usepackage{listings}
\def\inputGnumericTable{}                                 
\usepackage[latin1]{inputenc}                                
\usepackage{color}                                            
\usepackage{array}                                            
\usepackage{longtable}                                       
\usepackage{calc}                                             
\usepackage{multirow}                                         
\usepackage{hhline}                                           
\usepackage{ifthen}                                           
\usepackage{lscape}
\begin{document}

\bibliographystyle{IEEEtran}


\title{5.5.29}
\author{AI25BTECH11012 - GARIGE UNNATHI}
% \maketitle
% \newpage
% \bigskip
{\let\newpage\relax\maketitle}


\renewcommand{\thefigure}{\theenumi}
\renewcommand{\thetable}{\theenumi}
\setlength{\intextsep}{10pt} % Space between text and floats


\numberwithin{equation}{enumi}
\numberwithin{figure}{enumi}

\vspace{-1cm}

\textbf{Question}:\\
If the inverse of the matrix \myvec{7 &-3&-3\\
                                    -1&1&0\\
                                     -1&0&1}is the matrix \myvec{1&3&3\\1&\lambda&3\\1&3&4} , then find the value of $\lambda$ .


\textbf{Solution: }\\
Let : 
\begin{align*}
    \vec{A}  = \myvec{7 &-3&-3\\
                      -1&1&0\\
                      -1&0&1}
\end{align*}
since we know that $\vec{A}$$\vec{A}^{-1}$ = $\vec{I}$
\begin{align}
  \myvec{7 &-3&-3\\
          -1&1&0\\
          -1&0&1} \myvec{1&3&3\\1&\lambda&3\\1&3&4} = \myvec{1&0&0\\0&1&0\\0&0&1}
\end{align}
we can find $\lambda$ just by comparing the  element $a_{22}$ :
\begin{align}
 a_{22} = -3 + \lambda + 0  = 1\\
 \lambda = 4
\end{align}






\end{document}


