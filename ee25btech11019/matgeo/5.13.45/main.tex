\let\negmedspace\undefined
\let\negthickspace\undefined
\documentclass[journal]{IEEEtran}
\usepackage[a5paper, margin=10mm, onecolumn]{geometry}
%\usepackage{lmodern} % Ensure lmodern is loaded for pdflatex
\usepackage{tfrupee} % Include tfrupee package

\setlength{\headheight}{1cm} % Set the height of the header box
\setlength{\headsep}{0mm}     % Set the distance between the header box and the top of the text

\usepackage{gvv-book}
%\usepackage{gvv}
\usepackage{cite}
\usepackage{amsmath,amssymb,amsfonts,amsthm}
\usepackage{algorithmic}
\usepackage{graphicx}
\usepackage{textcomp}
\usepackage{xcolor}
\usepackage{txfonts}
\usepackage{listings}
\usepackage{enumitem}
\usepackage{mathtools}
\usepackage{gensymb}
\usepackage{comment}
\usepackage[breaklinks=true]{hyperref}
\usepackage{tkz-euclide} 
\usepackage{listings}
\usepackage{gvv}                                        
\def\inputGnumericTable{}                                 
\usepackage[latin1]{inputenc}                                
\usepackage{color}                                            
\usepackage{array}                                            
\usepackage{longtable}                                       
\usepackage{calc}                                             
\usepackage{multirow}                                         
\usepackage{hhline}                                           
\usepackage{ifthen}                                           
\usepackage{lscape}
\begin{document}

\bibliographystyle{IEEEtran}

\title{5.13.45}
\author{EE25BTECH11019 - Darji Vivek M.}
{\let\newpage\relax\maketitle}

\renewcommand{\thefigure}{\theenumi}
\renewcommand{\thetable}{\theenumi}
\setlength{\intextsep}{10pt}
\numberwithin{figure}{enumi}
\renewcommand{\thetable}{\theenumi}

\textbf{Question:}\\
Let $x\in\mathbb R$
\[
\vec{P}=\myvec{1 & 1 & 1\\[2pt] 0 & 2 & 2\\[2pt] 0 & 0 & 3},\qquad
\vec{Q}=\myvec{2 & x & x\\[2pt] 0 & 4 & 0\\[2pt] x & x & 5},
\qquad R=PQP^{-1}.
\]
Then which of the following options is/are correct?\\[4pt]

\begin{enumerate}
\item $\det \vec{R}=\det\!\myvec{2 & x & x\\[2pt]0 & 4 & 0\\[2pt]x & x & 5}+8$ for all $x\in\mathbb R$.
\item For $x=1$, there exists a unit vector $\alpha\vec{i}+\beta\vec{j}+\gamma\vec{k}$ such that $\vec{R}\myvec{\alpha\\[2pt]\beta\\[2pt]\gamma}=\myvec{0\\[2pt]0\\[2pt]0}$.
\item There exists a real number $x$ such that $\vec{PQ}=\vec{QP}$.
\item For $x=0$, if $\vec{R}=\myvec{1\\[2pt]a\\[2pt]b}=6\myvec{1\\[2pt]a\\[2pt]b}$, then $a+b=5$.
\end{enumerate}

\textbf{Solution:}\\[2pt]
\textbf{Matrix Method:}\\[-2mm]

Given
\[
\vec{R} = \vec{P}\vec{Q}\vec{P}^{-1}
\]
\[
\vec{P}=\myvec{1 & 1 & 1\\ 0 & 2 & 2\\ 0 & 0 & 3}, \quad
\vec{Q}=\myvec{2 & x & x\\ 0 & 4 & 0\\ x & x & 5}
\]

\textbf{(a)}\\[-2mm]
Since $\mydet{\vec{R}}=\mydet{\vec{Q}}$,  
\[
\mydet{\vec{Q}} =
2\mydet{4 & 0\\ x & 5}
- x\mydet{0 & 0\\ x & 5}
+ x\mydet{0 & 4\\ x & x}
\]
\[
= 2(20) - x(0) + x(0 - 4x)
\Rightarrow \mydet{\vec{Q}} = 40 - 4x^2
\]
\[
\Rightarrow \mydet{\vec{R}} = 40 - 4x^2
\]
Hence, option (a) is \textbf{false}.

\textbf{(b)}\\[-2mm]
For $x=1$,
\[
\mydet{\vec{R}} = 40 - 4(1)^2 = 36 \neq 0
\]
Since $\mydet{\vec{R}} \neq 0$, $\vec{R}$ is invertible,  
so $\vec{R}\myvec{\alpha\\ \beta\\ \gamma} = \myvec{0\\ 0\\ 0}$  
has only the trivial solution.  
Hence, no unit vector exists $\Rightarrow$ (b) \textbf{false}.

\textbf{(c)}\\[-2mm]
\[
\vec{P}\vec{Q}
=\myvec{1 & 1 & 1\\ 0 & 2 & 2\\ 0 & 0 & 3}
\myvec{2 & x & x\\ 0 & 4 & 0\\ x & x & 5}
=\myvec{2+x & 2x+4 & x+5\\ 2x & 8+2x & 10\\ 3x & 3x & 15}
\]
\[
\vec{Q}\vec{P}
=\myvec{2 & x & x\\ 0 & 4 & 0\\ x & x & 5}
\myvec{1 & 1 & 1\\ 0 & 2 & 2\\ 0 & 0 & 3}
=\myvec{2 & 2+2x & 2+5x\\ 0 & 8 & 8\\ x & 3x & 3x+15}
\]
Comparing entries, no $x$ satisfies $\vec{P}\vec{Q}=\vec{Q}\vec{P}$.  
Hence, (c) \textbf{false}.

\textbf{(d)}\\[-2mm]
For $x=0$,
\[
\vec{Q}=\myvec{2 & 0 & 0\\ 0 & 4 & 0\\ 0 & 0 & 5}
\Rightarrow \text{Eigenvalues of }\vec{R}=\{2,4,5\}
\]
If $\vec{R}\myvec{1\\ a\\ b}=6\myvec{1\\ a\\ b}$,  
then $6$ must be an eigenvalue, which is not possible.  
Hence, (d) \textbf{false}.

\[
\boxed{\text{All options are incorrect.}}
\]

\end{document}