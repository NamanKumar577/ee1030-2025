\let\negmedspace\undefined
\let\negthickspace\undefined
\documentclass[journal,12pt,onecolumn]{IEEEtran}
\usepackage{cite}
\usepackage{amsmath,amssymb,amsfonts,amsthm}
\usepackage{algorithmic}
\usepackage{graphicx}
\graphicspath{{./figs/}}
\usepackage{textcomp}
\usepackage{xcolor}
\usepackage{txfonts}
\usepackage{listings}
\usepackage{enumitem}
\usepackage{mathtools}
\usepackage{gensymb}
\usepackage{comment}
\usepackage{caption}
\usepackage[breaklinks=true]{hyperref}
\usepackage{tkz-euclide} 
\usepackage{listings}
\usepackage{gvv}                                        
%\def\inputGnumericTable{}                                 
\usepackage[latin1]{inputenc}     
\usepackage{xparse}
\usepackage{color}                                            
\usepackage{array}                                            
\usepackage{longtable}                                       
\usepackage{calc}                                             
\usepackage{multirow}
\usepackage{multicol}
\usepackage{hhline}                                           
\usepackage{ifthen}                                           
\usepackage{lscape}
\usepackage{tabularx}
\usepackage{array}
\usepackage{listings}
\lstset{language=C, basicstyle=\ttfamily\footnotesize, keywordstyle=\color{blue}, commentstyle=\color{green!50!black}}
\usepackage{float}
%\newtheorem{theorem}{Theorem}[section]
%\newtheorem{theorem}{Theorem}[section]
%\newtheorem{problem}{Problem}
%\newtheorem{proposition}{Proposition}[section]
%\newtheorem{lemma}{Lemma}[section]
%\newtheorem{corollary}[theorem]{Corollary}
%\newtheorem{example}{Example}[section]
%\newtheorem{definition}[problem]{Definition}

\begin{document}

\title{5.2.23}
\author{EE25BTECH11020 - Darsh Pankaj Gajare}
% \maketitle
% \newpage
% \bigskip
%\begin{document}
{\let\newpage\relax\maketitle}
%\renewcommand{\thefigure}{\theenumi}
%\renewcommand{\thetable}{\theenumi}
Question:\\Using elementary transformations, find inverse of the matrix $\myvec{2&1\\7&4}$
\solution
\begin{table}[H]
	\centering
	\caption{}
	\begin{tabular}{|c|c|}
\hline
\textbf{Name} & \textbf{Value} \\ \hline
$\vec{A}$ & $\myvec{2 & 1 \\0 & 3}$ \\ \hline
\end{tabular}

	\label{}
\end{table}
\begin{align}
	\vec{A}\vec{A}^{-1}=\vec{I}
\end{align}
Using Augmented matrix,
\begin{align}
	\augvec{2}{2}{2&1&1&0\\7&4&0&1}
\end{align}
$R_2=R_2-3R_1$
\begin{align}
	\augvec{2}{2}{2&1&1&0\\1&1&-3&1}
\end{align}
$R_1=R_1-R_2$
\begin{align}
	\augvec{2}{2}{1&0&4&-1\\1&1&-3&1}
\end{align}
$R_2=R_2-R_1$
\begin{align}
	\augvec{2}{2}{1&0&4&-1\\0&1&-7&2}
\end{align}
\begin{align}
	\vec{A}^{-1}=\myvec{4&-1\\-7&2}
\end{align}
C function to find inverse:
\begin{lstlisting}
int inverse2x2(double mat[4], double inv[4]) {
    double a = mat[0], b = mat[1];
    double c = mat[2], d = mat[3];

    double det = a*d - b*c;
    if(det == 0.0) {
        return -1;  // not invertible
    }

    inv[0] =  d / det;
    inv[1] = -b / det;
    inv[2] = -c / det;
    inv[3] =  a / det;

    return 0; // success
}
\end{lstlisting}

\end{document}

