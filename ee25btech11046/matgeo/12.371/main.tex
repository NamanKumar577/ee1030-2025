\documentclass[journal]{IEEEtran}
\usepackage[a5paper, margin:10mm, onecolumn]{geometry}
\usepackage{amsmath,amssymb,amsfonts,amsthm}
\usepackage{mathtools}
\usepackage{gvv-book}
\usepackage{gvv}
\usepackage{hyperref}

\begin{document}

\title{12.371}
\author{Puni Aditya - EE25BTECH11046}
\maketitle

\textbf{Question:}

Let T : $P_3$[0, 1] $\to$ $P_2$[0, 1] be defined by $\brak{T p}\brak{x} = p''\brak{x} + p'\brak{x}$. Then the matrix
representation of T with respect to the bases $\{1, x, x^2, x^3 \}$ and $\{1, x, x^2 \}$ of $P_3$[0, 1]
and $P_2$[0, 1] respectively is
\begin{enumerate}
    \item $\myvec{0 & 0 & 0 \\ 1 & 0 & 0 \\ 2 & 2 & 0 \\ 0 & 6 & 3}$
    \item $\myvec{0 & 1 & 2 & 0 \\ 0 & 0 & 2 & 6 \\ 0 & 0 & 0 & 3}$
    \item $\myvec{0 & 2 & 1 & 0 \\ 6 & 2 & 0 & 0 \\ 3 & 0 & 0 & 0}$
    \item $\myvec{0 & 0 & 0 \\ 0 & 0 & 1 \\ 0 & 2 & 2 \\ 3 & 6 & 0}$
\end{enumerate}

\textbf{Solution:}

The transformation is 
\begin{align}
    T\brak{p}\brak{x} = p''\brak{x} + p'\brak{x} \label{eq:31}
\end{align}
The domain basis is 
\begin{align}
    \mathcal{B} = \cbrak{1, x, x^2, x^3} = \cbrak{\vec{v}_1, \vec{v}_2, \vec{v}_3, \vec{v}_4}
\end{align}
The codomain basis is 
\begin{align}
    \mathcal{C} = \cbrak{1, x, x^2} = \cbrak{\vec{w}_1, \vec{w}_2, \vec{w}_3}
\end{align}
The matrix representation $\vec{A}$ is given by
\begin{align}
    \vec{A} = \myvec{ [T\brak{\vec{v}_1}]_\mathcal{C} & [T\brak{\vec{v}_2}]_\mathcal{C} & [T\brak{\vec{v}_3}]_\mathcal{C} & [T\brak{\vec{v}_4}]_\mathcal{C} }
\end{align}
The columns of $\vec{A}$ are computed by applying the transformation \eqref{eq:31} to each basis vector in $\mathcal{B}$.
\begin{align}
    T\brak{\vec{v}_1} = T\brak{1} &= 0\brak{1} + 0\brak{x} + 0\brak{x^2} \implies [T\brak{\vec{v}_1}]_\mathcal{C} = \myvec{0 \\ 0 \\ 0} \\
    T\brak{\vec{v}_2} = T\brak{x} &= 1\brak{1} + 0\brak{x} + 0\brak{x^2} \implies [T\brak{\vec{v}_2}]_\mathcal{C} = \myvec{1 \\ 0 \\ 0} \\
    T\brak{\vec{v}_3} = T\brak{x^2} &= 2\brak{1} + 2\brak{x} + 0\brak{x^2} \implies [T\brak{\vec{v}_3}]_\mathcal{C} = \myvec{2 \\ 2 \\ 0} \\
    T\brak{\vec{v}_4} = T\brak{x^3} &= 0\brak{1} + 6\brak{x} + 3\brak{x^2} \implies [T\brak{\vec{v}_4}]_\mathcal{C} = \myvec{0 \\ 6 \\ 3}
\end{align}
This gives
\begin{align}
    \vec{A} = \myvec{0 & 1 & 2 & 0 \\ 0 & 0 & 2 & 6 \\ 0 & 0 & 0 & 3}
\end{align}
The correct option is \textbf{2)}.

\end{document}
