\documentclass{beamer}
\usepackage[utf8]{inputenc}

\usetheme{Madrid}
\usecolortheme{default}
\usepackage{amsmath,amssymb,amsfonts,amsthm}
\usepackage{txfonts}
\usepackage{tkz-euclide}
\usepackage{listings}
\usepackage{adjustbox}
\usepackage{array}
\usepackage{tabularx}
\usepackage{gvv}
\usepackage{lmodern}
\usepackage{circuitikz}
\usepackage{tikz}
\usepackage{graphicx}
\usepackage{mathtools}

\setbeamertemplate{page number in head/foot}[totalframenumber]

\usepackage{tcolorbox}
\tcbuselibrary{minted,breakable,xparse,skins}



\definecolor{bg}{gray}{0.95}
\DeclareTCBListing{mintedbox}{O{}m!O{}}{%
  breakable=true,
  listing engine=minted,
  listing only,
  minted language=#2,
  minted style=default,
  minted options={%
    linenos,
    gobble=0,
    breaklines=true,
    breakafter=,,
    fontsize=\small,
    numbersep=8pt,
    #1},
  boxsep=0pt,
  left skip=0pt,
  right skip=0pt,
  left=25pt,
  right=0pt,
  top=3pt,
  bottom=3pt,
  arc=5pt,
  leftrule=0pt,
  rightrule=0pt,
  bottomrule=2pt,
  toprule=2pt,
  colback=bg,
  colframe=orange!70,
  enhanced,
  overlay={%
    \begin{tcbclipinterior}
    \fill[orange!20!white] (frame.south west) rectangle ([xshift=20pt]frame.north west);
    \end{tcbclipinterior}},
  #3,
}
\lstset{
    language=C,
    basicstyle=\ttfamily\small,
    keywordstyle=\color{blue},
    stringstyle=\color{orange},
    commentstyle=\color{green!60!black},
    numbers=left,
    numberstyle=\tiny\color{gray},
    breaklines=true,
    showstringspaces=false,
}
\title{12.371}
\date{3rd October, 2025}
\author{Puni Aditya - EE25BTECH11046}

\begin{document}

\frame{\titlepage}
\begin{frame}{Question}
Let T : $P_3$[0, 1] $\to$ $P_2$[0, 1] be defined by $\brak{T\text{  }p}\brak{x} = p''\brak{x} + p'\brak{x}$. Then the matrix
representation of T with respect to the bases $\{1, x, x^2, x^3 \}$ and $\{1, x, x^2 \}$ of $P_3$[0, 1]
and $P_2$[0, 1] respectively is
\begin{enumerate}
    \item $\myvec{0 & 0 & 0 \\ 1 & 0 & 0 \\ 2 & 2 & 0 \\ 0 & 6 & 3}$
    \item $\myvec{0 & 1 & 2 & 0 \\ 0 & 0 & 2 & 6 \\ 0 & 0 & 0 & 3}$
    \item $\myvec{0 & 2 & 1 & 0 \\ 6 & 2 & 0 & 0 \\ 3 & 0 & 0 & 0}$
    \item $\myvec{0 & 0 & 0 \\ 0 & 0 & 1 \\ 0 & 2 & 2 \\ 3 & 6 & 0}$
\end{enumerate}
\end{frame}

\begin{frame}{Theoretical Solution}
The transformation is 
\begin{align}
    T\brak{p}\brak{x} = p''\brak{x} + p'\brak{x} \label{eq:31}
\end{align}
The domain basis is 
\begin{align}
    \mathcal{B} = \cbrak{1, x, x^2, x^3} = \cbrak{\vec{v}_1, \vec{v}_2, \vec{v}_3, \vec{v}_4}
\end{align}
The codomain basis is 
\begin{align}
    \mathcal{C} = \cbrak{1, x, x^2} = \cbrak{\vec{w}_1, \vec{w}_2, \vec{w}_3}
\end{align}
\end{frame}

\begin{frame}{Theoretical Solution}
The matrix representation $\vec{A}$ is given by
\begin{align}
    \vec{A} = \myvec{ [T\brak{\vec{v}_1}]_\mathcal{C} & [T\brak{\vec{v}_2}]_\mathcal{C} & [T\brak{\vec{v}_3}]_\mathcal{C} & [T\brak{\vec{v}_4}]_\mathcal{C} }
\end{align}
The columns of $\vec{A}$ are computed by applying the transformation \eqref{eq:31} to each basis vector in $\mathcal{B}$.
\begin{align}
    T\brak{\vec{v}_1} = T\brak{1} &= 0\brak{1} + 0\brak{x} + 0\brak{x^2} \implies [T\brak{\vec{v}_1}]_\mathcal{C} = \myvec{0 \\ 0 \\ 0} \\
    T\brak{\vec{v}_2} = T\brak{x} &= 1\brak{1} + 0\brak{x} + 0\brak{x^2} \implies [T\brak{\vec{v}_2}]_\mathcal{C} = \myvec{1 \\ 0 \\ 0}
\end{align}
\end{frame}

\begin{frame}{Theoretical Solution}
\begin{align}
    T\brak{\vec{v}_3} = T\brak{x^2} &= 2\brak{1} + 2\brak{x} + 0\brak{x^2} \implies [T\brak{\vec{v}_3}]_\mathcal{C} = \myvec{2 \\ 2 \\ 0} \\
    T\brak{\vec{v}_4} = T\brak{x^3} &= 0\brak{1} + 6\brak{x} + 3\brak{x^2} \implies [T\brak{\vec{v}_4}]_\mathcal{C} = \myvec{0 \\ 6 \\ 3}
\end{align}
This gives
\begin{align}
    \vec{A} = \myvec{0 & 1 & 2 & 0 \\ 0 & 0 & 2 & 6 \\ 0 & 0 & 0 & 3}
\end{align}
The correct option is \textbf{2)}.
\end{frame}

\end{document}
