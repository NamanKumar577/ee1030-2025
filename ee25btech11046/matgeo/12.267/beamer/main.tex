\documentclass{beamer}
\usepackage[utf8]{inputenc}

\usetheme{Madrid}
\usecolortheme{default}
\usepackage{amsmath,amssymb,amsfonts,amsthm}
\usepackage{txfonts}
\usepackage{tkz-euclide}
\usepackage{listings}
\usepackage{adjustbox}
\usepackage{array}
\usepackage{tabularx}
\usepackage{gvv}
\usepackage{lmodern}
\usepackage{circuitikz}
\usepackage{tikz}
\usepackage{graphicx}
\usepackage{mathtools}

\setbeamertemplate{page number in head/foot}[totalframenumber]

\usepackage{tcolorbox}
\tcbuselibrary{minted,breakable,xparse,skins}



\definecolor{bg}{gray}{0.95}
\DeclareTCBListing{mintedbox}{O{}m!O{}}{%
  breakable=true,
  listing engine=minted,
  listing only,
  minted language=#2,
  minted style=default,
  minted options={%
    linenos,
    gobble=0,
    breaklines=true,
    breakafter=,,
    fontsize=\small,
    numbersep=8pt,
    #1},
  boxsep=0pt,
  left skip=0pt,
  right skip=0pt,
  left=25pt,
  right=0pt,
  top=3pt,
  bottom=3pt,
  arc=5pt,
  leftrule=0pt,
  rightrule=0pt,
  bottomrule=2pt,
  toprule=2pt,
  colback=bg,
  colframe=orange!70,
  enhanced,
  overlay={%
    \begin{tcbclipinterior}
    \fill[orange!20!white] (frame.south west) rectangle ([xshift=20pt]frame.north west);
    \end{tcbclipinterior}},
  #3,
}
\lstset{
    language=C,
    basicstyle=\ttfamily\small,
    keywordstyle=\color{blue},
    stringstyle=\color{orange},
    commentstyle=\color{green!60!black},
    numbers=left,
    numberstyle=\tiny\color{gray},
    breaklines=true,
    showstringspaces=false,
}
\title{12.267}
\date{2nd October, 2025}
\author{Puni Aditya - EE25BTECH11046}

\begin{document}

\frame{\titlepage}
\begin{frame}{Question}
Two matrices $\vec{A}$ and $\vec{B}$ are said to be similar if 
\begin{align*}
    \vec{B} = \vec{P}^{-1} \vec{A}\vec{P}
\end{align*}
for some invertible matrix $\vec{P}$. Which of the following statements is NOT TRUE?
\begin{enumerate}
    \item det $\vec{A}$ = det $\vec{B}$
    \item Trace of $\vec{A}$ = Trace of $\vec{B}$
    \item $\vec{A}$ and $\vec{B}$ have the same eigenvectors
    \item $\vec{A}$ and $\vec{B}$ have the same eigenvalues
\end{enumerate}
\end{frame}

\begin{frame}{Theoretical Solution}
Let $\vec{A}$ and $\vec{B}$ be similar matrices, such that 
\begin{align} 
    \vec{B} = \vec{P}^{-1}\vec{A}\vec{P}
\end{align}
for an invertible matrix $\vec{P}$. \\
For the determinant in 1),
\begin{align}
    \mydet{\vec{B}} &= \mydet{\vec{P}^{-1}\vec{A}\vec{P}} \\
    &= \mydet{\vec{P}^{-1}}\mydet{\vec{A}}\mydet{\vec{P}} \\
    &= \frac{1}{\mydet{\vec{P}}}\mydet{\vec{A}}\mydet{\vec{P}} = \mydet{\vec{A}}
\end{align}
The statement is true.
\end{frame}

\begin{frame}{Theoretical Solution}
For the trace in 2), the cyclic property of trace states 
\begin{align}
    \text{Tr}\brak{\vec{X}\vec{Y}\vec{Z}} = \text{Tr}\brak{\vec{Z}\vec{X}\vec{Y}} \label{eq:46}
\end{align}
Using \eqref{eq:46}
\begin{align}
    \text{Tr}\brak{\vec{B}} &= \text{Tr}\brak{\vec{P}^{-1}\vec{A}\vec{P}} \\
    &= \text{Tr}\brak{\vec{A}\vec{P}\vec{P}^{-1}} = \text{Tr}\brak{\vec{A}}
\end{align}
The statement is true.
\end{frame}

\begin{frame}{Theoretical Solution}
For the eigenvalues in 4), we examine the characteristic polynomial.
\begin{align}
    \mydet{\vec{B} - \lambda\vec{I}} &= \mydet{\vec{P}^{-1}\vec{A}\vec{P} - \lambda\vec{I}} \\
    &= \mydet{\vec{P}^{-1}\vec{A}\vec{P} - \lambda\vec{P}^{-1}\vec{I}\vec{P}} \\
    &= \mydet{\vec{P}^{-1}\brak{\vec{A} - \lambda\vec{I}}\vec{P}} \\
    &= \mydet{\vec{P}^{-1}}\mydet{\vec{A} - \lambda\vec{I}}\mydet{\vec{P}} \\
    &= \mydet{\vec{A} - \lambda\vec{I}}
\end{align}
Since the characteristic polynomials are identical, the eigenvalues are the same. The statement is true.
\end{frame}

\begin{frame}{Theoretical Solution}
For the eigenvectors in 3), let $\vec{v}$ be an eigenvector of $\vec{A}$ with eigenvalue $\lambda$, so that
\begin{align}
    \vec{A}\vec{v} = \lambda\vec{v}
\end{align}
\begin{align}
    \vec{B}\brak{\vec{P}^{-1}\vec{v}} &= \brak{\vec{P}^{-1}\vec{A}\vec{P}}\brak{\vec{P}^{-1}\vec{v}} \\
    &= \vec{P}^{-1}\vec{A}\brak{\vec{P}\vec{P}^{-1}}\vec{v} \\
    &= \vec{P}^{-1}\vec{A}\vec{v} = \vec{P}^{-1}\brak{\lambda\vec{v}} \\
    &= \lambda\brak{\vec{P}^{-1}\vec{v}}
\end{align}
This shows that if $\vec{v}$ is an eigenvector of $\vec{A}$, then $\vec{P}^{-1}\vec{v}$ is the eigenvector of $\vec{B}$. Since $\vec{v} \neq \vec{P}^{-1}\vec{v}$ in general, the statement is not true.
\end{frame}

\begin{frame}{Theoretical Solution}
\textbf{Example:} Let
\begin{align}
    \vec{A} = \myvec{1 & 1 \\ 0 & 2}, \quad \vec{P} = \myvec{1 & 1 \\ 0 & 1} \implies \vec{P}^{-1} = \myvec{1 & -1 \\ 0 & 1}
\end{align}
The matrix $\vec{A}$ has an eigenvalue $\lambda=2$ with corresponding eigenvector 
\begin{align}
	\vec{v} = \myvec{1 \\ 1}
\end{align}
\end{frame}

\begin{frame}{Theoretical Solution}
The similar matrix $\vec{B}$ is
\begin{align}
    \vec{B} = \vec{P}^{-1}\vec{A}\vec{P} = \myvec{1 & -1 \\ 0 & 1}\myvec{1 & 1 \\ 0 & 2}\myvec{1 & 1 \\ 0 & 1} = \myvec{1 & 0 \\ 0 & 2}
\end{align}
The corresponding eigenvector of $\vec{B}$ for the eigenvalue $\lambda=2$ is 
\begin{align}
	\vec{w} = \vec{P}^{-1}\vec{v}
\end{align}
\begin{align}
    \vec{w} = \myvec{1 & -1 \\ 0 & 1}\myvec{1 \\ 1} = \myvec{0 \\ 1}
\end{align}
Clearly, $\vec{v} \neq \vec{w}$.
\end{frame}

\begin{frame}{Conclusion}
The statement that is NOT TRUE is \textbf{3) $\vec{A}$ and $\vec{B}$ have the same eigenvectors}.
\end{frame}

\end{document}
