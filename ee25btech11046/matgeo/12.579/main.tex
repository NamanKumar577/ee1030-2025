\documentclass[journal]{IEEEtran}
\usepackage[a5paper, margin:10mm, onecolumn]{geometry}
\usepackage{amsmath,amssymb,amsfonts,amsthm}
\usepackage{mathtools}
\usepackage{gvv-book}
\usepackage{gvv}
\usepackage{hyperref}

\begin{document}

\title{12.579}
\author{Puni Aditya - EE25BTECH11046}
\maketitle

\textbf{Question:}

Let V be the vector space of all 3 × 3 matrices with complex entries over the real
field. If
\begin{align*}
    W_1 &= \cbrak{\vec{A} \in V : \vec{A} = \bar{\vec{A}}^\top} \\
    W_2 &= \cbrak{\vec{A} \in V : \text{trace} \brak{\vec{A}} = 0}
\end{align*}
then the dimension of $W_1 + W_2$ is equal to \rule{2cm}{0.4pt}.

\textbf{Solution:}

A basis for a vector space $V$ over a field $F$ is a set $\mathcal{B} = \cbrak{\vec{v}_1, \dots, \vec{v}_n}$ where for any $\vec{v} \in V$, there exist unique scalars $c_i \in F$ such that:
\begin{align}
    \vec{v} = \sum_{i=1}^n c_i \vec{v}_i
\end{align}
This uniqueness implies linear independence, where the only solution to the following equation is $c_1 = \dots = c_n = 0$.
\begin{align}
    \sum_{i=1}^n c_i \vec{v}_i = \vec{0}
\end{align}
Let $V$ be the space of $n \times n$ complex matrices over $\mathbb{R}$. Let $\vec{E}_{jk}$ be the matrix with 1 at position $\brak{j,k}$ and 0 elsewhere. Any $\vec{A} \in V$ can be written as:
\begin{align}
    \vec{A} = \sum_{j=1}^n \sum_{k=1}^n a_{jk} \vec{E}_{jk} = \sum_{j,k} \brak{x_{jk} + i y_{jk}} \vec{E}_{jk} = \sum_{j,k} x_{jk}\vec{E}_{jk} + \sum_{j,k} y_{jk}\brak{i\vec{E}_{jk}}
\end{align}
The set 
\begin{align}
    \mathcal{B}_V = \cbrak{\vec{E}_{11}, \dots, \vec{E}_{nn}, i\vec{E}_{11}, \dots, i\vec{E}_{nn}}
\end{align}
spans $V$. Linear independence over $\mathbb{R}$ is shown by:
\begin{align}
    \sum_{j,k} x_{jk}\vec{E}_{jk} + \sum_{j,k} y_{jk}\brak{i\vec{E}_{jk}} = \vec{0} \implies \sum_{j,k} \brak{x_{jk} + i y_{jk}} \vec{E}_{jk} = \vec{0}
\end{align}
This implies 
\begin{align}
    x_{jk} + i y_{jk} = 0\text{, thus }x_{jk}=0\text{ and }y_{jk}=0\text{ for all }j,k
\end{align}    
$\mathcal{B}_V$ is a basis.
\begin{align}
    \dim\brak{V} = n^2 + n^2 = 2n^2
\end{align}
Let $\mathcal{B}_{1 \cap 2}$ be a basis for $W_1 \cap W_2$. Extending it to bases $\mathcal{B}_1$ for $W_1$ and $\mathcal{B}_2$ for $W_2$. The set 
\begin{align*}
    \mathcal{B}_{1+2} = \mathcal{B}_1 \cup \mathcal{B}_2
\end{align*}
spans $W_1+W_2$. For linear independence:
\begin{align}
    \sum a_i \vec{u}_i + \sum b_j \vec{v}_j + \sum c_l \vec{w}_l = \vec{0} \implies \sum a_i \vec{u}_i + \sum b_j \vec{v}_j = - \sum c_l \vec{w}_l
\end{align}
The vector is in $W_1 \cap W_2$, so $-\sum c_l \vec{w}_l = \sum d_i \vec{u}_i$. Since $\mathcal{B}_2$ is a basis, all $c_l=0, d_i=0$, which implies all $a_i=0, b_j=0$.
\begin{align}
    \dim\brak{W_1 + W_2} = \dim\brak{W_1} + \dim\brak{W_2} - \dim\brak{W_1 \cap W_2}
\end{align}
For $W_1$, the Hermitian condition 
\begin{align}
    a_{jk} = \bar{a}_{kj} \label{eq:69}
\end{align}
From \eqref{eq:69},
\begin{align}
    x_{jk} + iy_{jk} = x_{kj} - iy_{kj} \implies x_{jk}=x_{kj} \text{ and } y_{jk}=-y_{kj}
\end{align}
For diagonal elements, 
\begin{align}
y_{jj} = -y_{jj} \implies y_{jj} = 0
\end{align}
This gives $n$ real parameters.
For off-diagonal elements, there are 
\begin{align}
    \frac{n^2-n}{2}
\end{align}
pairs. Each pair is determined by one complex number $\myvec{x_{jk}, y_{jk}}$, giving 
\begin{align}
    2 \times \frac{n^2-n}{2} = n^2-n
\end{align}
real parameters.
\begin{align}
    \dim\brak{W_1} = n + \brak{n^2-n} = n^2
\end{align}
For $W_2$, the trace condition imposes two independent real constraints on the $2n^2$ parameters of $V$:
\begin{align}
    \text{trace}\brak{\vec{A}} = \sum x_{jj} + i \sum y_{jj} = 0 \implies \sum x_{jj} = 0 \text{ and } \sum y_{jj} = 0
\end{align}
\begin{align}
    \dim\brak{W_2} = 2n^2 - 2
\end{align}
For $W_1 \cap W_2$, matrices are Hermitian, so $y_{jj}=0$. The trace condition becomes one real constraint on the $n^2$ parameters of $W_1$.
\begin{align}
    \dim\brak{W_1 \cap W_2} = n^2 - 1
\end{align}
For this problem,
\begin{align}
    n=3
\end{align}
\begin{align}
    \dim\brak{W_1} &= 3^2 = 9 \\
    \dim\brak{W_2} &= 2\brak{3^2} - 2 = 16 \\
    \dim\brak{W_1 \cap W_2} &= 3^2 - 1 = 8
\end{align}
\begin{align}
    \therefore \dim\brak{W_1 + W_2} = 9 + 16 - 8 = 17
\end{align}

\end{document}
