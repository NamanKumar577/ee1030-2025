\documentclass{beamer}

\mode<presentation>
\usepackage{amsmath,amssymb,mathtools}
\usepackage{textcomp}
\usepackage{gensymb}
\usepackage{adjustbox}
\usepackage{subcaption}
\usepackage{enumitem}
\usepackage[utf8]{inputenc}
\usepackage{amssymb}
\usepackage{newunicodechar}
\usepackage{enumitem}
\setlist{nosep} % optional: removes vertical gaps
\setlist[enumerate]{label=\arabic*)} % custom numbering if you want

\newunicodechar{√}{$\sqrt{\;}$}
\newunicodechar{✅}{\checkmark}
\newunicodechar{❌}{\texttimes}
\usepackage{multicol}
\usepackage{listings}
\usepackage{url}
\usepackage{graphicx} % <-- needed for images
\def\UrlBreaks{\do\/\do-}

\usetheme{Boadilla}
\usecolortheme{lily}
\setbeamertemplate{footline}{
  \leavevmode%
  \hbox{%
  \begin{beamercolorbox}[wd=\paperwidth,ht=2ex,dp=1ex,right]{author in head/foot}%
    \insertframenumber{} / \inserttotalframenumber\hspace*{2ex}
  \end{beamercolorbox}}%
  \vskip0pt%
}
\setbeamertemplate{navigation symbols}{}

\lstset{
  frame=single,
  breaklines=true,
  columns=fullflexible,
  basicstyle=\ttfamily\tiny   % tiny font so code fits
}

\numberwithin{equation}{section}

% ---- your macros ----
\providecommand{\nCr}[2]{\,^{#1}{#2}}
\providecommand{\nPr}[2]{\,^{#1}P_{#2}}
\providecommand{\mbf}{\mathbf}
\providecommand{\pr}[1]{\ensuremath{\Pr\left(#1\right)}}
\providecommand{\qfunc}[1]{\ensuremath{Q\left(#1\right)}}
\providecommand{\sbrak}[1]{\ensuremath{{}\left[#1\right]}}
\providecommand{\lsbrak}[1]{\ensuremath{{}\left[#1\right.}}
\providecommand{\rsbrak}[1]{\ensuremath{\left.#1\right]}}
\providecommand{\brak}[1]{\ensuremath{\left(#1\right)}}
\providecommand{\lbrak}[1]{\ensuremath{\left(#1\right.}}
\providecommand{\rbrak}[1]{\ensuremath{\left.#1\right)}}
\providecommand{\cbrak}[1]{\ensuremath{\left\{#1\right\}}}
\providecommand{\lcbrak}[1]{\ensuremath{\left\{#1\right.}}
\providecommand{\rcbrak}[1]{\ensuremath{\left.#1\right\}}}
\theoremstyle{remark}


\providecommand{\abs}[1]{\left\vert#1\right\vert}
\providecommand{\res}[1]{\Res\displaylimits_{#1}}
\providecommand{\norm}[1]{\lVert#1\rVert}
\providecommand{\mtx}[1]{\mathbf{#1}}
\providecommand{\mean}[1]{E\left[ #1 \right]}
\providecommand{\fourier}{\overset{\mathcal{F}}{ \rightleftharpoons}}

\providecommand{\dec}[2]{\ensuremath{\overset{#1}{\underset{#2}{\gtrless}}}}

\usepackage{gvv}
% ---------------------

\title{Matgeo Presentation - Problem 5.4.16}
\author{ee25btech11021 - Dhanush sagar}

\begin{document}
	

		




%---------------- Title Page ----------------
\begin{frame}
  \titlepage
\end{frame}

%---------------- Problem Statement ----------------
\begin{frame}{Problem Statement}
Using elementary transformations, find the inverse of the following matrix. 
\begin{align*}
\myvec{2 & 5 \\ 1 & 3}
\end{align*}
\end{frame}

%---------------- Mathematical Formula ----------------
\begin{frame}{solution}
Given  
\begin{align}
\vec{A}=\myvec{2 & 5 \\ 1 & 3}
\end{align}
Let $\vec{A}^{-1}$ be the inverse of $\vec{A}$. Then
\begin{align}
    \vec{A}\vec{A}^{-1}=\vec{I}
\end{align}
Augmented matrix of $\augvec{1}{1}{\vec{A} & \vec{I}}$ is given by
\begin{align}
    \augvec{2}{2}{2 & 5 & 1 & 0 \\ 1 & 3 & 0 & 1}
\end{align}
Perform the elementary row operation $R_2 \rightarrow 2R_2 - R_1$ to eliminate the first column entry of $R_2$:
\begin{align}
    \augvec{2}{2}{2 & 5 & 1 & 0 \\[4pt] 1 & 3 & 0 & 1}
    \xrightarrow{R_2 \rightarrow 2R_2 - R_1}
    \augvec{2}{2}{2 & 5 & 1 & 0 \\[4pt] 0 & 1 & -1 & 2}
\end{align}
\end{frame}

\begin{frame}{solution}
Now eliminate the 5 above the (2,2) pivot by $R_1 \rightarrow R_1 - 5R_2$:
\begin{align}
    \augvec{2}{2}{2 & 5 & 1 & 0 \\[4pt] 0 & 1 & -1 & 2}
    \xrightarrow{R_1 \rightarrow R_1 - 5R_2}
    \augvec{2}{2}{2 & 0 & 6 & -10 \\[4pt] 0 & 1 & -1 & 2}
\end{align}
Finally make the leading entry of $R_1$ unity by $R_1 \rightarrow \tfrac{1}{2}R_1$:
\begin{align}
    \augvec{2}{2}{2 & 0 & 6 & -10 \\[4pt] 0 & 1 & -1 & 2}
    \xrightarrow{R_1 \rightarrow \tfrac{1}{2}R_1}
    \augvec{2}{2}{1 & 0 & 3 & -5 \\[4pt] 0 & 1 & -1 & 2}
\end{align}
Hence the inverse of the matrix $\myvec{2 & 5 \\ 1 & 3}$ is
\[
\vec{A}^{-1} = \myvec{3 & -5 \\ -1 & 2}.
\]
\end{frame}
%---------------- C Source Code ----------------
\begin{frame}[fragile]{C Source Code:matrix gen.c}
\begin{verbatim}
#include <stdio.h>


void get_matrix(double* mat) {
    mat[0] = 2;  mat[1] = 5;
    mat[2] = 1;  mat[3] = 3;
}
\end{verbatim}
\end{frame}

%---------------- Python solve.py ----------------
\begin{frame}[fragile]{Python Script:inverse matrix.py}
\begin{verbatim}
mport ctypes
import numpy as np
# Load the shared library
lib = ctypes.CDLL("./libmatrix.so")
# Define function signature: get_matrix(double* mat)
lib.get_matrix.argtypes = [ctypes.POINTER(ctypes.c_double)]
lib.get_matrix.restype = None
# Prepare a NumPy array (2x2) for the matrix
A = np.zeros((2, 2), dtype=np.double)
# Pass pointer to C function (as 1D flattened array)
lib.get_matrix(A.ctypes.data_as(ctypes.POINTER(ctypes.c_double)))
print("Matrix A =\n", A)
try:
    A_inv = np.linalg.inv(A)
    print("\nComputed A^{-1} =\n", A_inv)
except np.linalg.LinAlgError:
    print("Matrix is singular, no inverse exists.")
\end{verbatim}
\end{frame}


%---------------- Result Plot ----------------


\end{document}
