\documentclass[journal,12pt,onecolumn]{IEEEtran}
\usepackage{cite}
\usepackage{caption}
\usepackage{graphicx}
\usepackage{amsmath,amssymb,amsfonts,amsthm}
\usepackage{algorithmic}
\usepackage{graphicx}
\usepackage{textcomp}
\usepackage{xcolor}
\usepackage{tfrupee}
\usepackage{txfonts}
\usepackage{listings}
\usepackage{enumitem}
\usepackage{mathtools}
\usepackage{gensymb}
\usepackage{comment}
\usepackage[breaklinks=true]{hyperref}
\usepackage{tkz-euclide} 
\usepackage{listings}
\usepackage{gvv}
%\def\inputGnumericTable{}
\usepackage[latin1]{inputenc} 
\usetikzlibrary{arrows.meta, positioning}
\usepackage{xparse}
\usepackage{color}                                            
\usepackage{array}                                            
\usepackage{longtable}                                       
\usepackage{calc}                                             
\usepackage{multirow}
\usepackage{multicol}
\usepackage{hhline}                                           
\usepackage{ifthen}                                           
\usepackage{lscape}
\usepackage{tabularx}
\usepackage{array}
\usepackage{float}
\usepackage{marvosym}
\usepackage{float}
%\newcommand{\define}{\stackrel{\triangle}{=}}
\theoremstyle{remark}
\usepackage{circuitikz}
\captionsetup{justification=centering}
\usepackage{tikz}

\title{Matrices in Geometry 12.51}
\author{EE25BTECH11037 - Divyansh}
\begin{document}
\vspace{3cm}
\maketitle
{\let\newpage\relax\maketitle}
\textbf{Question: }
Let the eigenvalues of a square matrix $\vec{A}$ of order two be 1 and 2. The corresponding eigenvectors are $\myvec{0.6 \\ 0.8}$ and $\myvec{0.8 \\ -0.6}$, respectively. Then, the element $\vec{A}\brak{2, 2}$ is
\begin{multicols}{4}
    \begin{enumerate}[label=\alph*)]
        \item -0.48
        \item 0.48
        \item 1.36
        \item 1.64
    \end{enumerate}
\end{multicols}
\vspace{2mm}

\textbf{Solution:}\\
The eigenvalues of $\vec{A}$ are $\lambda_1=1$ and $\lambda_2=2$.\\
Let the given eigenvectors be
\begin{align}
    \vec{v_1}=\myvec{0.6 \\ 0.8} \ , \ \vec{v_2}=\myvec{0.8 \\ -0.6}
\end{align}
Let
\begin{align}
    \vec{P}=\myvec{\vec{v_1} & \vec{v_2}}=\myvec{0.6 & 0.8 \\ 0.8 & -0.6}
\end{align}
The given eigen vectors $\vec{v_1}$ and $\vec{v_2}$ are orthonormal, that is, they are unit vectors and their scalar product is zero. 
\begin{align}
    \therefore \vec{P}^{\top} = \vec{P}^{-1}
\end{align}
Using spectral decomposition, we can find the matrix $\vec{A}$.
\begin{align}
    \vec{A}=\vec{P}\vec{D}\vec{P}^{\top} \ , \ \vec{D}=\myvec{\lambda_1 & 0 \\ 0 & \lambda_2}=\myvec{1 & 0 \\ 0 &2}
\end{align}
\begin{align}
    \vec{A}=\myvec{0.6 & 0.8 \\ 0.8 & -0.6}\myvec{1 & 0 \\ 0 &2}\myvec{0.6 & 0.8 \\ 0.8 & -0.6}\\
    \implies \vec{A}=\myvec{1.64 & -0.48 \\ -0.48 & 1.36}
\end{align}
The element $\vec{A}\brak{2, 2} = 1.36$ which is option c)
\end{document}
