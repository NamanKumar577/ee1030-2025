\documentclass{beamer}
\usepackage[utf8]{inputenc}

\usetheme{Madrid}
\usecolortheme{default}
\usepackage{amsmath,amssymb,amsfonts,amsthm}
\usepackage{txfonts}
\usepackage{tkz-euclide}
\usepackage{listings}
\usepackage{adjustbox}
\usepackage{array}
\usepackage{tabularx}
\usepackage{gvv}
\usepackage{lmodern}
\usepackage{circuitikz}
\usepackage{tikz}
\usepackage{graphicx}
\usepackage{mathtools}
\setbeamertemplate{page number in head/foot}[totalframenumber]

\usepackage{tcolorbox}
\tcbuselibrary{minted,breakable,xparse,skins}



\definecolor{bg}{gray}{0.95}
\DeclareTCBListing{mintedbox}{O{}m!O{}}{%
  breakable=true,
  listing engine=minted,
  listing only,
  minted language=#2,
  minted style=default,
  minted options={%
    linenos,
    gobble=0,
    breaklines=true,
    breakafter=,,
    fontsize=\small,
    numbersep=8pt,
    #1},
  boxsep=0pt,
  left skip=0pt,
  right skip=0pt,
  left=25pt,
  right=0pt,
  top=3pt,
  bottom=3pt,
  arc=5pt,
  leftrule=0pt,
  rightrule=0pt,
  bottomrule=2pt,
  toprule=2pt,
  colback=bg,
  colframe=orange!70,
  enhanced,
  overlay={%
    \begin{tcbclipinterior}
    \fill[orange!20!white] (frame.south west) rectangle ([xshift=20pt]frame.north west);
    \end{tcbclipinterior}},
  #3,
}
\lstset{
    language=C,
    basicstyle=\ttfamily\small,
    keywordstyle=\color{blue},
    stringstyle=\color{orange},
    commentstyle=\color{green!60!black},
    numbers=left,
    numberstyle=\tiny\color{gray},
    breaklines=true,
    showstringspaces=false,
}
%This block of code defines the information to appear in the
%Title page
\title %optional
{5.5.11}
%\subtitle{A short story}

\author % (optional)
{Vaishnavi - EE25BTECH11059}



\begin{document}


\frame{\titlepage}
\begin{frame}{Question}
Find inverse of the following matrix,using elementary transformation\\
A=\myvec{2 & 0 & -1
          \\
        5 & 1 & 0
         \\
        0 & 1 & 3
}
\end{frame}

\begin{frame}{Solution}
Construct the augmented matrix of A and I
\begin{align}
&\left(
\begin{array}{ccc|ccc}
2 & 0 & -1 & 1 & 0 & 0 \\
5 & 1 & 0 & 0 & 1 & 0 \\
0 & 1 & 3 & 0 & 0 & 1
\end{array}
\right)
\xrightarrow{R_1 \gets \frac{1}{2}R_1}
\left(
\begin{array}{ccc|ccc}
1 & 0 & -\frac{1}{2} & \frac{1}{2} & 0 & 0 \\
5 & 1 & 0 & 0 & 1 & 0 \\
0 & 1 & 3 & 0 & 0 & 1
\end{array}
\right) \\[12pt]
&\xrightarrow{R_2 \gets R_2 - 5R_1}
\left(
\begin{array}{ccc|ccc}
1 & 0 & -\frac{1}{2} & \frac{1}{2} & 0 & 0 \\
0 & 1 & \frac{5}{2} & -\frac{5}{2} & 1 & 0 \\
0 & 1 & 3 & 0 & 0 & 1
\end{array}
\right)
\end{align}
\end{frame}

\begin{frame}{solution}
\begin{align}
&\xrightarrow{R_3 \gets R_3 - R_2}
\left(
\begin{array}{ccc|ccc}
1 & 0 & -\frac{1}{2} & \frac{1}{2} & 0 & 0 \\
0 & 1 & \frac{5}{2} & -\frac{5}{2} & 1 & 0 \\
0 & 0 & \frac{1}{2} & \frac{5}{2} & -1 & 1
\end{array}
\right) \\[12pt]
&\xrightarrow{R_3 \gets 2R_3}
\left(
\begin{array}{ccc|ccc}
1 & 0 & -\frac{1}{2} & \frac{1}{2} & 0 & 0 \\
0 & 1 & \frac{5}{2} & -\frac{5}{2} & 1 & 0 \\
0 & 0 & 1 & 5 & -2 & 2
\end{array}
\right)
\end{align}
\end{frame}
\begin{frame}{Solutions}
\begin{align}
&\xrightarrow{R_2 \gets R_2 - \frac{5}{2}R_3}
\left(
\begin{array}{ccc|ccc}
1 & 0 & -\frac{1}{2} & \frac{1}{2} & 0 & 0 \\
0 & 1 & 0 & -15 & 6 & -5 \\
0 & 0 & 1 & 5 & -2 & 2
\end{array}
\right) \\[12pt]
&\xrightarrow{R_1 \gets R_1 + \frac{1}{2}R_3}
\left(
\begin{array}{ccc|ccc}
1 & 0 & 0 & 3 & -1 & 1 \\
0 & 1 & 0 & -15 & 6 & -5 \\
0 & 0 & 1 & 5 & -2 & 2
\end{array}
\right)
\end{align}
\end{frame}
\begin{frame}{Solutions}
  \begin{align}
A^{-1}
   =\myvec{3 & -1 & 1
          \\
        -15 & 6 & -5
         \\
        5 & -2 & 2
}
\end{align}  
\end{frame}





\begin{frame}[fragile]
    \frametitle{Python Code}
    \begin{lstlisting}
import numpy as np

# Define the matrix A
A = np.array([[2, 0, -1],
              [5, 1, 0],
              [0, 1, 3]])

# Calculate the inverse of A
A_inv = np.linalg.inv(A)

print("Inverse of the matrix A is:")
print(A_inv)
\end{lstlisting}
\end{frame}

\begin{frame}[fragile]
\frametitle{C Code}
\begin{lstlisting}
#include <stdio.h>

void invert_matrix(double A[3][3], double inv[3][3]) {
    int i, j, k;
    double aug[3][6];
    // Build augmented matrix [A | I]
    for (i = 0; i < 3; ++i) {
        for (j = 0; j < 3; ++j) {
            aug[i][j] = A[i][j];
            aug[i][j+3] = (i == j) ? 1.0 : 0.0;
        }
    }
          
    \end{lstlisting}

\end{frame}
\begin{frame}[fragile]
\frametitle{C Code}
\begin{lstlisting}
  // Forward elimination
    for (i = 0; i < 3; ++i) {
        double f = aug[i][i];
        for (j = 0; j < 6; ++j)
            aug[i][j] /= f;
        for (k = 0; k < 3; ++k) {
            if(k != i) {
                double f2 = aug[k][i];
                for (j = 0; j < 6; ++j)
                    aug[k][j] -= f2 * aug[i][j];
            }
        }
    }
    // Extract inverse
    for (i = 0; i < 3; ++i)
        for(j = 0; j < 3; ++j)
            inv[i][j] = aug[i][j+3];
}
    \end{lstlisting}

\end{frame}


\begin{frame}[fragile]
\frametitle{Python and C Code}

\begin{lstlisting}
import ctypes
import numpy as np

# Load shared library
lib = ctypes.CDLL('./code.so')

# Prepare arguments
A = np.array([[2, 0, -1],
              [5, 1, 0],
              [0, 1, 3]], dtype=np.float64)
A_inv = np.zeros((3, 3), dtype=np.float64)

# Set up argtypes for the C function
lib.invert_matrix.argtypes = [ctypes.POINTER(ctypes.c_double * 3 * 3), ctypes.POINTER(ctypes.c_double * 3 * 3)]


\end{lstlisting}

\end{frame}

\begin{frame}[fragile]
\frametitle{Python and C Code}

\begin{lstlisting}
# Create ctypes pointers
A_ct = (ctypes.c_double * 3 * 3)(*A.flatten())
A_inv_ct = (ctypes.c_double * 3 * 3)()

# Call C function
lib.invert_matrix(ctypes.byref(A_ct), ctypes.byref(A_inv_ct))

# Convert result back to numpy
A_inv_result = np.array(A_inv_ct).reshape((3, 3))
print("Inverse matrix from C:\n", A_inv_result)

\end{lstlisting}

\end{frame}

\end{document}