
\documentclass{beamer}
\mode<presentation>
\usepackage{amsmath}
\usepackage{amssymb}
%\usepackage{advdate}
\usepackage{graphicx}
\graphicspath{{../figs/}}
\usepackage{adjustbox}
\usepackage{subcaption}
\usepackage{enumitem}
\usepackage{multicol}
\usepackage{mathtools}
\usepackage{listings}
\usepackage{url}
\def\UrlBreaks{\do\/\do-}
\usetheme{Boadilla}
\usecolortheme{lily}
\setbeamertemplate{footline}
{
  \leavevmode%
  \hbox{%
  \begin{beamercolorbox}[wd=\paperwidth,ht=2.25ex,dp=1ex,right]{author in head/foot}%
    \insertframenumber{} / \inserttotalframenumber\hspace*{2ex} 
  \end{beamercolorbox}}%
  \vskip0pt%
}
\setbeamertemplate{navigation symbols}{}
\let\solution\relax
\usepackage{gvv}
\lstset{
%language=C,
frame=single, 
breaklines=true,
columns=fullflexible
}

\numberwithin{equation}{section}
\title{12.495}
\author{AI25BTECH11001 - ABHISEK MOHAPATRA}
\begin{document}
{\let\newpage\relax\maketitle}
\renewcommand{\thefigure}{\theenumi}
\renewcommand{\thetable}{\theenumi}


	 	\textbf{Question}:
The directional derivative of the function

\begin{align}
		f(x, y) =\frac{x^2 + xy^2}{\sqrt{5}}
\end{align}
in the direction
\begin{align}
		\vec{d} = \myvec{2\\-4}
\end{align}

at X = $\myvec{1\\ 1}$ is

a) -$\frac{1}{\sqrt{5}}$

b) -$\frac{2}{\sqrt{5}}$

c) 0

d) -$\frac{1}{3}$

		\textbf{Solution:}

Let $\vec{R}$ = $\myvec{\cos{\theta}&\sin{\theta}\\-\sin{\theta}&\cos{\theta}}$ be a rotaion matrix such that $\vec{R}\vec{d} = \vec{e}_1$.


\begin{align}
		\Rightarrow\myvec{\cos{\theta}&\sin{\theta}\\-\sin{\theta}&\cos{\theta}} \myvec{2\\-4} = \myvec{1\\0}
\end{align}

\begin{align}
		2\cos{\theta} - 4\sin{\theta} = 1
\end{align}
\begin{align}
		-2\sin{\theta} - 4\cos{\theta} = 0
\end{align}
Combing the two equations,
\begin{align}
		\myvec{2 &- 4\\-4&-2}\myvec{\cos{\theta}\\\sin{\theta}} = \myvec{1\\0}
\end{align}
\begin{align}
	\Rightarrow	\myvec{\cos{\theta}\\\sin{\theta}} = -\frac{1}{10}\myvec{-1&2\\2&1}\myvec{1\\0} = \frac{1}{10}\myvec{1\\-2}
\end{align}

So, 
\begin{align}
		\vec{R}=\frac{1}{10}\myvec{1&-2\\2&1}
\end{align}

\begin{align}
\vec{X}^` = \vec{R}\vec{X}
\end{align}
\begin{align}
	\Rightarrow	\vec{X} = \vec{R}^{-1}\vec{X}^` = 2\myvec{1&2\\-2&1}\myvec{x_1\\y_1} = 2\myvec{x`+2y`\\-2x`+y`}
\end{align}
\begin{align}
		so, f(x,y) = \frac{1}{\sqrt{5}}\brak{4\brak{x`+2y`}^2 + 8\brak{x`+2y`}\brak{2x`-y`}^2}
\end{align}
\begin{align}
		so, \frac{\partial f(x,y)}{\partial x`} = \frac{1}{\sqrt{5}}\brak{8\brak{x`+2y`} +8\brak{2x`-y`}^2+32\brak{x`+2y`}\brak{2x`-y`}}
\end{align}
for $\myvec{1\\1}$
\begin{align}
		\vec{X_o} = \vec{R}\myvec{1\\1} = \frac{1}{10}\myvec{-1\\3}
\end{align}
\begin{align}
		so, \frac{\partial f(1,1)}{\partial x`} = \frac{1}{\sqrt{5}}\brak{4+2-8} = -\frac{2}{\sqrt{5}} (b)
\end{align}



\end{document}




