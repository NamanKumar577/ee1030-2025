\let\negmedspace\undefined
\let\negthickspace\undefined
\documentclass[journal,12pt,onecolumn]{IEEEtran}
\usepackage{cite}
\usepackage{amsmath,amssymb,amsfonts,amsthm}
\usepackage{algorithmic}
\usepackage{graphicx}
\graphicspath{{./figs/}}
\usepackage{textcomp}
\usepackage{xcolor}
\usepackage{txfonts}
\usepackage{listings}
\usepackage{enumitem}
\usepackage{mathtools}
\usepackage{gensymb}
\usepackage{comment}
\usepackage{caption}
\usepackage[breaklinks=true]{hyperref}
\usepackage{tkz-euclide} 
\usepackage{listings}
\usepackage{gvv}                                        
%\def\inputGnumericTable{}                                 
\usepackage[latin1]{inputenc}     
\usepackage{xparse}
\usepackage{color}                                            
\usepackage{array}                                            
\usepackage{longtable}                                       
\usepackage{calc}                                             
\usepackage{multirow}
\usepackage{multicol}
\usepackage{hhline}                                           
\usepackage{ifthen}                                           
\usepackage{lscape}
\usepackage{tabularx}
\usepackage{array}
\usepackage{float}

\begin{document}

\title{12.391}
\author{AI25BTECH11001 - ABHISEK MOHAPATRA}
{\let\newpage\relax\maketitle}
	
	 	\textbf{Question}:
The directional derivative of the function

\begin{align}
		f(x, y) =\frac{x^2 + xy^2}{\sqrt{5}}
\end{align}
in the direction
\begin{align}
		\vec{d} = \myvec{2\\-4}
\end{align}

at X = $\myvec{1\\ 1}$ is

a) -$\frac{1}{\sqrt{5}}$

b) -$\frac{2}{\sqrt{5}}$

c) 0

d) -$\frac{1}{3}$

		\textbf{Solution:}

Let $\vec{R}$ = $\myvec{\cos{\theta}&\sin{\theta}\\-\sin{\theta}&\cos{\theta}}$ be a rotaion matrix such that $\vec{R}\vec{d} = \vec{e}_1$.


\begin{align}
		\Rightarrow\myvec{\cos{\theta}&\sin{\theta}\\-\sin{\theta}&\cos{\theta}} \myvec{2\\-4} = \myvec{1\\0}
\end{align}

\begin{align}
		2\cos{\theta} - 4\sin{\theta} = 1
\end{align}
\begin{align}
		-2\sin{\theta} - 4\cos{\theta} = 0
\end{align}
Combing the two equations,
\begin{align}
		\myvec{2 &- 4\\-4&-2}\myvec{\cos{\theta}\\\sin{\theta}} = \myvec{1\\0}
\end{align}
\begin{align}
	\Rightarrow	\myvec{\cos{\theta}\\\sin{\theta}} = -\frac{1}{10}\myvec{-1&2\\2&1}\myvec{1\\0} = \frac{1}{10}\myvec{1\\-2}
\end{align}

So, 
\begin{align}
		\vec{R}=\frac{1}{10}\myvec{1&-2\\2&1}
\end{align}

\begin{align}
\vec{X}^` = \vec{R}\vec{X}
\end{align}
\begin{align}
	\Rightarrow	\vec{X} = \vec{R}^{-1}\vec{X}^` = 2\myvec{1&2\\-2&1}\myvec{x_1\\y_1} = 2\myvec{x`+2y`\\-2x`+y`}
\end{align}
\begin{align}
		so, f(x,y) = \frac{1}{\sqrt{5}}\brak{4\brak{x`+2y`}^2 + 8\brak{x`+2y`}\brak{2x`-y`}^2}
\end{align}
\begin{align}
		so, \frac{\partial f(x,y)}{\partial x`} = \frac{1}{\sqrt{5}}\brak{8\brak{x`+2y`} +8\brak{2x`-y`}^2+32\brak{x`+2y`}\brak{2x`-y`}}
\end{align}
for $\myvec{1\\1}$
\begin{align}
		\vec{X_o} = \vec{R}\myvec{1\\1} = \frac{1}{10}\myvec{-1\\3}
\end{align}
\begin{align}
		so, \frac{\partial f(1,1)}{\partial x`} = \frac{1}{\sqrt{5}}\brak{4+2-8} = -\frac{2}{\sqrt{5}} (b)
\end{align}



\end{document}



