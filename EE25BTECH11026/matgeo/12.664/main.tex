\let\negmedspace\undefined
\let\negthickspace\undefined
\documentclass[journal]{IEEEtran}
\usepackage[a5paper, margin=10mm, onecolumn]{geometry}
%\usepackage{lmodern} % Ensure lmodern is loaded for pdflatex
\usepackage{tfrupee} % Include tfrupee package

\setlength{\headheight}{1cm} % Set the height of the header box
\setlength{\headsep}{0mm}     % Set the distance between the header box and the top of the text

\usepackage{gvv-book}
\usepackage{gvv}
\usepackage{cite}
\usepackage{amsmath,amssymb,amsfonts,amsthm}
\usepackage{algorithmic}
\usepackage{graphicx}
\usepackage{textcomp}
\usepackage{xcolor}
\usepackage{txfonts}
\usepackage{listings}
\usepackage{enumitem}
\usepackage{mathtools}
\usepackage{gensymb}
\usepackage[breaklinks=true]{hyperref}
\usepackage{tkz-euclide} 
\usepackage{listings}
% \usepackage{gvv}                                        
\def\inputGnumericTable{}                                 
\usepackage[latin1]{inputenc}                                
\usepackage{color}                                            
\usepackage{array}                                            
\usepackage{longtable}                                       
\usepackage{calc}                                             
\usepackage{multirow}                                         
\usepackage{hhline}                                           
\usepackage{ifthen}                                           
\usepackage{lscape}
\usepackage{circuitikz}
\usepackage{comment}
\tikzstyle{block} = [rectangle, draw, fill=blue!20, 
    text width=4em, text centered, rounded corners, minimum height=3em]
\tikzstyle{sum} = [draw, fill=blue!10, circle, minimum size=1cm, node distance=1.5cm]
\tikzstyle{input} = [coordinate]
\tikzstyle{output} = [coordinate]


\begin{document}

\bibliographystyle{IEEEtran}
\vspace{3cm}

\title{12.664}
\author{EE25BTECH11026-Harsha}
 \maketitle
% \newpage
% \bigskip
{\let\newpage\relax\maketitle}

\renewcommand{\thefigure}{\theenumi}
\renewcommand{\thetable}{\theenumi}
\setlength{\intextsep}{10pt} % Space between text and floats


\numberwithin{equation}{enumi}
\numberwithin{figure}{enumi}
\renewcommand{\thetable}{\theenumi}

\textbf{Question}:\\
A real, invertible $3 \times 3$ matrix $\vec{M}$ has eigenvalues $\lambda_i$, $\brak{i=1,2,3}$ and the corresponding eigenvectors are $\vec{e_i}$, $\brak{i=1,2,3}$ respectively. Which one of the following is correct?
\begin{enumerate}
    \item $\vec{M}\vec{e_i}=\frac{1}{\lambda_i}\vec{e_i}$, for i=1,2,3
    \item $\vec{M}^{-1}\vec{e_i}=\frac{1}{\lambda_i}\vec{e_i}$, for i=1,2,3
    \item $\vec{M}^{-1}\vec{e_i}=\lambda_i\vec{e_i}$, for i=1,2,3
    \item The eigenvalues of $\vec{M}$ and $\vec{M}^{-1}$ are not related.
\end{enumerate}
\solution \\
Let us solve the given question theoretically and then verify the solution computationally.\\
\\
According to the definition of eigen-vector,
\begin{align}
    \vec{M}\vec{e_i}=\lambda_i\vec{e_i}
\end{align}
Pre-multiplying $\vec{M}^{-1}$ on both sides,
\begin{align}
    \therefore \brak{\vec{M}^{-1}\vec{M}}\vec{e_i}=\vec{M}^{-1}\lambda_i\vec{e_i}
\end{align}
\begin{align}
    \implies \vec{e_i}=\lambda_i\vec{M}^{-1}\vec{e_i}
\end{align}
\begin{align}
    \therefore \vec{M}^{-1}\vec{e_i}=\frac{1}{\lambda_i}\vec{e_i}
\end{align}



\end{document}

