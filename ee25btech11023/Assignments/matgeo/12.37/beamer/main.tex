\documentclass{beamer}
\usepackage[utf8]{inputenc}

\usetheme{Boadilla}
\usecolortheme{lily}
\usepackage{amsmath,amssymb,amsfonts,amsthm}
\usepackage{mathtools}
\usepackage{txfonts}
\usepackage{tkz-euclide}
\usepackage{listings}
\usepackage{adjustbox}
\usepackage{array}
\usepackage{tabularx}
\usepackage{lmodern}
\usepackage{gvv}
\usepackage{circuitikz}
\usepackage{tikz}
\usepackage{graphicx}

\setbeamertemplate{footline}
{
  \leavevmode%
  \hbox{%
  \begin{beamercolorbox}[wd=\paperwidth,ht=2.25ex,dp=1ex,right]{author in head/foot}%
    \insertframenumber{} / \inserttotalframenumber\hspace*{2ex} 
  \end{beamercolorbox}}%
  \vskip0pt%
}

\usepackage{tcolorbox}
\tcbuselibrary{minted,breakable,xparse,skins}




\providecommand{\nCr}[2]{\,^{#1}C_{#2}} % nCr
\providecommand{\nPr}[2]{\,^{#1}P_{#2}} % nPr
\providecommand{\mbf}{\mathbf}
\providecommand{\pr}[1]{\ensuremath{\Pr\left(#1\right)}}
\providecommand{\qfunc}[1]{\ensuremath{Q\left(#1\right)}}
\providecommand{\sbrak}[1]{\ensuremath{{}\left[#1\right]}}
\providecommand{\lsbrak}[1]{\ensuremath{{}\left[#1\right.}}
\providecommand{\rsbrak}[1]{\ensuremath{{}\left.#1\right]}}
\providecommand{\brak}[1]{\ensuremath{\left(#1\right)}}
\providecommand{\lbrak}[1]{\ensuremath{\left(#1\right.}}
\providecommand{\rbrak}[1]{\ensuremath{\left.#1\right)}}
\providecommand{\cbrak}[1]{\ensuremath{\left\{#1\right\}}}
\providecommand{\lcbrak}[1]{\ensuremath{\left\{#1\right.}}
\providecommand{\rcbrak}[1]{\ensuremath{\left.#1\right\}}}
\theoremstyle{remark}
\newcommand{\sgn}{\mathop{\mathrm{sgn}}}
\providecommand{\abs}[1]{\left\vert#1\right\vert}
\providecommand{\res}[1]{\Res\displaylimits_{#1}} 
\providecommand{\norm}[1]{\lVert#1\rVert}
\providecommand{\mtx}[1]{\mathbf{#1}}
\providecommand{\mean}[1]{E\left[ #1 \right]}
\providecommand{\fourier}{\overset{\mathcal{F}}{ \rightleftharpoons}}
%\providecommand{\hilbert}{\overset{\mathcal{H}}{ \rightleftharpoons}}
\providecommand{\system}{\overset{\mathcal{H}}{ \longleftrightarrow}}
	%\newcommand{\solution}[2]{\textbf{Solution:}{#1}}
%\newcommand{\solution}{\noindent \textbf{Solution: }}
\providecommand{\dec}[2]{\ensuremath{\overset{#1}{\underset{#2}{\gtrless}}}}
\newcommand{\myvec}[1]{\ensuremath{\begin{pmatrix}#1\end{pmatrix}}}
\let\vec\mathbf

\lstset{
%language=C,
frame=single, 
breaklines=true,
columns=fullflexible
}

\numberwithin{equation}{section}

\lstset{
  language=Python,
  basicstyle=\ttfamily\small,
  keywordstyle=\color{blue},
  stringstyle=\color{orange},
  numbers=left,
  numberstyle=\tiny\color{gray},
  breaklines=true,
  showstringspaces=false
}

\title{Problem 12.37}
\author{ee25btech11023-Venkata Sai}

\date{\today} 
\begin{document}

\begin{frame}
\titlepage
\end{frame}

\section*{Outline}
\begin{frame}
\tableofcontents
\end{frame}

\section{Problem}

\begin{frame}
\frametitle{Problem}
Let $\mathcal{M}$ be the set of $3\times3 $ real symmetric positive definite matrices. Consider S =
$\cbrak{\vec{A} \in \mathcal{M}: \vec{A}^{50} - \vec{A}^{48} = 0 }$. The number of elements in S equals
\end{frame}
%\subsection{Literature}
\section{Solution}

 
\subsection{Condition}
\begin{frame}
\frametitle{Condition}
 If a matrix is symmetric then it is  diagonalizable. Hence
 \begin{align}
     \vec{A}&=\vec{P}\vec{D}\vec{P}^{-1} \\
      \vec{A}^2&=\brak{\vec{P}\vec{D}\vec{P}^{-1}}^2 \\
      &=\vec{P}\vec{D}\vec{P}^{-1}\vec{P}\vec{D}\vec{P}^{-1}\\
      &=\vec{P}\vec{D}\vec{I}\vec{D}\vec{P}^{-1}\\
      &=\vec{P}\vec{D}^2\vec{P}^{-1}
 \end{align}
 \begin{align}
     \vec{A}^k&=\vec{P}\vec{D}^k\vec{P}^{-1}\\
     \vec{A}^{50}&=\vec{P}\vec{D}^{50}\vec{P}^{-1}\\
     \vec{A}^{48}&=\vec{P}\vec{D}^{48}\vec{P}^{-1}
 \end{align}
 Given 
 \begin{align}
   \vec{A}^{50} - \vec{A}^{48} = 0\\
   \vec{P}\vec{D}^{50}\vec{P}^{-1}-\vec{P}\vec{D}^{48}\vec{P}^{-1}=0 
   \end{align}
\end{frame}
\subsection{Conclusion}
\begin{frame}
\frametitle{Conclusion}
  \begin{align}
   \vec{P}\brak{\vec{D}^{50}-\vec{D}^{48}}\vec{P}^{-1}=0 \\
   \implies \brak{\vec{D}^{50}-\vec{D}^{48}}=0\\
   \implies \brak{\vec{\lambda}^{50}-\vec{\lambda}^{48}}=0
 \end{align}
 where $\lambda$ are the eigen values
\begin{align}
    \lambda^{48}\brak{\lambda^2-1}=0&\implies
    \lambda^{48}=0\ \text{or}\ \lambda^{2}-1=0 \\
     \lambda=0\ &\text{or}\ \lambda=\pm1
\end{align}
For a positive definite matrix, eigen values must be greater than 0. Hence
\begin{align}
    \lambda=1=\lambda_1=\lambda_2=\lambda_3 &\implies \vec{D}=\myvec{\lambda_1&0&0\\0&\lambda_2&0\\0&0&\lambda_3}=\myvec{1&0&0\\0&1&0\\0&0&1}=\vec{I}\\
    \vec{A}&=\vec{P}\vec{I}\vec{P}^{-1}
    =\vec{P}\vec{P}^{-1}=\vec{I}
\end{align}
Hence Number of elements in S is 1
 \end{frame}
 
\end{document}
