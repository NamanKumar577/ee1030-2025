\documentclass{beamer}
\usepackage[utf8]{inputenc}

\usetheme{Boadilla}
\usecolortheme{lily}
\usepackage{amsmath,amssymb,amsfonts,amsthm}
\usepackage{mathtools}
\usepackage{txfonts}
\usepackage{tkz-euclide}
\usepackage{listings}
\usepackage{multicol}
\usepackage{adjustbox}
\usepackage{array}
\usepackage{tabularx}
\usepackage{lmodern}
\usepackage{gvv}
\usepackage{circuitikz}
\usepackage{tikz}
\usepackage{graphicx}

\setbeamertemplate{footline}
{
  \leavevmode%
  \hbox{%
  \begin{beamercolorbox}[wd=\paperwidth,ht=2.25ex,dp=1ex,right]{author in head/foot}%
    \insertframenumber{} / \inserttotalframenumber\hspace*{2ex} 
  \end{beamercolorbox}}%
  \vskip0pt%
}

\usepackage{tcolorbox}
\tcbuselibrary{minted,breakable,xparse,skins}




\providecommand{\nCr}[2]{\,^{#1}C_{#2}} % nCr
\providecommand{\nPr}[2]{\,^{#1}P_{#2}} % nPr
\providecommand{\mbf}{\mathbf}
\providecommand{\pr}[1]{\ensuremath{\Pr\left(#1\right)}}
\providecommand{\qfunc}[1]{\ensuremath{Q\left(#1\right)}}
\providecommand{\sbrak}[1]{\ensuremath{{}\left[#1\right]}}
\providecommand{\lsbrak}[1]{\ensuremath{{}\left[#1\right.}}
\providecommand{\rsbrak}[1]{\ensuremath{{}\left.#1\right]}}
\providecommand{\brak}[1]{\ensuremath{\left(#1\right)}}
\providecommand{\lbrak}[1]{\ensuremath{\left(#1\right.}}
\providecommand{\rbrak}[1]{\ensuremath{\left.#1\right)}}
\providecommand{\cbrak}[1]{\ensuremath{\left\{#1\right\}}}
\providecommand{\lcbrak}[1]{\ensuremath{\left\{#1\right.}}
\providecommand{\rcbrak}[1]{\ensuremath{\left.#1\right\}}}
\theoremstyle{remark}
\newcommand{\sgn}{\mathop{\mathrm{sgn}}}
\providecommand{\abs}[1]{\left\vert#1\right\vert}
\providecommand{\res}[1]{\Res\displaylimits_{#1}} 
\providecommand{\norm}[1]{\lVert#1\rVert}
\providecommand{\mtx}[1]{\mathbf{#1}}
\providecommand{\mean}[1]{E\left[ #1 \right]}
\providecommand{\fourier}{\overset{\mathcal{F}}{ \rightleftharpoons}}
%\providecommand{\hilbert}{\overset{\mathcal{H}}{ \rightleftharpoons}}
\providecommand{\system}{\overset{\mathcal{H}}{ \longleftrightarrow}}
	%\newcommand{\solution}[2]{\textbf{Solution:}{#1}}
%\newcommand{\solution}{\noindent \textbf{Solution: }}
\providecommand{\dec}[2]{\ensuremath{\overset{#1}{\underset{#2}{\gtrless}}}}
\newcommand{\myvec}[1]{\ensuremath{\begin{pmatrix}#1\end{pmatrix}}}
\let\vec\mathbf

\lstset{
%language=C,
frame=single, 
breaklines=true,
columns=fullflexible
}

\numberwithin{equation}{section}

\lstset{
  language=Python,
  basicstyle=\ttfamily\small,
  keywordstyle=\color{blue},
  stringstyle=\color{orange},
  numbers=left,
  numberstyle=\tiny\color{gray},
  breaklines=true,
  showstringspaces=false
}

\title{Problem 12.141}
\author{ee25btech11023-Venkata Sai}

\date{\today} 
\begin{document}

\begin{frame}
\titlepage
\end{frame}

\section*{Outline}
\begin{frame}
\tableofcontents
\end{frame}

\section{Problem}

\begin{frame}
\frametitle{Problem}
Let $\vec{A}$ be a 3$\times$3 matrix. Suppose that the eigenvalues of $\vec{A}$ are -1, 0, 1 with respective
eigenvectors $(1, -1, 0)^\top , (1, 1, -2)^\top\ \text{and}\ (1, 1, 1)^\top$ . Then 6$\vec{A}$ equals
\begin{enumerate}
\begin{multicols}{4}
    \item \myvec{-1&5&2\\5&-1&2\\2&2&2}
    \item \myvec{1&0&0\\0&-1&0\\0&0&0}
    \item \myvec{1&5&3\\5&1&3\\3&3&3}
\item \myvec{-3&9&0\\9&-3&0\\0&0&6}
\end{multicols}
\end{enumerate}
\end{frame}
%\subsection{Literature}
\section{Solution}

 
\subsection{Equation}
\begin{frame}
\frametitle{Equation}
 For an invertible matrix $\vec{P}$
 \begin{align}
     \vec{A}=\vec{P}\vec{D}\vec{P}^{-1}
 \end{align}
 Given eigen values are
 \begin{align}
    \lambda_1=-1,\lambda_2=0,\lambda_3=1
 \end{align}
 Given eigen vectors are
 \begin{align}
\vec{x_1}=\myvec{1\\-1\\0},\vec{x_2}=\myvec{1\\1\\-2},\vec{x_3}=\myvec{1\\1\\1}
 \end{align}
 where
 \begin{align}
\vec{D}=\myvec{\lambda_1&0&0\\0&\lambda_2&0\\0&0&\lambda_3}=\myvec{-1&0&0\\0&0&0\\0&0&1}\\
 \end{align}
\end{frame}
\subsection{Augmented matrix}
\begin{frame}
\frametitle{Augmented matrix }
 \begin{align}
\vec{P}=\myvec{\vec{x_1} & \vec{x_2} & \vec{x_3}}=\myvec{1&1&1\\-1&1&1\\0&-2&1} 
\end{align}
 \begin{align}
  |\vec{P}|=1\brak{1+2}-1\brak{-1+0}+1\brak{2+0}=3+1+2=6 \neq 0
 \end{align}
 \begin{align}
     \vec{P}\vec{P}^{-1}=\vec{I}
 \end{align}
 Augmented matrix of $\augvec{1}{1}{\vec{P} & \vec{I}}$ is given by
 \begin{align}
     \augvec{3}{3}{
1 & 1 & 1  & 1 & 0 & 0 \\
-1 & 1 & 1 & 0 & 1 & 0\\
0 & -2 & 1 & 0 & 0 & 1}
& \xrightarrow{R_2\rightarrow \frac{1}{2}\brak{R_1+R_2}}  \augvec{3}{3}{
1 & 1 & 1  & 1 & 0 & 0 \\
0 & 1 & 1 & \frac{1}{2} & \frac{1}{2} & 0\\
0 & -2 & 1 & 0 & 0 & 1}   \\
 \end{align}
 \end{frame}
 \subsection{Operations}
\begin{frame}
\frametitle{Operations}
\begin{align}
\augvec{3}{3}{
1 & 1 & 1  & 1 & 0 & 0 \\
0 & 1 & 1 & \frac{1}{2} & \frac{1}{2} & 0\\
0 & -2 & 1 & 0 & 0 & 1} &\xrightarrow[R_1\rightarrow R_1-R_2]{R_3\rightarrow R_3+2R_2}\augvec{3}{3}{
1 & 0 & 0  &  \frac{1}{2} & -\frac{1}{2} & 0 \\
0 & 1 & 1 & \frac{1}{2} & \frac{1}{2} & 0\\
0 & 0 & 3 & 1 & 1 & 1} \\
\augvec{3}{3}{
1 & 0 & 0  &  \frac{1}{2} & -\frac{1}{2} & 0 \\
0 & 1 & 1 & \frac{1}{2} & \frac{1}{2} & 0\\
0 & 0 & 3 & 1 & 1 & 1}  &\xrightarrow{R_2\rightarrow R_2-\frac{1}{3}R_3} \augvec{3}{3}{
1 & 0 & 0  &  \frac{1}{2} & -\frac{1}{2} & 0 \\
0 & 1 & 0 & \frac{1}{6} & \frac{1}{6} & -\frac{1}{3}\\
0 & 0 & 1 & \frac{1}{3} & \frac{1}{3} & \frac{1}{3}} 
 \end{align}
 \begin{align}
    \vec{P}^{-1}=\myvec{\frac{1}{2} & -\frac{1}{2} & 0 \\\frac{1}{6} & \frac{1}{6} & -\frac{1}{3}\\ \frac{1}{3} & \frac{1}{3} & \frac{1}{3}}
 \end{align}
 \begin{align}
6\vec{A}=6\vec{P}\vec{D}\vec{P}^{-1}=\brak{\vec{P}\vec{D}}\brak{6\vec{P}^{-1}} 
\end{align}
 \end{frame}
\subsection{Conclusion}
\begin{frame}[fragile]
\frametitle{Conclusion}
 \begin{align}
6\vec{A}&=\brak{\myvec{1&1&1\\-1&1&1\\0&-2&1}\myvec{-1&0&0\\0&0&0\\0&0&1}}\brak{6\myvec{\frac{1}{2} & -\frac{1}{2} & 0 \\\frac{1}{6} & \frac{1}{6} & -\frac{1}{3}\\ \frac{1}{3} & \frac{1}{3} & \frac{1}{3}}}\\
&=\brak{\myvec{1&1&1\\-1&1&1\\0&-2&1}\myvec{-1&0&0\\0&0&0\\0&0&1} }\myvec{3 & -3 & 0 \\1 & 1 & -2\\ 2 & 2 & 2}\\
&=\myvec{-1 & 0 & 1 \\ 1& 0 & 1 \\0 & 0 & 1}\myvec{3 & -3 & 0 \\1 & 1 & -2\\ 2 & 2 & 2}
=\myvec{-3+2& 3+2&0+2\\3+2&-3+2&0+2\\0+2&0+2&0+2} \\
&=\myvec{-1&5&2\\5&-1&2\\2&2&2}
\end{align}
\end{frame}

\section{C Code}
\begin{frame}[fragile]
\frametitle{C Code}
\begin{lstlisting}[language=C]
void get_eigen_data(double* out_data) {
    // Eigenvalues
    out_data[0] = -1.0;
    out_data[1] = 0.0;
    out_data[2] = 1.0;
    // Eigenvector 1,
    out_data[3] = 1.0;
    out_data[4] = -1.0;
    out_data[5] = 0.0;
    // Eigenvector 2
    out_data[6] = 1.0;
    out_data[7] = 1.0;
    out_data[8] = -2.0;
    // Eigenvector 3
    out_data[9] = 1.0;
    out_data[10] = 1.0;
    out_data[11] = 1.0;
}
\end{lstlisting}
\end{frame}

\section{Python Code}
\begin{frame}[fragile]
\frametitle{Python Code for Solving}
\begin{lstlisting}[language=Python]
import ctypes
import numpy as np

def calculate():
    lib = ctypes.CDLL('./code.so')
    double_array_12 = ctypes.c_double * 12
    lib.get_eigen_data.argtypes = [ctypes.POINTER(ctypes.c_double)]

    out_data_c = double_array_12()
    lib.get_eigen_data(out_data_c)
    raw_data = np.array(list(out_data_c))

    eigenvalues = raw_data[0:3]
    v1 = raw_data[3:6]
    v2 = raw_data[6:9]
    v3 = raw_data[9:12]
\end{lstlisting}
\end{frame}
 \begin{frame}[fragile]
\frametitle{Python Code for Solving}
\begin{lstlisting} 
D = np.diag(eigenvalues)
    P = np.vstack([v1, v2, v3]).T
    P_inv = np.linalg.inv(P)
    A = P @ D @ P_inv
    result_6A = 6 * A
    
    return result_6A

if __name__ == '__main__':
    final_result = calculate()
    print("\n 6A =\n", final_result)
\end{lstlisting}
\end{frame}
 
\end{document}
