\let\negmedspace\undefined
\let\negthickspace\undefined
\documentclass[journal]{IEEEtran}
\usepackage[a5paper, margin=10mm, onecolumn]{geometry}
%\usepackage{lmodern} % Ensure lmodern is loaded for pdflatex
\usepackage{tfrupee} % Include tfrupee package

\setlength{\headheight}{1cm} % Set the height of the header box
\setlength{\headsep}{0mm}     % Set the distance between the header box and the top of the text

\usepackage{gvv-book}
\usepackage{gvv}
\usepackage{cite}
\usepackage{amsmath,amssymb,amsfonts,amsthm}
\usepackage{algorithmic}
\usepackage{graphicx}
\usepackage{textcomp}
\usepackage{xcolor}
\usepackage{txfonts}
\usepackage{listings}
\usepackage{enumitem}
\usepackage{mathtools}
\usepackage{gensymb}
\usepackage{comment}
\usepackage[breaklinks=true]{hyperref}
\usepackage{tkz-euclide} 
\usepackage{listings}
% \usepackage{gvv}                                        
\def\inputGnumericTable{}                                 
\usepackage[latin1]{inputenc}                                
\usepackage{color}                                            
\usepackage{array}                                            
\usepackage{longtable}                                       
\usepackage{calc}                                             
\usepackage{multirow}                                         
\usepackage{hhline}                                           
\usepackage{ifthen}                                           
\usepackage{lscape}
\begin{document}

\bibliographystyle{IEEEtran}

\title{12.661}
\author{EE25BTECH11023 - Venkata Sai}
% \maketitle
% \newpage
% \bigskip
\maketitle 
\renewcommand{\thefigure}{\theenumi}
\renewcommand{\thetable}{\theenumi}
\setlength{\intextsep}{10pt} % Space between text and floats

\numberwithin{align}{enumi}
\numberwithin{figure}{enumi}
\renewcommand{\thetable}{\theenumi}
\vspace{-1em}
\textbf{Question:}  \\
Let $\vec{A}=\myvec{1&2\\2&1}, \vec{X}=\myvec{1&a\\b&0}$ and $\vec{Y}=\myvec{3&1\\3&2}$.If $\vec{AX=Y}$.Then $a+b$ equals\\
\textbf{Solution:}  \\
Given
  \begin{align}
\vec{A}=\myvec{1&2\\2&1}, \vec{X}=\myvec{1&a\\b&0}\ \text{and}\ \vec{Y}=\myvec{3&1\\3&2} 
\end{align}
\begin{align}
\vec{A}\vec{X}=\vec{Y} \\
\vec{X}=\vec{A}^{-1}\vec{Y}
  \end{align}
  Augmented matrix of $\augvec{1}{1}{\vec{A} & \vec{I}}$ is given by
  \begin{align}
      \augvec{2}{2}{1& 2 & 1 & 0 \\ 2 & 1 & 0 & 1} \xrightarrow{R_2\rightarrow R_2-2R_1} \augvec{2}{2}{1& 2 & 1 & 0 \\ 0 & -3 & -2 & 1} \xrightarrow{R_2 \rightarrow \frac{2}{3}R_2} \augvec{2}{2}{1& 2 & 1 & 0 \\ 0 & \frac{2}{3}\brak{-3} & \frac{2}{3}\brak{-2} & \frac{2}{3}\brak{1}}
      \end{align}
      \begin{align}
      \augvec{2}{2}{1& 2 & 1 & 0 \\ 0 &-2 & -\frac{4}{3} & \frac{2}{3}} \xrightarrow{R_1 \rightarrow R_1+R_2} \augvec{2}{2}{1& 0 & 1-\frac{4}{3} & \frac{2}{3} \\ 0 &-2 & -\frac{4}{3} & \frac{2}{3}} \xrightarrow{R_2\rightarrow -\frac{1}{2}R_2} \augvec{2}{2}{1& 0 & -\frac{1}{3} & \frac{2}{3} \\ 0 &1 & \frac{2}{3} & -\frac{1}{3}}
  \end{align}
  \begin{align}
      \vec{A}^{-1}=\myvec{-\frac{1}{3} & \frac{2}{3} \\\frac{2}{3} & -\frac{1}{3}}
  \end{align}
  \begin{align}
      \vec{X}=\myvec{-\frac{1}{3} & \frac{2}{3} \\\frac{2}{3} & -\frac{1}{3}}\myvec{3&1\\3&2} =\myvec{-\frac{1}{3}\brak{3}+\frac{2}{3}\brak{3}&-\frac{1}{3}\brak{1}+\frac{2}{3}\brak{2} \\
      \frac{2}{3}\brak{3}-\frac{1}{3}\brak{3}&\frac{2}{3}\brak{1}-\frac{1}{3}\brak{2}  } 
      \end{align}
      \begin{align}
      \myvec{1&a\\b&0}&=\myvec{1& 1 \\1&0} 
  \end{align}
  Hence $a=1,b=1\implies a+b=1+1=2$
 \end{document}
