\documentclass{beamer}
\usepackage[utf8]{inputenc}

\usetheme{Boadilla}
\usecolortheme{lily}
\usepackage{amsmath,amssymb,amsfonts,amsthm}
\usepackage{mathtools}
\usepackage{txfonts}
\usepackage{tkz-euclide}
\usepackage{listings}
\usepackage{adjustbox}
\usepackage{array}
\usepackage{tabularx}
\usepackage{gvv}
\usepackage{lmodern}
\usepackage{circuitikz}
\usepackage{tikz}
\usepackage{graphicx}

\setbeamertemplate{footline}
{
  \leavevmode%
  \hbox{%
  \begin{beamercolorbox}[wd=\paperwidth,ht=2.25ex,dp=1ex,right]{author in head/foot}%
    \insertframenumber{} / \inserttotalframenumber\hspace*{2ex} 
  \end{beamercolorbox}}%
  \vskip0pt%
}

\usepackage{tcolorbox}
\tcbuselibrary{minted,breakable,xparse,skins}




\providecommand{\nCr}[2]{\,^{#1}C_{#2}} % nCr
\providecommand{\nPr}[2]{\,^{#1}P_{#2}} % nPr
\providecommand{\mbf}{\mathbf}
\providecommand{\pr}[1]{\ensuremath{\Pr\left(#1\right)}}
\providecommand{\qfunc}[1]{\ensuremath{Q\left(#1\right)}}
\providecommand{\sbrak}[1]{\ensuremath{{}\left[#1\right]}}
\providecommand{\lsbrak}[1]{\ensuremath{{}\left[#1\right.}}
\providecommand{\rsbrak}[1]{\ensuremath{{}\left.#1\right]}}
\providecommand{\brak}[1]{\ensuremath{\left(#1\right)}}
\providecommand{\lbrak}[1]{\ensuremath{\left(#1\right.}}
\providecommand{\rbrak}[1]{\ensuremath{\left.#1\right)}}
\providecommand{\cbrak}[1]{\ensuremath{\left\{#1\right\}}}
\providecommand{\lcbrak}[1]{\ensuremath{\left\{#1\right.}}
\providecommand{\rcbrak}[1]{\ensuremath{\left.#1\right\}}}
\theoremstyle{remark}
\newcommand{\sgn}{\mathop{\mathrm{sgn}}}
\providecommand{\abs}[1]{\left\vert#1\right\vert}
\providecommand{\res}[1]{\Res\displaylimits_{#1}} 
\providecommand{\norm}[1]{\lVert#1\rVert}
\providecommand{\mtx}[1]{\mathbf{#1}}
\providecommand{\mean}[1]{E\left[ #1 \right]}
\providecommand{\fourier}{\overset{\mathcal{F}}{ \rightleftharpoons}}
%\providecommand{\hilbert}{\overset{\mathcal{H}}{ \rightleftharpoons}}
\providecommand{\system}{\overset{\mathcal{H}}{ \longleftrightarrow}}
	%\newcommand{\solution}[2]{\textbf{Solution:}{#1}}
%\newcommand{\solution}{\noindent \textbf{Solution: }}
\providecommand{\dec}[2]{\ensuremath{\overset{#1}{\underset{#2}{\gtrless}}}}
\newcommand{\myvec}[1]{\ensuremath{\begin{pmatrix}#1\end{pmatrix}}}
\let\vec\mathbf

\lstset{
%language=C,
frame=single, 
breaklines=true,
columns=fullflexible
}

\numberwithin{equation}{section}

\lstset{
  language=Python,
  basicstyle=\ttfamily\small,
  keywordstyle=\color{blue},
  stringstyle=\color{orange},
  numbers=left,
  numberstyle=\tiny\color{gray},
  breaklines=true,
  showstringspaces=false
}

\title{Problem 5.8.21}
\author{ee25btech11023-Venkata Sai}

\date{\today} 
\begin{document}

\begin{frame}
\titlepage
\end{frame}

\section*{Outline}
\begin{frame}
\tableofcontents
\end{frame}

\section{Problem}

\begin{frame}
\frametitle{Problem}
\setcounter{section}{1}
The Sum of the digits of a two-digit number is 9. Also, nine times this number is twice the number obtained by reversing the order of the digits. Find the number
\end{frame}
%\subsection{Literature}
\section{Solution}

\subsection{Equations}
\begin{frame}
\frametitle{Equations}
 Let $\vec{x}$ be the matrix that contains the digits of the required number $N$
 \begin{align}
   N= \myvec{10&1}\vec{x} 
 \end{align}
Given Sum of the digits of a two-digit number is 9
\begin{align}
    \myvec{1&1}\vec{x}=9
\end{align}
Nine times this number is twice the number obtained by reversing the order of the digits.
\begin{align}
9\myvec{10&1}\vec{x}=2\myvec{1&10}\vec{x} \\ 
\myvec{90-2&9-20}\vec{x}=0 \\
\myvec{88&-11}\vec{x}=0 
\end{align}
\begin{align}
11\myvec{8&-1}\vec{x}=0 \implies \myvec{8&-1}\vec{x}=0
\end{align}
\end{frame}
\begin{frame}
\frametitle{Augmented matrix}
\begin{align}
\myvec{1&1 \\ 8&-1}\vec{x}=\myvec{9\\0}
\end{align}
Augmented Matrix:
\begin{align}
\augvec{2}{1}{1&1&9\\8&-1&0} \xrightarrow{R_2\rightarrow R_2-8R_1} \augvec{2}{1}{1&1&9\\0&-9&-72} \xrightarrow{R_1\rightarrow R_1+\frac{1}{9}R_2}\augvec{2}{1}{1&0&1\\0&-1&-8} 
\end{align}
\begin{align}
\augvec{2}{1}{1&0&1\\0&-1&-8} \xrightarrow{R_2\rightarrow-R_2}\augvec{2}{1}{1&0&1\\0&1&8}
\end{align}
\begin{align}
\vec{x}=\myvec{1\\8}
\end{align}
\begin{align}
N=\myvec{10&1}\myvec{1\\8}=10+8=18
\end{align}
Hence the Required Number is 18
\end{frame}
\subsection{Conclusion}
\begin{frame}
\frametitle{Conclusion}
 Hence the Required Number is 18
\end{frame}
\subsection{Plot}
\begin{frame}[fragile]
\frametitle{Plot}
 
\end{frame}

\section{C Code}
\begin{frame}[fragile]
\frametitle{C Code}
\begin{lstlisting}[language=C]
void get_system_coeffs(double* out_data) {

    out_data[0] = 1.0;
    out_data[1] = 1.0;
    out_data[2] = 8.0;
    out_data[3] = -1.0;
    out_data[4] = 9.0;
    out_data[5] = 0.0;
}
    \end{lstlisting}
\end{frame}
\section{Python Code}
\begin{frame}[fragile]
\frametitle{Python Code for Calling}
\begin{lstlisting}[language=Python]
import ctypes
import sympy

def solve_number_problem():

    lib = ctypes.CDLL('./code.so')
    double_array_6 = ctypes.c_double * 6
    # The C function will fill an array of 6 doubles
    lib.get_system_coeffs.argtypes = [ctypes.POINTER(ctypes.c_double)]
    
    out_data_c = double_array_6()
    lib.get_system_coeffs(out_data_c)
    
    # Unpack the raw coefficient data from C
    m_coeffs = list(int(v) for v in out_data_c)[:4]
    c_coeffs = list(int(v) for v in out_data_c)[4:]
   
\end{lstlisting}
\end{frame}
\begin{frame}[fragile]
\frametitle{Python Code for Solving}
\begin{lstlisting}[language=Python]
 aug_M = sympy.Matrix([
        [m_coeffs[0], m_coeffs[1], c_coeffs[0]],
        [m_coeffs[2], m_coeffs[3], c_coeffs[1]]
    ])
    print("Initial Augmented Matrix:\n", aug_M)
    rref_matrix, _ = aug_M.rref()
    print("\nReduced Row Echelon Form:\n", rref_matrix)
    
    x_digit = rref_matrix[0, 2]
    y_digit = rref_matrix[1, 2]
    
    return int(x_digit), int(y_digit)
x, y = solve_number_problem()
    
    # Calculate the final two-digit number
number = 10 * x + y

print(f"The number is: {number}")
\end{lstlisting}
\end{frame}
 

\end{document}
