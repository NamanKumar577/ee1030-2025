\let\negmedspace\undefined
\let\negthickspace\undefined
\documentclass[journal]{IEEEtran}
\usepackage[a5paper, margin=10mm, onecolumn]{geometry}
%\usepackage{lmodern} % Ensure lmodern is loaded for pdflatex
\usepackage{tfrupee} % Include tfrupee package

\setlength{\headheight}{1cm} % Set the height of the header box
\setlength{\headsep}{0mm}     % Set the distance between the header box and the top of the text

\usepackage{gvv-book}
\usepackage{gvv}
\usepackage{cite}
\usepackage{amsmath,amssymb,amsfonts,amsthm}
\usepackage{algorithmic}
\usepackage{graphicx}
\usepackage{textcomp}
\usepackage{xcolor}
\usepackage{txfonts}
\usepackage{listings}
\usepackage{enumitem}
\usepackage{mathtools}
\usepackage{gensymb}
\usepackage{comment}
\usepackage[breaklinks=true]{hyperref}
\usepackage{tkz-euclide} 
\usepackage{listings}
% \usepackage{gvv}                                        
\def\inputGnumericTable{}                                 
\usepackage[latin1]{inputenc}                                
\usepackage{color}                                            
\usepackage{array}                                            
\usepackage{longtable}                                       
\usepackage{calc}                                             
\usepackage{multirow}                                         
\usepackage{hhline}                                           
\usepackage{ifthen}                                           
\usepackage{lscape}
\begin{document}

\bibliographystyle{IEEEtran}

\title{5.13.49}
\author{EE25BTECH11023 - Venkata Sai}
% \maketitle
% \newpage
% \bigskip
\maketitle \vspace{-1cm}
\renewcommand{\thefigure}{\theenumi}
\renewcommand{\thetable}{\theenumi}
\setlength{\intextsep}{10pt} % Space between text and floats

\numberwithin{align}{enumi}
\numberwithin{figure}{enumi}
\renewcommand{\thetable}{\theenumi}

\textbf{Question:}  \\
If the system of equations
\begin{align}
    x-ky-z=0\\
    kx-y-z=0\\
    x+y+z=0
\end{align}
has a non-zero solution, then the possible values of $k$ are 

\textbf{Solution:}  \\
For the given homogeneous system
\begin{align}
\vec{A}\vec{x}=0
\end{align}
Augmented matrix of $\augvec{1}{1}{\vec{A} & 0}$ is given by \\
\begin{align}
\augvec{3}{1}{1 & -k & -1  & 0\\ k & -1 & -1  & 0\\1 & 1 & -1  & 0}\xrightarrow[R_3\rightarrow R_3-R_1]{R_2\rightarrow R_2-kR_1}\augvec{3}{1}{1 & -k & -1&0\\ 0 & k^2-1 & k-1&0\\ 0 & 1+k & 0 & 0}\xrightarrow{R_2-\brak{k-1}R_3}\augvec{3}{1}{1&-k&-1&0\\0&k+1&0&0\\0&0&k-1&0}
\end{align}
For a non-zero solution,The rank of the matrix must be less than the number of variables \\
From \brak{5}, In order to be Rank$<$3 
\begin{align}
    k+1=0\ \brak{\text{or}}\ k-1=0 \\
    k=-1\ \brak{\text{or}}\ k=1
\end{align}
\end{document}  
