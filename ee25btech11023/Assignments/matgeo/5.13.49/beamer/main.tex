\documentclass{beamer}
\usepackage[utf8]{inputenc}

\usetheme{Boadilla}
\usecolortheme{lily}
\usepackage{amsmath,amssymb,amsfonts,amsthm}
\usepackage{mathtools}
\usepackage{txfonts}
\usepackage{tkz-euclide}
\usepackage{listings}
\usepackage{adjustbox}
\usepackage{array}
\usepackage{tabularx}
\usepackage{lmodern}
\usepackage{gvv}
\usepackage{circuitikz}
\usepackage{tikz}
\usepackage{graphicx}

\setbeamertemplate{footline}
{
  \leavevmode%
  \hbox{%
  \begin{beamercolorbox}[wd=\paperwidth,ht=2.25ex,dp=1ex,right]{author in head/foot}%
    \insertframenumber{} / \inserttotalframenumber\hspace*{2ex} 
  \end{beamercolorbox}}%
  \vskip0pt%
}

\usepackage{tcolorbox}
\tcbuselibrary{minted,breakable,xparse,skins}




\providecommand{\nCr}[2]{\,^{#1}C_{#2}} % nCr
\providecommand{\nPr}[2]{\,^{#1}P_{#2}} % nPr
\providecommand{\mbf}{\mathbf}
\providecommand{\pr}[1]{\ensuremath{\Pr\left(#1\right)}}
\providecommand{\qfunc}[1]{\ensuremath{Q\left(#1\right)}}
\providecommand{\sbrak}[1]{\ensuremath{{}\left[#1\right]}}
\providecommand{\lsbrak}[1]{\ensuremath{{}\left[#1\right.}}
\providecommand{\rsbrak}[1]{\ensuremath{{}\left.#1\right]}}
\providecommand{\brak}[1]{\ensuremath{\left(#1\right)}}
\providecommand{\lbrak}[1]{\ensuremath{\left(#1\right.}}
\providecommand{\rbrak}[1]{\ensuremath{\left.#1\right)}}
\providecommand{\cbrak}[1]{\ensuremath{\left\{#1\right\}}}
\providecommand{\lcbrak}[1]{\ensuremath{\left\{#1\right.}}
\providecommand{\rcbrak}[1]{\ensuremath{\left.#1\right\}}}
\theoremstyle{remark}
\newcommand{\sgn}{\mathop{\mathrm{sgn}}}
\providecommand{\abs}[1]{\left\vert#1\right\vert}
\providecommand{\res}[1]{\Res\displaylimits_{#1}} 
\providecommand{\norm}[1]{\lVert#1\rVert}
\providecommand{\mtx}[1]{\mathbf{#1}}
\providecommand{\mean}[1]{E\left[ #1 \right]}
\providecommand{\fourier}{\overset{\mathcal{F}}{ \rightleftharpoons}}
%\providecommand{\hilbert}{\overset{\mathcal{H}}{ \rightleftharpoons}}
\providecommand{\system}{\overset{\mathcal{H}}{ \longleftrightarrow}}
	%\newcommand{\solution}[2]{\textbf{Solution:}{#1}}
%\newcommand{\solution}{\noindent \textbf{Solution: }}
\providecommand{\dec}[2]{\ensuremath{\overset{#1}{\underset{#2}{\gtrless}}}}
\newcommand{\myvec}[1]{\ensuremath{\begin{pmatrix}#1\end{pmatrix}}}
\let\vec\mathbf

\lstset{
%language=C,
frame=single, 
breaklines=true,
columns=fullflexible
}

\numberwithin{equation}{section}

\lstset{
  language=Python,
  basicstyle=\ttfamily\small,
  keywordstyle=\color{blue},
  stringstyle=\color{orange},
  numbers=left,
  numberstyle=\tiny\color{gray},
  breaklines=true,
  showstringspaces=false
}

\title{Problem 5.13.49}
\author{ee25btech11023-Venkata Sai}

\date{\today} 
\begin{document}

\begin{frame}
\titlepage
\end{frame}

\section*{Outline}
\begin{frame}
\tableofcontents
\end{frame}

\section{Problem}

\begin{frame}
\frametitle{Problem}
\setcounter{section}{1}
If the system of equations
\begin{align}
    x-ky-z=0\\
    kx-y-z=0\\
    x+y+z=0
\end{align}
has a non-zero solution, then the possible values of $k$ are 
\end{frame}
%\subsection{Literature}
\section{Solution}

\subsection{Matrix Equation}
\begin{frame}
\frametitle{Matrix Equation}
 For the given homogeneous system
\begin{align}
\vec{A}\vec{x}=0
\end{align}
Augmented matrix of $\augvec{1}{1}{\vec{A} & 0}$ is given by \\
\begin{align}
\augvec{3}{1}{1 & -k & -1  & 0\\ k & -1 & -1  & 0\\1 & 1 & -1  & 0}\xrightarrow[R_3\rightarrow R_3-R_1]{R_2\rightarrow R_2-kR_1}\augvec{3}{1}{1 & -k & -1&0\\ 0 & k^2-1 & k-1&0\\ 0 & 1+k & 0 & 0} 
\end{align}
\begin{align}
 \augvec{3}{1}{1 & -k & -1&0\\ 0 & k^2-1 & k-1&0\\ 0 & 1+k & 0 & 0}\xrightarrow{R_2-\brak{k-1}R_3}\augvec{3}{1}{1&-k&-1&0\\0&k+1&0&0\\0&0&k-1&0}   
\end{align}
\end{frame}
\subsection{Conclusion}
\begin{frame}
\frametitle{Conclusion}
For a non-zero solution,The rank of the matrix must be less than the number of variables \\
From \brak{5}, In order to be Rank$<$3 
\begin{align}
    k+1=0\ \brak{\text{or}}\ k-1=0 \\
    k=-1\ \brak{\text{or}}\ k=1
\end{align}
\end{frame}
 
 
\end{frame}

\section{C Code}
\begin{frame}[fragile]
\frametitle{C Code}
\begin{lstlisting}[language=C]
void get_system_coeffs(double* out_coeffs) {
     
    out_coeffs[0] = 1.0;
    out_coeffs[1] = -1.0;
   
    out_coeffs[2] = -1.0;
    out_coeffs[3] = -1.0;
    

    out_coeffs[4] = 1.0;
    out_coeffs[5] = 1.0;
    out_coeffs[6] = 1.0;
}
    \end{lstlisting}
\end{frame}

\section{Python Code}
\begin{frame}[fragile]
\frametitle{Python Code for Solving}
\begin{lstlisting}[language=Python]
import ctypes
import sympy
 
lib = ctypes.CDLL('./code.so')

double_array_7 = ctypes.c_double * 7
lib.get_system_coeffs.argtypes = [ctypes.POINTER(ctypes.c_double)]
    
out_data_c = double_array_7()
lib.get_system_coeffs(out_data_c)
    
 
coeffs = list(int(v) for v in out_data_c)
 
k = sympy.Symbol('k')
    

\end{lstlisting}
\end{frame}
\begin{frame}[fragile]
\frametitle{Python Code for Solving}
\begin{lstlisting}[language=Python]
 
M = sympy.Matrix([
        [coeffs[0], -k,         coeffs[1]],  
        [k,         coeffs[2],  coeffs[3]],  
        [coeffs[4], coeffs[5],  coeffs[6]]  
    ])
det_M = M.det()
print(f"\nDeterminant = {det_M}")
    
     
solutions = sympy.solve(det_M, k)

print(f"k can be {solutions}")
\end{lstlisting}
\end{frame}
\end{document}
