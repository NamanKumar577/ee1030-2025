\documentclass{beamer}
\mode<presentation>
\usepackage{amsmath,amssymb,mathtools}
\usepackage{textcomp}
\usepackage{gensymb}
\usepackage{adjustbox}
\usepackage{subcaption}
\usepackage{enumitem}
\usepackage{multicol}
\usepackage{listings}
\usepackage{url}
\usepackage{graphicx} % <-- needed for images
\def\UrlBreaks{\do\/\do-}

\usetheme{Boadilla}
\usecolortheme{lily}
\setbeamertemplate{footline}{
  \leavevmode%
  \hbox{%
  \begin{beamercolorbox}[wd=\paperwidth,ht=2ex,dp=1ex,right]{author in head/foot}%
    \insertframenumber{} / \inserttotalframenumber\hspace*{2ex}
  \end{beamercolorbox}}%
  \vskip0pt%
}
\setbeamertemplate{navigation symbols}{}

\lstset{
  frame=single,
  breaklines=true,
  columns=fullflexible,
  basicstyle=\ttfamily\tiny   % tiny font so code fits
}

\numberwithin{equation}{section}

% ---- your macros ----
\providecommand{\nCr}[2]{\,^{#1}C_{#2}}
\providecommand{\nPr}[2]{\,^{#1}P_{#2}}
\providecommand{\mbf}{\mathbf}
\providecommand{\pr}[1]{\ensuremath{\Pr\left(#1\right)}}
\providecommand{\qfunc}[1]{\ensuremath{Q\left(#1\right)}}
\providecommand{\sbrak}[1]{\ensuremath{{}\left[#1\right]}}
\providecommand{\lsbrak}[1]{\ensuremath{{}\left[#1\right.}}
\providecommand{\rsbrak}[1]{\ensuremath{\left.#1\right]}}
\providecommand{\brak}[1]{\ensuremath{\left(#1\right)}}
\providecommand{\lbrak}[1]{\ensuremath{\left(#1\right.}}
\providecommand{\rbrak}[1]{\ensuremath{\left.#1\right)}}
\providecommand{\cbrak}[1]{\ensuremath{\left\{#1\right\}}}
\providecommand{\lcbrak}[1]{\ensuremath{\left\{#1\right.}}
\providecommand{\rcbrak}[1]{\ensuremath{\left.#1\right\}}}
\theoremstyle{remark}
\newtheorem{rem}{Remark}
\newcommand{\sgn}{\mathop{\mathrm{sgn}}}
\providecommand{\abs}[1]{\left\vert#1\right\vert}
\providecommand{\res}[1]{\Res\displaylimits_{#1}}
\providecommand{\norm}[1]{\lVert#1\rVert}
\providecommand{\mtx}[1]{\mathbf{#1}}
\providecommand{\mean}[1]{E\left[ #1 \right]}
\providecommand{\fourier}{\overset{\mathcal{F}}{ \rightleftharpoons}}
\providecommand{\system}{\overset{\mathcal{H}}{ \longleftrightarrow}}
\providecommand{\dec}[2]{\ensuremath{\overset{#1}{\underset{#2}{\gtrless}}}}
\newcommand{\myvec}[1]{\ensuremath{\begin{pmatrix}#1\end{pmatrix}}}
\let\vec\mathbf

\title{Matgeo Presentation - 12.238}
\author{ee25btech11063 - Vejith}

\begin{document}


\frame{\titlepage}
\begin{frame}{Question}
The inverse of the matrix
$\begin{pmatrix}
    1 & 2\\
    3 & 4
\end{pmatrix}$ is \hspace{4cm} \brak{\text{CH }2010}
\end{frame}

\begin{frame}{Solution}
    Let 
\begin{align}
    \Vec{A}=\begin{pmatrix}
    1 & 2\\
    3 & 4
\end{pmatrix}
\end{align}
The augmented matrix is 
\begin{align}
    \left(\begin{array}{c|c}
        \Vec{A} & \Vec{I}
\end{array}\right)
\implies 
\left(\begin{array}{cc|cc}
        1 & 2 & 1 & 0\\
        3 & 4 &  0 & 1
\end{array}\right)  &\xleftrightarrow{R_2 \rightarrow R_2-3R_1} \left(\begin{array}{cc|cc}
        1 & 2 & 1 & 0\\
        0 & -2 &  -3 & 1
\end{array}\right)\\
&\xleftrightarrow{R_1 \rightarrow R_1 + R_2} \left(\begin{array}{cc|cc}
        1 & 0 & -2 & 1\\
        0 & -2 &  -3 & 1
\end{array}\right)\\
&\xleftrightarrow{R_2 \rightarrow \frac{-1}{2} \times R_2} \left(\begin{array}{cc|cc}
        1 & 0 & -2 & 1\\
        0 & 1 &  \frac{3}{2} & \frac{-1}{2}
\end{array}\right)
\end{align}
As the left block of the Augmented matrix is $\Vec{I}$ the right block is $\Vec{A^{-1}}$.
\begin{align}
    \Vec{A^{-1}}=\begin{pmatrix}
    -2 & 1\\
    \frac{3}{2} & \frac{-1}{2}
\end{pmatrix}
\end{align}
\end{frame}

% --------- CODE APPENDIX ---------
\section*{Appendix: Code}

% C program
\begin{frame}[fragile]{C Code: inverse.c}
\begin{lstlisting}[language=C]
#include <stdio.h>

int main() {
    FILE *fp;
    float a = 1, b = 2, c = 3, d = 4;
    float det, inv[2][2];

    det = a * d - b * c;

    if(det == 0) {
        printf("Inverse does not exist (determinant is zero).\n");
        return 0;
    }
    inv[0][0] =  d / det;
    inv[0][1] = -b / det;
    inv[1][0] = -c / det;
    inv[1][1] =  a / det;

    // Open file to write
    fp = fopen("inverse.dat", "w");
    if(fp == NULL) {
        printf("Error opening file!\n");
        return 1;
    }

    fprintf(fp, "The inverse of the matrix is:\n");
    fprintf(fp, "[ %.2f  %.2f ]\n", inv[0][0], inv[0][1]);
    fprintf(fp, "[ %.2f  %.2f ]\n", inv[1][0], inv[1][1]);

    fclose(fp);
    printf("Inverse matrix has been written to inverse.dat successfully.\n");
     return 0;}


                \end{lstlisting}
\end{frame}

% Python plotting
\begin{frame}[fragile]{Python: Solution.py}
\begin{lstlisting}[language=Python]
import numpy as np

# Define the matrix
A = np.array([[1, 2],
              [3, 4]])

# Compute the determinant
det = np.linalg.det(A)
print("Determinant of A:", det)

# Check if inverse exists
if det == 0:
    print("Inverse does not exist (determinant is zero).")
else:
    # Compute the inverse
    A_inv = np.linalg.inv(A)
    print("Inverse of the matrix A is:\n", A_inv)

\end{lstlisting}
\end{frame}
\end{document}
