\documentclass[12pt]{article}
\usepackage{graphicx} % Required for inserting images
\usepackage{mathtools}
\usepackage{amsmath}
\usepackage{gvv-book}
\usepackage{gvv}
\usepackage[shortlabels]{enumitem}

\title{\textbf{5.13.30}}
\author{\textbf{EE25BTECH11004 - Aditya Appana}}
\date{September 30, 2025}
\renewcommand{\labelenumi}{\Alph{enumi})}
\begin{document}

\maketitle

\section*{Question}
Let $\vec{A}$ be a square matrix all of whose entries are integers. Then which of the
following is true?
\begin{enumerate}
    \item If $det(\vec{A})\neq $ ±$1$, then $\vec{A}^{-1}$ exists but all its entries are not necessarily integers
    \item If $det(\vec{A})\neq $ ±$1$, then $\vec{A}^{-1}$ exists and all its entries are non-integers
    \item If $det(\vec{A}) = $ ±$1$, then $\vec{A}^{-1}$ exists but all its entries are integers
    \item If $det(\vec{A})= $ ±$1$, then $\vec{A}^{-1}$ need not exist

\end{enumerate}

\section*{Solution}
We will proceed by checking each option.\\

\underline{A) If $det(\vec{A})\neq $ ±$1$, then $\vec{A}^{-1}$ exists but all its entries are not necessarily integers} \\

Let us take a square matrix $\vec{A}$ having all integer entries. Let rows $R_1$ and $R_2$ be equal.
By performing row operation $R_1 \xrightarrow{} R_1 - R_2$, all elements in $R_1 $ become 0. Therefore, $|\vec{A}| = 0$. We know that if $|\vec{A}|= 0$, $\vec{A^{-1}}$ does not exist. Therefore, this option is wrong.\\

For example, consider a matrix $\vec{A}$ = \myvec{3&3&3\\3&3&3\\ 1&4&2}. $\vec{A}$ has only integer entries.
\begin{align*}
\myvec{3&3&3\\3&3&3\\ 1&4&2} \xrightarrow{\text{R_1 \rightarrow $R_1 - R_2$}} \myvec{0&0&0\\3&3&3\\ 1&4&2}
\end{align*}\\

Since $R_1$ consists of only 0's, $|\vec{A}|$ = 0. Hence $\vec{A}$ is not invertible.
\vspace{1.5cm}

\underline{B) If $det(\vec{A})\neq $ ±$1$, then $\vec{A}^{-1}$ exists and all its entries are non-integers}\\

For example, consider a matrix $\vec{A} = \myvec{2 & 1 \\ 2 & 2}$. $|\vec{A}| = 2$
\begin{align*}
    \vec{A^{-1}} = \frac{1}{2}\myvec{2 & -1 \\ -2 & 2} = \myvec{1 & -1/2 \\ 1 & 1}.
\end{align*}\\

This is a counterexample to the statement; hence, option \textbf{B} is wrong.

\vspace{1cm}
\underline{D) If $det(\vec{A})= $ ±$1$, then $\vec{A}^{-1}$ need not exist}\\

We know that if $|\vec{A}| \neq 0$, $\vec{A^{-1}}$ exists. By this logic, this option is wrong.\\

For example, consider a matrix $\vec{A}$ = $\myvec{\frac{1}{\sqrt{2}} & \frac{1}{\sqrt{2}} \\ \frac{-1}{\sqrt{2}} & \frac{1}{\sqrt{2}}}$. $|\vec{A}| = 1$, and since this is an orthogonal matrix, 
\begin{align*}
\vec{A^{-1}} = \vec{A^T} = \myvec{\frac{1}{\sqrt{2}} & \frac{-1}{\sqrt{2}} \\ \frac{1}{\sqrt{2}} & \frac{1}{\sqrt{2}}}. \end{align*}\\

$\vec{A^{-1}}$ exists, which is a contradiction to the statement in option \textbf{D}.


Therefore, the correct answer is $\texbtf{C)}$.

\end{document}