\documentclass{beamer}
\usepackage[utf8]{inputenc}
\usetheme{Madrid}
\usecolortheme{default}
\usepackage{amsmath,amssymb,amsfonts,amsthm}
\usepackage{txfonts}
\usepackage{tkz-euclide}
\usepackage{listings}
\usepackage{adjustbox}
\usepackage{array}
\usepackage{tabularx}
\usepackage{gvv}
\usepackage{lmodern}
\usepackage{circuitikz}
\usepackage{tikz}
\usepackage{graphicx}

\setbeamertemplate{page number in head/foot}[totalframenumber]

\usepackage{tcolorbox}
\tcbuselibrary{minted,breakable,xparse,skins}



\definecolor{bg}{gray}{0.95}
\DeclareTCBListing{mintedbox}{O{}m!O{}}{%
  breakable=true,
  listing engine=minted,
  listing only,
  minted language=#2,
  minted style=default,
  minted options={%
    linenos,
    gobble=0,
    breaklines=true,
    breakafter=,,
    fontsize=\small,
    numbersep=8pt,
    #1},
  boxsep=0pt,
  left skip=0pt,
  right skip=0pt,
  left=25pt,
  right=0pt,
  top=3pt,
  bottom=3pt,
  arc=5pt,
  leftrule=0pt,
  rightrule=0pt,
  bottomrule=2pt,
  toprule=2pt,
  colback=bg,
  colframe=orange!70,
  enhanced,
  overlay={%
    \begin{tcbclipinterior}
    \fill[orange!20!white] (frame.south west) rectangle ([xshift=20pt]frame.north west);
    \end{tcbclipinterior}},
  #3,
}
\lstset{
    language=C,
    basicstyle=\ttfamily\small,
    keywordstyle=\color{blue},
    stringstyle=\color{orange},
    commentstyle=\color{green!60!black},
    numbers=left,
    numberstyle=\tiny\color{gray},
    breaklines=true,
    showstringspaces=false,
}
%------------------------------------------------------------
%This block of code defines the information to appear in the
%Title page
\title %optional
{4.4.13}

%\subtitle{A short story}

\author % (optional)
{BALU-ai25btech11017}



\begin{document}


\frame{\titlepage}
\begin{frame}{Question}
 If $\vec{A}, \vec{B}, \vec{C}$ are three non-coplanar vectors, then
\begin{align}
\frac{\vec{A} \cdot(\vec{B}\times\vec{C})}{(\vec{C} \times \vec{A}) \cdot \vec{B}}
+ \frac{\vec{B} \cdot(\vec{A}\times \vec{C})}{\vec{C} \cdot (\vec{A} \times \vec{B)}}
= 
\end{align}\\ 
\end{frame}
\begin{frame}{Theoretical Solution}
Let us solve the given equation theoretically and then verify the solution computationally \\
According to the question, \\
Let us take three non coplanar vectors \\
\begin{align}
\vec{A}=\begin{myvec}{1\\0\\0}\end{myvec}\
\vec{B}=\begin{myvec}{0\\1\\0}\end{myvec}\
\vec{C}=\begin{myvec}{0\\0\\1}\end{myvec}\
\end{align}
\begin{align}
\vec{A}^T(\vec{B}\times\vec{C})=[\vec{A}\quad\vec{B}\quad\vec{C}]=
\left\|\begin{myvec}{1&&0&&0\\0&&1&&0\\0&&0&&1}\end{myvec}\right\|=1
\end{align}
\begin{align}
(\vec{C} \times \vec{A})^T\vec{B}=[\vec{C}\quad\vec{A}\quad\vec{B}]=\left\|\begin{myvec}{0&&1&&0\\0&&0&&1\\1&&0&&0}\end{myvec}\right\|=1\
\end{align}

\end{frame}
\begin{frame}{Theoretical Solution}
\begin{align}
\vec{B}^T(\vec{A}\times \vec{C})=[\vec{B}\quad\vec{A}\quad\vec{C}]=\left\|\begin{myvec}{0&&1&&0\\1&&0&&0\\0&&0&&1}\end{myvec}\right\|=-1\
\end{align}
\begin{align}
\vec{C}^T (\vec{A} \times \vec{B})=[\vec{C}\quad\vec{A}\quad\vec{B}]=\left\|\begin{myvec}{0&&1&&0\\0&&0&&1\\1&&0&&0}\end{myvec}\right\|=1\ 
\end{align}
\begin{align}
\frac{\vec{A}^T(\vec{B}\times\vec{C})}{(\vec{C} \times \vec{A})^T \vec{B}}
+ \frac{\vec{B}^T(\vec{A}\times \vec{C})}{\vec{C}^T (\vec{A} \times \vec{B)}}
= \frac{1}{1}+\frac{-1}{1}=1-1=0\
\end{align}
By verification method we showed the result is 0
\end{frame}
\begin{frame}[fragile]
    \frametitle{C Code}
    \begin{lstlisting}
#include <stdio.h>

// Function to compute cross product of two vectors
void crossProduct(double u[], double v[], double result[]) {
    result[0] = u[1]*v[2] - u[2]*v[1];
    result[1] = u[2]*v[0] - u[0]*v[2];
    result[2] = u[0]*v[1] - u[1]*v[0];
}

// Function to compute dot product of two vectors
double dotProduct(double u[], double v[]) {
    return u[0]*v[0] + u[1]*v[1] + u[2]*v[2];
}

int main() {
    // Define vectors A = i, B = j, C = k
    double A[3] = {1, 0, 0};
    double B[3] = {0, 1, 0};
    double C[3] = {0, 0, 1};
     \end{lstlisting}
\end{frame}
\begin{frame}[fragile]
    \frametitle{C Code }
    \begin{lstlisting}
     double BxC[3], CxA[3], AxC[3], AxB[3];
    double numerator1, denominator1, numerator2, denominator2, result;

    // Compute cross products
    crossProduct(B, C, BxC);
    crossProduct(C, A, CxA);
    crossProduct(A, C, AxC);
    crossProduct(A, B, AxB);

    // Compute terms
    numerator1 = dotProduct(A, BxC);
    denominator1 = dotProduct(CxA, B);
    numerator2 = dotProduct(B, AxC);
    denominator2 = dotProduct(C, AxB);
\end{lstlisting}
\end{frame}
\begin{frame}[fragile]
    \frametitle{C Code }
    \begin{lstlisting}
     // Final result
    result = (numerator1 / denominator1) + (numerator2 / denominator2);

    // Print results
    printf("Numerator1 = %.2f, Denominator1 = %.2f\n", numerator1, denominator1);
    printf("Numerator2 = %.2f, Denominator2 = %.2f\n", numerator2, denominator2);
    printf("Final Result = %.2f\n", result);

    return 0;
}
    \end{lstlisting}
\end{frame}

\end{document}