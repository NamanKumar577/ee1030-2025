\documentclass[journal]{IEEEtran}
\usepackage[a5paper, margin=10mm]{geometry}
%\usepackage{lmodern} % Ensure lmodern is loaded for pdflatex
\usepackage{tfrupee} % Include tfrupee package


\setlength{\headheight}{1cm} % Set the height of the header box
\setlength{\headsep}{0mm}     % Set the distance between the header box and the top of the text


%\usepackage[a5paper, top=10mm, bottom=10mm, left=10mm, right=10mm]{geometry}

%
\setlength{\intextsep}{10pt} % Space between text and floats

\makeindex


\usepackage{cite}
\usepackage{amsmath,amssymb,amsfonts,amsthm}
\usepackage{algorithmic}
\usepackage{graphicx}
\usepackage{textcomp}
\usepackage{xcolor}
\usepackage{txfonts}
\usepackage{listings}
\usepackage{enumitem}
\usepackage{mathtools}
\usepackage{gensymb}
\usepackage{comment}
\usepackage[breaklinks=true]{hyperref}
\usepackage{tkz-euclide} 
\usepackage{listings}
\usepackage{multicol}
\usepackage{xparse}
\usepackage{gvv}
%\def\inputGnumericTable{}                                 
\usepackage[latin1]{inputenc}                                
\usepackage{color}                                            
\usepackage{array}                                            
\usepackage{longtable}                                       
\usepackage{calc}                                             
\usepackage{multirow}                                         
\usepackage{hhline}                                           
\usepackage{ifthen}                                               
\usepackage{lscape}
\usepackage{tabularx}
\usepackage{array}
\usepackage{float}
\usepackage{ar}
\usepackage[version=4]{mhchem}


\newtheorem{theorem}{Theorem}[section]
\newtheorem{problem}{Problem}
\newtheorem{proposition}{Proposition}[section]
\newtheorem{lemma}{Lemma}[section]
\newtheorem{corollary}[theorem]{Corollary}
\newtheorem{example}{Example}[section]
\newtheorem{definition}[problem]{Definition}
\newcommand{\BEQA}{\begin{eqnarray}}
\newcommand{\EEQA}{\end{eqnarray}}

\theoremstyle{remark}


\begin{document}
\bibliographystyle{IEEEtran}
\onecolumn

\title{12.145}
\author{INDHIRESH S- EE25BTECH11027}
\maketitle


\renewcommand{\thefigure}{\theenumi}
\renewcommand{\thetable}{\theenumi}

\textbf{Question}. If a square matrix $\Vec{A}$ is real and symmetric, then the eigenvalues
\begin{enumerate}
    \item are always real
    \item  are always real and positive
    \item are always real and non-negative
    \item occur in complex conjupairs
\end{enumerate}
\textbf{Solution}:\\
The correct statement is (1). This is a fundamental property of real symmetric matrices.\\
Let $\Vec{A}$ be a real and symmetric matrix, which means
\begin{align}
\Vec{A} = \Vec{A}^T\;\; and\;\; \bar{\Vec{A}} = \Vec{A}
\end{align}
Let $\lambda$ be an eigenvalue of $\Vec{A}$ with a corresponding non-zero eigenvector $\Vec{x}$. The eigenvalue equation is:
\begin{align}
    \Vec{A}\Vec{x} = \lambda\Vec{x}
\end{align}
To prove that $\lambda$ is real, we must show it is equal to its own complex conjugate, i.e., $\lambda = \bar{\lambda}$.
We take the conjugate transpose (Hermitian conjugate) on both sides:
\begin{align}
    (\Vec{A}\Vec{x})^H &= (\lambda\Vec{x})^H \\
    \Vec{x}^H \Vec{A}^H &= \bar{\lambda}\Vec{x}^H
\end{align}
For a real and symmetric matrix, its conjugate transpose is itself:
\begin{align}
    \Vec{A}^H = \bar{\Vec{A}}^T = \Vec{A}^T = \Vec{A}
\end{align}
Substituting this into the Eq.4 gives:
\begin{align}
    \Vec{x}^H \Vec{A} = \bar{\lambda}\Vec{x}^H \label{eq:conj}
\end{align}
Now, we pre-multiply the Eq.2 by $\Vec{x}^H$:
\begin{align}
    \Vec{x}^H \Vec{A}\Vec{x} = \lambda(\Vec{x}^H\Vec{x}) \label{eq:pre}
\end{align}
And we post-multiply Eq.6 by $\Vec{x}$:
\begin{align}
    \Vec{x}^H \Vec{A}\Vec{x} = \bar{\lambda}(\Vec{x}^H\Vec{x}) \label{eq:post}
\end{align}
By comparing Eq.7 and Eq.8 , we see that:
\begin{align}
 \lambda(\Vec{x}^H\Vec{x}) = \bar{\lambda}(\Vec{x}^H\Vec{x})
\end{align}
This can be rearranged to:
\begin{align}
 (\lambda - \bar{\lambda})(\Vec{x}^H\Vec{x}) = 0
\end{align}
 Since an eigenvector $\Vec{x}$ is non-zero by definition, its magnitude $||\Vec{x}||^2$ is a positive real number.So,
 \begin{align}
     \lambda - \bar{\lambda} = 0 
 \end{align}
 \begin{align}
     \lambda = \bar{\lambda}
 \end{align}
A number that is equal to its own complex conjugate must be a real number.\\
From above statement it is clear that eigenvalues are always real\\\\
Options (2) and (3) are incorrect because a real symmetric matrix can have negative eigenvalues.\\
Example:\\
Consider the matrix 
\begin{align}
    \Vec{A} = \myvec{0 & 1 \\ 1 & 0}
\end{align}
This matrix is real and symmetric.Now finding the eigen value for the matrix:
\begin{align}
   \mydet{-\lambda&1\\1&-\lambda}=\lambda^2-1= 0
\end{align}
The eigenvalues are
\begin{align}
    \lambda = 1\;\;and\;\;\lambda = -1
\end{align}
Therefore the eigenvalues can be negative\\
Option (1) is correct
\end{document}