\documentclass{beamer}
    \usepackage[utf8]{inputenc}
    
    \usetheme{Madrid}
    \usecolortheme{default}
    \usepackage{amsmath,amssymb,amsfonts,amsthm}
    \usepackage{mathtools}
    \usepackage{txfonts}
    \usepackage{tkz-euclide}
    \usepackage{listings}
    \usepackage{adjustbox}
    \usepackage{array}
    \usepackage{gensymb}
    \usepackage{tabularx}
    \usepackage{gvv}
    \usepackage{lmodern}
    \usepackage{circuitikz}
    \usepackage{tikz}
    \lstset{literate={·}{{$\cdot$}}1 {λ}{{$\lambda$}}1 {→}{{$\to$}}1}
    \usepackage{graphicx}
    
    \setbeamertemplate{page number in head/foot}[totalframenumber]
    
    \usepackage{tcolorbox}
    \tcbuselibrary{minted,breakable,xparse,skins}
    
    
    
    \definecolor{bg}{gray}{0.95}
    \DeclareTCBListing{mintedbox}{O{}m!O{}}{%
      breakable=true,
      listing engine=minted,
      listing only,
      minted language=#2,
      minted style=default,
      minted options={%
        linenos,
        gobble=0,
        breaklines=true,
        breakafter=,,
        fontsize=\small,
        numbersep=8pt,
        #1},
      boxsep=0pt,
      left skip=0pt,
      right skip=0pt,
      left=25pt,
      right=0pt,
      top=3pt,
      bottom=3pt,
      arc=5pt,
      leftrule=0pt,
      rightrule=0pt,
      bottomrule=2pt,
      toprule=2pt,
      colback=bg,
      colframe=orange!70,
      enhanced,
      overlay={%
        \begin{tcbclipinterior}
        \fill[orange!20!white] (frame.south west) rectangle ([xshift=20pt]frame.north west);
        \end{tcbclipinterior}},
      #3,
    }
    \lstset{
        language=C,
        basicstyle=\ttfamily\small,
        keywordstyle=\color{blue},
        stringstyle=\color{orange},
        commentstyle=\color{green!60!black},
        numbers=left,
        numberstyle=\tiny\color{gray},
        breaklines=true,
        showstringspaces=false,
    }
    %------------------------------------------------------------
    %This block of code defines the information to appear in the
    %Title page
    \title %optional
    {12.249}
    \date{10 October, 2025}
    %\subtitle{A short story}
    
    \author % (optional)
    {INDHIRESH S - EE25BTECH11027}
    
    \begin{document}
    
    \frame{\titlepage}
    
    \begin{frame}{Question}
  A plane contains the following three points: $\Vec{P}(2, 1, 5), \Vec{Q}(-1, 3, 4)$ and $\Vec{R}(3, 0, 6)$. The vector perpendicular to the above plane can be represented as
    \end{frame}
    
    \begin{frame}[allowframebreaks] 
    \frametitle{Equation}
        \centering
        \label{tab:parameters}
Given points are:
\begin{align}
 \Vec{P}=\myvec{2\\1\\5}\;\;,\Vec{Q}=\myvec{-1\\3\\4}\;\;and\;\;\Vec{R}=\myvec{3\\0\\6}
\end{align}
    \end{frame}
    
    \begin{frame}
    \frametitle{Theoretical Solution}
   Now finding two vectors in the plane:

\begin{align}
 \Vec{P}\Vec{Q}=\myvec{-3\\2\\-1}
\end{align}
\begin{align}
\Vec{P}\Vec{R}=\myvec{1\\-1\\1}
\end{align}
Now the required vector be $\Vec{n}$
\begin{align}
\Vec{n}=\Vec{P}\Vec{Q}\times\Vec{P}\Vec{R}
\end{align}




    \end{frame}
    
    \begin{frame}
    \frametitle{Theoretical solution}
Let
\begin{align}
    \Vec{A}=\Vec{P}\Vec{Q}\;\;and\;\;\Vec{B}=\Vec{P}\Vec{R}
\end{align}
Then
\begin{align}
    \Vec{n}=\Vec{A}\times\Vec{B}
\end{align}
\begin{align}
\Vec{A}\times\Vec{B}=\myvec{\mydet{\Vec{A_{23}}\Vec{B_{23}}}\\\\\mydet{\Vec{A_{31}} \Vec{B_{31}}}\\\\\mydet{\Vec{A_{12}} \Vec{B_{12}}}}
\end{align}

\begin{align}
 \mydet{\Vec{A_{23}}\Vec{B_{23}}}=\mydet{2&-1\\-1&1}=1
\end{align}





    \end{frame}

     \begin{frame}
    \frametitle{Theoretical solution}

    \begin{align}
 \mydet{\Vec{A_{31}} \Vec{B_{31}}}=-\mydet{-3&-1\\1&1}=2
\end{align}

\begin{align}
 \mydet{\Vec{A_{12}} \Vec{B_{12}}}=\mydet{-3&2\\1&-1}=1
\end{align}

\begin{align}
  \Vec{A}\times\Vec{B}=\myvec{1\\2\\1}
\end{align}
\begin{align}
   \Vec{n}=\myvec{1\\2\\1}
\end{align}

    \end{frame}

    
    
    
   
    \begin{frame}[fragile]
        \frametitle{C Code}
        \begin{lstlisting}
typedef struct {
    double x, y, z;
} Vector3D;

void find_normal_vector_lib(Vector3D p, Vector3D q, Vector3D r, Vector3D* normal_out) {
    Vector3D vec_pq = {q.x - p.x, q.y - p.y, q.z - p.z};
    Vector3D vec_pr = {r.x - p.x, r.y - p.y, r.z - p.z};
    
    normal_out->x = (vec_pq.y * vec_pr.z) - (vec_pq.z * vec_pr.y);
    normal_out->y = (vec_pq.z * vec_pr.x) - (vec_pq.x * vec_pr.z);
    normal_out->z = (vec_pq.x * vec_pr.y) - (vec_pq.y * vec_pr.x);
}
        \end{lstlisting}
    \end{frame}
    
   
    
    \begin{frame}[fragile]
        \frametitle{Python Code}
        \begin{lstlisting}
import ctypes
import os
import numpy as np
import matplotlib.pyplot as plt

def plot_3d(p, q, r, normal):
    """Visualizes the plane, points, and normal vector with coordinate labels."""
    fig = plt.figure(figsize=(10, 8))
    ax = fig.add_subplot(111, projection='3d')

    points = np.array([p, q, r])
    ax.scatter(points[:,0], points[:,1], points[:,2], color='red', s=100)

    # Add text labels with coordinates
    ax.text(p[0], p[1], p[2] + 0.2, f' P({p[0]}, {p[1]}, {p[2]})', color='darkred')
  

        \end{lstlisting}
    \end{frame}
    
    \begin{frame}[fragile]
        \frametitle{Python Code}
        \begin{lstlisting}
   ax.text(q[0], q[1], q[2] + 0.2, f' Q({q[0]}, {q[1]}, {q[2]})', color='darkred')
    ax.text(r[0], r[1], r[2] + 0.2, f' R({r[0]}, {r[1]}, {r[2]})', color='darkred')
    
    # Create and plot the plane
    d = np.dot(normal, p)
    x_range = np.linspace(min(points[:,0])-2, max(points[:,0])+2, 10)
    y_range = np.linspace(min(points[:,1])-2, max(points[:,1])+2, 10)
    xx, yy = np.meshgrid(x_range, y_range)
    zz = (d - normal[0] * xx - normal[1] * yy) / normal[2]
    ax.plot_surface(xx, yy, zz, alpha=0.4, color='cyan')

    
   
        \end{lstlisting}
    \end{frame}
    
    \begin{frame}[fragile]
        \frametitle{Python Code}
        \begin{lstlisting}
   # Plot the normal vector
    ax.quiver(p[0], p[1], p[2], normal[0], normal[1], normal[2],
              length=4, normalize=True, color='black', arrow_length_ratio=0.2, label='Normal Vector')

    ax.set_xlabel('X-axis'); ax.set_ylabel('Y-axis'); ax.set_zlabel('Z-axis')
    ax.set_title('Plane and Perpendicular Vector')
    ax.legend()
    plt.savefig("/media/indhiresh-s/New Volume/Matrix/ee1030-2025/ee25btech11027/MATGEO/12.249/figs/figure1.png")
    plt.show()

class Vector3D(ctypes.Structure):
    _fields_ = [("x", ctypes.c_double), ("y", ctypes.c_double), ("z", ctypes.c_double)]



   
        \end{lstlisting}
    \end{frame}
    
    \begin{frame}[fragile]
        \frametitle{Python Code}
        \begin{lstlisting}
   c_lib_file = 'plane.so'

c_lib = ctypes.CDLL(os.path.abspath(c_lib_file))
c_lib.find_normal_vector_lib.argtypes = [Vector3D, Vector3D, Vector3D, ctypes.POINTER(Vector3D)]
c_lib.find_normal_vector_lib.restype = None

p_pt, q_pt, r_pt = Vector3D(2, 1, 5), Vector3D(-1, 3, 4), Vector3D(3, 0, 6)
normal_out = Vector3D()
c_lib.find_normal_vector_lib(p_pt, q_pt, r_pt, ctypes.byref(normal_out))



        \end{lstlisting}
    \end{frame}

    \begin{frame}[fragile]
        \frametitle{Python Code}
        \begin{lstlisting}
    normal_vector = np.array([normal_out.x, normal_out.y, normal_out.z])
p_np, q_np, r_np = np.array([p_pt.x, p_pt.y, p_pt.z]), np.array([q_pt.x, q_pt.y, q_pt.z]), np.array([r_pt.x, r_pt.y, r_pt.z])

print(f"Perpendicular vector (from C): {normal_vector}")
plot_3d(p_np, q_np, r_np, normal_vector)

   
        \end{lstlisting}
    \end{frame}

    
    
    \end{document}