\documentclass[journal]{IEEEtran}
\usepackage[a5paper, margin=10mm]{geometry}
%\usepackage{lmodern} % Ensure lmodern is loaded for pdflatex
\usepackage{tfrupee} % Include tfrupee package


\setlength{\headheight}{1cm} % Set the height of the header box
\setlength{\headsep}{0mm}     % Set the distance between the header box and the top of the text


%\usepackage[a5paper, top=10mm, bottom=10mm, left=10mm, right=10mm]{geometry}

%
\setlength{\intextsep}{10pt} % Space between text and floats

\makeindex


\usepackage{cite}
\usepackage{amsmath,amssymb,amsfonts,amsthm}
\usepackage{algorithmic}
\usepackage{graphicx}
\usepackage{textcomp}
\usepackage{xcolor}
\usepackage{txfonts}
\usepackage{listings}
\usepackage{enumitem}
\usepackage{mathtools}
\usepackage{gensymb}
\usepackage{comment}
\usepackage[breaklinks=true]{hyperref}
\usepackage{tkz-euclide} 
\usepackage{listings}
\usepackage{multicol}
\usepackage{xparse}
\usepackage{gvv}
%\def\inputGnumericTable{}                                 
\usepackage[latin1]{inputenc}                                
\usepackage{color}                                            
\usepackage{array}                                            
\usepackage{longtable}                                       
\usepackage{calc}                                             
\usepackage{multirow}                                         
\usepackage{hhline}                                           
\usepackage{ifthen}                                               
\usepackage{lscape}
\usepackage{tabularx}
\usepackage{array}
\usepackage{float}
\usepackage{ar}
\usepackage[version=4]{mhchem}


\newtheorem{theorem}{Theorem}[section]
\newtheorem{problem}{Problem}
\newtheorem{proposition}{Proposition}[section]
\newtheorem{lemma}{Lemma}[section]
\newtheorem{corollary}[theorem]{Corollary}
\newtheorem{example}{Example}[section]
\newtheorem{definition}[problem]{Definition}
\newcommand{\BEQA}{\begin{eqnarray}}
\newcommand{\EEQA}{\end{eqnarray}}

\theoremstyle{remark}


\begin{document}
\bibliographystyle{IEEEtran}
\onecolumn

\title{5.4.22}
\author{INDHIRESH S- EE25BTECH11027}
\maketitle


\renewcommand{\thefigure}{\theenumi}
\renewcommand{\thetable}{\theenumi}

\textbf{Question}.Using elementary transformations, find the inverse of the following matrices
\begin{align*}
    \myvec{6&-3\\-2&1}
\end{align*}
\textbf{Solution}:\\

Let the given matrix  be:
\begin{align}
   \Vec{A}=\myvec{6&-3\\-2&1}
\end{align}
Now finding the inverse of a matrix by elementary operation.\\
Now forming the augmented matrix $[\Vec{A}|\Vec{I}]$

\begin{align}
[\Vec{A}|\Vec{I}]=\augvec{2}{2}{6&-3&1&0\\-2&1&0&1}
\end{align}

\begin{align}
   \augvec{2}{2}{6&-3&1&0\\-2&1&0&1}\xleftrightarrow{R_2\longleftarrow R_2+\frac{1}{3}R_1}  \augvec{2}{2}{6&-3&1&0\\0&0&\frac{1}{3}&1}
\end{align}
From above we can observe that the rank of the left-side augmented matrix is 1\\
Therefore the matrix $\Vec{A}$ is singular and hence the inverse does not exist for the given matrix\\




\end{document}