\documentclass[journal,12pt,onecolumn]{IEEEtran}
\usepackage{cite}
 \usepackage{caption}
\usepackage{graphicx}
\usepackage{amsmath,amssymb,amsfonts,amsthm}
\usepackage{algorithmic}
\usepackage{graphicx}
\usepackage{textcomp}
\usepackage{xcolor}
\usepackage{txfonts}
\usepackage{listings}
\usepackage{enumitem}
\usepackage{mathtools}
\usepackage{gensymb}
\usepackage{comment}
\usepackage[breaklinks=true]{hyperref}
\usepackage{tkz-euclide} 
\usepackage{listings}
\usepackage{gvv}
%\def\inputGnumericTable{}                                 
\usepackage[latin1]{inputenc} 
\usetikzlibrary{arrows.meta, positioning}
\usepackage{xparse}
\usepackage{color}                                            
\usepackage{array}                                            
\usepackage{longtable}                                       
\usepackage{calc}                                             
\usepackage{multirow}
\usepackage{multicol}
\usepackage{hhline}                                           
\usepackage{ifthen}                                           
\usepackage{lscape}
\usepackage{tabularx}
\usepackage{array}
\usepackage{float}

\usepackage{float}
%\newcommand{\define}{\stackrel{\triangle}{=}}
\theoremstyle{remark}
\usepackage{circuitikz}
\captionsetup{justification=centering}
\usepackage{tikz}

\title{Matrices in Geometry 5.13.61}
\author{EE25BTECH11035 - Kushal B N}
\begin{document}
\vspace{3cm}
\maketitle
{\let\newpage\relax\maketitle}
\textbf{Question: }
Let $\vec{P} = \myvec{1&0&0\\4&1&0\\16&4&1}$ and $\vec{I}$ be the identity matrix of order 3. If $\vec{Q} = q_{ij}$ is a matrix such that $\vec{P}^{50} - \vec{Q} = \vec{I}$, then $\frac{q_{31}+q_{32}}{q_{21}}$ equals
\hfill{\brak{JEE Adv. 2016}}

\begin{enumerate}
\begin{multicols}{4}
    \item 52
    \item 103
    \item 201
    \item 205
\end{multicols}
\end{enumerate}

\textbf{Given: } \\
The matrix $\vec{P} = \myvec{1&0&0\\4&1&0\\16&4&1}$ and $\vec{Q} = \vec{P}^{50} - \vec{I}$

\textbf{Solution: }\\
Let us express the matrix $\vec{P}$ as\\
\begin{equation}
    \vec{P} = \vec{I} + \vec{N}
\end{equation}
where 
\begin{equation}
    \vec{N} = \myvec{0&0&0\\4&0&0\\16&4&0}
\end{equation}
Now we see that 
\begin{equation}
    \vec{N}^2 = \myvec{0&0&0\\0&0&0\\16&0&0}
\end{equation}

\begin{equation}
    \vec{N}^3 = \vec{0}
\end{equation}

So that now by binomial expansion we have,
\begin{equation}
    \vec{P}^{50} = \brak{\vec{I}+\vec{N}}^{50} 
\end{equation}

from (4),
\begin{equation}
    \implies \vec{P}^{50} = \vec{I} + 50\vec{N} + 1225\vec{N}^2
\end{equation}

\begin{equation}
    \implies \vec{Q} = 50\vec{N} + 1225\vec{N}^2
\end{equation}

\begin{equation}
    \vec{Q} = \myvec{0&0&0\\200&0&0\\20400&200&0}
\end{equation}

\begin{equation}
    \implies \fbox{$\frac{q_{31}+q_{32}}{q_{21}} = 103$}
\end{equation}

\textbf{Final Answer: }\\
$\therefore$ The value of the given expression $\frac{q_{31}+q_{32}}{q_{21}} = 103$.\\
Hence, the correct answer is (2).
\end{document}