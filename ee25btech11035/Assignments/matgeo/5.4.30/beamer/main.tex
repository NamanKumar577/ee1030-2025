\documentclass{beamer}
\let\vec\mathbf
\mode<presentation>
\usepackage{amsmath}
\usepackage{amssymb}
%\usepackage{advdate}
\usepackage{adjustbox}
%\usepackage{subcaption}
\usepackage{xparse}
\usepackage{enumitem}
\usepackage{multicol}
\usepackage{mathtools}
\usepackage{listings}
\usepackage{gvv}
\usepackage{url}
\usetheme{Boadilla}
\usecolortheme{lily}
\setbeamertemplate{footline}
{
  \leavevmode%
  \hbox{%
  \begin{beamercolorbox}[wd=\paperwidth,ht=2.25ex,dp=1ex,right]{author in head/foot}%
    \insertframenumber{} / \inserttotalframenumber\hspace*{2ex} 
  \end{beamercolorbox}}%
  \vskip0pt%
}
\setbeamertemplate{navigation symbols}{}
\providecommand{\nCr}[2]{\,^{#1}C_{#2}} % nCr
\providecommand{\nPr}[2]{\,^{#1}P_{#2}} % nPr
\providecommand{\mbf}{\mathbf}
\providecommand{\pr}[1]{\ensuremath{\Pr\left(#1\right)}}
\providecommand{\qfunc}[1]{\ensuremath{Q\left(#1\right)}}
\providecommand{\sbrak}[1]{\ensuremath{{}\left[#1\right]}}
\providecommand{\lsbrak}[1]{\ensuremath{{}\left[#1\right.}}
\providecommand{\rsbrak}[1]{\ensuremath{{}\left.#1\right]}}
\providecommand{\brak}[1]{\ensuremath{\left(#1\right)}}
\providecommand{\lbrak}[1]{\ensuremath{\left(#1\right.}}
\providecommand{\rbrak}[1]{\ensuremath{\left.#1\right)}}
\providecommand{\cbrak}[1]{\ensuremath{\left\{#1\right\}}}
\providecommand{\lcbrak}[1]{\ensuremath{\left\{#1\right.}}
\providecommand{\rcbrak}[1]{\ensuremath{\left.#1\right\}}}
\theoremstyle{remark}

\providecommand{\res}[1]{\Res\displaylimits_{#1}} 
\providecommand{\norm}[1]{\lVert#1\rVert}
\providecommand{\mtx}[1]{\mathbf{#1}}

\providecommand{\fourier}{\overset{\mathcal{F}}{ \rightleftharpoons}}
%\providecommand{\hilbert}{\overset{\mathcal{H}}{ \rightleftharpoons}}
\providecommand{\system}{\overset{\mathcal{H}}{ \longleftrightarrow}}
	%\newcommand{\solution}[2]{\textbf{Solution:}{#1}}
%\newcommand{\solution}{\noindent \textbf{Solution: }}
\providecommand{\dec}[2]{\ensuremath{\overset{#1}{\underset{#2}{\gtrless}}}}

\title{Matrices in Geometry - 5.4.30}
\author{EE25BTECH11035  Kushal B N}
\date{Oct, 2025}

\begin{document}

\maketitle

\section{Problem Statement}
\begin{frame}
\frametitle{Problem Statement}
Using elementary transformations, find the inverse of the following matrix.\\
$\myvec{x^2-x+1 & x-1 \\ x+1 & x+1}$
\end{frame}

\section{Solution}
\begin{frame}{Solution}
Let $\vec{A} = \myvec{x^2-x+1 & x-1 \\ x+1 & x+1}$

If $\vec{B}$ is the inverse of the matrix i.e,
\begin{equation}
    \vec{B} = \vec{A}^{-1}
\end{equation}
\begin{equation}
    \implies \vec{A}\vec{B} = \vec{I}
\end{equation}
Forming the augmented matrix for this in order to solve for $\vec{B}$

\begin{equation}
    \augvec{2}{2}{x^2-x+1 & x-1 &1&0\\x+1 & x+1&0&1}
\end{equation}
\end{frame}

\begin{frame}{Solution}
\begin{equation}
 \xleftrightarrow{R_2 \leftarrow R_2 - \frac{x+1}{x^2 - x + 1}R_1}
\augvec{2}{2}{
x^2 - x + 1 & x - 1 & 1 & 0 \\
0 & \frac{(x+1)(x^2 - 2x + 2)}{x^2 - x + 1} & -\frac{x+1}{x^2 - x + 1} & 1
}
\end{equation}

\begin{equation}
\xleftrightarrow{R_2 \leftarrow \frac{x^2 - x + 1}{(x+1)(x^2 - 2x + 2)}R_2}
\augvec{2}{2}{
x^2 - x + 1 & x - 1 & 1 & 0 \\
0 & 1 & -\frac{1}{x^2 - 2x + 2} & \frac{x^2 - x + 1}{(x+1)(x^2 - 2x + 2)}
}
\end{equation}

\begin{equation}
\xleftrightarrow{R_1 \leftarrow R_1 - (x-1)R_2}
\augvec{2}{2}{
x^2 - x + 1 & 0 & \frac{x^2 - x + 1}{x^2 - 2x + 2} & -\frac{(x-1)(x^2 - x + 1)}{(x+1)(x^2 - 2x + 2)} \\
0 & 1 & -\frac{1}{x^2 - 2x + 2} & \frac{x^2 - x + 1}{(x+1)(x^2 - 2x + 2)}
}
\end{equation}
\end{frame}

\begin{frame}{Solution}
\begin{equation}
\xleftrightarrow{R_1 \leftarrow \frac{1}{x^2 - x + 1}R_1}
\augvec{2}{2}{
1 & 0 & \frac{1}{x^2 - 2x + 2} & -\frac{x-1}{(x+1)(x^2 - 2x + 2)} \\
0 & 1 & -\frac{1}{x^2 - 2x + 2} & \frac{x^2 - x + 1}{(x+1)(x^2 - 2x + 2)}
}
\end{equation}

\begin{equation}
\implies \vec{A}^{-1} =
\myvec{
\frac{1}{x^2 - 2x + 2} & -\frac{x-1}{(x+1)(x^2 - 2x + 2)} \\
-\frac{1}{x^2 - 2x + 2} & \frac{x^2 - x + 1}{(x+1)(x^2 - 2x + 2)}
}
\end{equation}

\begin{equation}
    \fbox{$\vec{A}^{-1} = \myvec{
\frac{1}{x^2 - 2x + 2} & \frac{1-x}{x^3 - x^2 + 2} \\
-\frac{1}{x^2 - 2x + 2} & \frac{x^2 - x + 1}{x^3 - x^2 + 2}
}$} 
\end{equation}
\end{frame}

\section{Final Answer}
\begin{frame}{final Answer}
$\therefore$ The inverse of the given matrix is $\myvec{
\frac{1}{x^2 - 2x + 2} & \frac{1-x}{x^3 - x^2 + 2} \\
-\frac{1}{x^2 - 2x + 2} & \frac{x^2 - x + 1}{x^3 - x^2 + 2}
}$
\end{frame}
\end{document}

