 \let\negmedspace\undefined
\let\negthickspace\undefined
\documentclass[journal]{IEEEtran}
\usepackage[a5paper, margin=10mm, onecolumn]{geometry}
%\usepackage{lmodern} % Ensure lmodern is loaded for pdflatex
\usepackage{tfrupee} % Include tfrupee package

\setlength{\headheight}{1cm} % Set the height of the header box
\setlength{\headsep}{0mm}     % Set the distance between the header box and the top of the text
\usepackage{gvv-book}
\usepackage{gvv}
\usepackage{cite}
\usepackage{amsmath,amssymb,amsfonts,amsthm}
\usepackage{algorithmic}
\usepackage{graphicx}
\usepackage{textcomp}
\usepackage{xcolor}
\usepackage{txfonts}
\usepackage{listings}
\usepackage{enumitem}
\usepackage{mathtools}
\usepackage{gensymb}
\usepackage{comment}
\usepackage[breaklinks=true]{hyperref}
\usepackage{tkz-euclide} 
\usepackage{listings}
% \usepackage{gvv}                                        
\def\inputGnumericTable{}                                 
\usepackage[latin1]{inputenc}                                
\usepackage{color}                                            
\usepackage{array}                                            
\usepackage{longtable}                                       
\usepackage{calc}                                             
\usepackage{multirow}                                         
\usepackage{hhline}                                           
\usepackage{ifthen}                                           
\usepackage{lscape}



\usepackage{amsmath,amssymb}
\usepackage{booktabs}
\usepackage{tikz}
\usetikzlibrary{arrows.meta,angles,quotes}





\begin{document}

\bibliographystyle{IEEEtran}
\vspace{3cm}

\title{5.6.9}
\author{AI25BTECH11023 - Pratik R}
% \maketitle
% \newpage
% \bigskip
{\let\newpage\relax\maketitle}

\renewcommand{\thefigure}{\theenumi}
\renewcommand{\thetable}{\theenumi}
\setlength{\intextsep}{10pt} % Space between text and floats


\numberwithin{equation}{enumi}
\numberwithin{figure}{enumi}
\renewcommand{\thetable}{\theenumi}


\section*{\textbf{Question}}
Let $\vec{A}= \myvec{0&1\\0&0}$, show that $\brak{a\vec{I} + b\vec{A}}^n = a^n\vec{I} + na^{n-1}b\vec{A}$.

\subsection*{\textbf{Solution:}} 
Given
\begin{align}
  \vec{A}= \myvec{0&1\\0&0}
\end{align}
calculating $\vec{A}^2$ we get 
\begin{align}
   \vec{A}^2 = \vec{0}
\end{align}
Using binomial expansion
\begin{align}
\brak{a\vec{I} + b\vec{A}}^n = {n\choose 0}\brak{a\vec{I}}^n + {n\choose 1}\brak{a\vec{I}}^{n-1}\brak{b\vec{A}}^{1} + {n\choose 2}\brak{a\vec{I}}^{n-2}\brak{b\vec{A}}^{2} +... {n\choose n}\brak{b\vec{A}}^{n}
\end{align}
Since $\vec{A}^2 =0, \vec{A}^3 =0, \vec{A}^4=0,...\vec{A}^n =0$
\begin{align}
   \therefore \brak{a\vec{I} + b\vec{A}}^n = {n\choose 0}\brak{a\vec{I}}^n + {n\choose 1}\brak{a\vec{I}}^{n-1}\brak{b\vec{A}}^{1}
\end{align}
\begin{align}
   = a^n\vec{I} + na^{n-1}b\vec{A}
\end{align}
Hence proved.
\end{document}
