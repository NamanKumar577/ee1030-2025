\let\negmedspace\undefined
\let\negthickspace\undefined
\documentclass[journal]{IEEEtran}
\usepackage[a5paper, margin=10mm, onecolumn]{geometry}
%\usepackage{lmodern} % Ensure lmodern is loaded for pdflatex
\usepackage{tfrupee} % Include tfrupee package

\setlength{\headheight}{1cm} % Set the height of the header box
\setlength{\headsep}{0mm}     % Set the distance between the header box and the top of the text
\usepackage{gvv-book}
\usepackage{gvv}
\usepackage{cite}
\usepackage{amsmath,amssymb,amsfonts,amsthm}
\usepackage{algorithmic}
\usepackage{graphicx}
\usepackage{textcomp}
\usepackage{xcolor}
\usepackage{txfonts}
\usepackage{listings}
\usepackage{enumitem}
\usepackage{mathtools}
\usepackage{gensymb}
\usepackage{comment}
\usepackage[breaklinks=true]{hyperref}
\usepackage{tkz-euclide} 
\usepackage{listings}
% \usepackage{gvv}                                        
\def\inputGnumericTable{}                                 
\usepackage[latin1]{inputenc}                                
\usepackage{color}                                            
\usepackage{array}                                            
\usepackage{longtable}                                       
\usepackage{calc}                                             
\usepackage{multirow}                                         
\usepackage{hhline}                                           
\usepackage{ifthen}                                           
\usepackage{lscape}



\usepackage{amsmath,amssymb}
\usepackage{booktabs}
\usepackage{tikz}
\usetikzlibrary{arrows.meta,angles,quotes}





\begin{document}

\bibliographystyle{IEEEtran}
\vspace{3cm}

\title{5.13.9}
\author{AI25BTECH11023 - Pratik R}
% \maketitle
% \newpage
% \bigskip
{\let\newpage\relax\maketitle}

\renewcommand{\thefigure}{\theenumi}
\renewcommand{\thetable}{\theenumi}
\setlength{\intextsep}{10pt} % Space between text and floats


\numberwithin{equation}{enumi}
\numberwithin{figure}{enumi}
\renewcommand{\thetable}{\theenumi}


\section*{\textbf{Question}}
Let $\vec{P}$ and $\vec{Q}$ be $3\times 3$ matrices $\vec{P}\neq \vec{Q}$. If $\vec{P}^3 = \vec{Q}^3$ and $\vec{P}^2\vec{Q}=\vec{Q}^2\vec{P}$ then determinant of $(\vec{P}^2+\vec{Q}^2)$ is equal to 

\subsection*{\textbf{Solution:}} 
Given
\begin{align}
  \vec{P}\neq \vec{Q} \\
  \vec{P}^3 = \vec{Q}^3 \\
  \vec{P}^2\vec{Q}=\vec{Q}^2\vec{P}
\end{align}
let us solve for $(\vec{P}^2+\vec{Q}^2)(\vec{P}-\vec{Q})$
\begin{align}
   (\vec{P}^2+\vec{Q}^2)(\vec{P}-\vec{Q}) = \vec{P}^3 - \vec{P}^2\vec{Q} + \vec{Q}^2\vec{P} - \vec{Q}^3
\end{align}
from equation \textbf{(0.2)} and \textbf{(0.3)}
\begin{align}
 (\vec{P}^2+\vec{Q}^2)(\vec{P}-\vec{Q}) =\vec{0}
\end{align}
Let us assume $\det(\vec{P}^2+\vec{Q}^2) \neq 0$ \\
then $(\vec{P}^2+\vec{Q}^2)$ is invertible and hence 
$(\vec{P}^2+\vec{Q}^2)^{-1}$ exists
\begin{align}
   \therefore \vec{P}-\vec{Q} =\vec{0} \\
   \implies \vec{P}=\vec{Q}
\end{align}
which contradicts equation \textbf{(0.1)}\\
Hence 
\begin{align}
    \det(\vec{P}^2+\vec{Q}^2)= 0
\end{align}

\end{document}
