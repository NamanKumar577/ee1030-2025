\documentclass{beamer}
\usepackage[utf8]{inputenc}

\usetheme{Madrid}
\usecolortheme{default}
\usepackage{txfonts}
\usepackage{listings}
\usepackage{adjustbox}
\usepackage{tabularx}
\usepackage{lmodern}
\usepackage{circuitikz}
\usepackage{tikz}

\usepackage{gvv}
\usepackage{cite}
\usepackage{amsmath,amssymb,amsfonts,amsthm}
\usepackage{algorithmic}
\usepackage{graphicx}
\usepackage{textcomp}
\usepackage{xcolor}
\usepackage{txfonts}
\usepackage{listings}
\usepackage{enumitem}
\usepackage{mathtools}
\usepackage{gensymb}
\usepackage{comment}
\usepackage{tkz-euclide} 
\usepackage{listings}                                      
\def\inputGnumericTable{}                                
\usepackage{color}                                            
\usepackage{array}                                            
\usepackage{longtable}
\usepackage{multicol}
\usepackage{calc}                                             
\usepackage{multirow}                                         
\usepackage{hhline}                                           
\usepackage{ifthen}

\setbeamertemplate{page number in head/foot}[totalframenumber]

\usepackage{tcolorbox}
\tcbuselibrary{minted,breakable,xparse,skins}



\definecolor{bg}{gray}{0.95}
\DeclareTCBListing{mintedbox}{O{}m!O{}}{%
  breakable=true,
  listing engine=minted,
  listing only,
  minted language=#2,
  minted style=default,
  minted options={%
    linenos,
    gobble=0,
    breaklines=true,
    breakafter=,,
    fontsize=\small,
    numbersep=8pt,
    #1},
  boxsep=0pt,
  left skip=0pt,
  right skip=0pt,
  left=25pt,
  right=0pt,
  top=3pt,
  bottom=3pt,
  arc=5pt,
  leftrule=0pt,
  rightrule=0pt,
  bottomrule=2pt,
  toprule=2pt,
  colback=bg,
  colframe=orange!70,
  enhanced,
  overlay={
    \begin{tcbclipinterior}
    \fill[orange!20!white] (frame.south west) rectangle ([xshift=20pt]frame.north west);
    \end{tcbclipinterior}},
  #3,
}
\lstset{
    language=C,
    basicstyle=\ttfamily\small,
    keywordstyle=\color{blue},
    stringstyle=\color{orange},
    commentstyle=\color{green!60!black},
    numbers=left,
    numberstyle=\tiny\color{gray},
    breaklines=true,
    showstringspaces=false,
}

\title 
{2.8.14}
\date{September 14, 2025}


\author 
{Sai Sreevallabh - EE25BTECH11031}



\begin{document}


\frame{\titlepage}
\begin{frame}{Question}
Three vectors $\vec{a}$, $\vec{b}$ and $\vec{c}$ satisfy the condition $\vec{a}+\vec{b}+\vec{c}=0$. Evaluate the quantity $\vec{\mu} = \vec{a}\cdot\vec{b} + \vec{b}\cdot\vec{c} + \vec{c}\cdot\vec{a}$. If $\abs{\vec{a}} = 3$, $\abs{\vec{b}}=4$ and $\abs{\vec{c}}=2$. \\
\end{frame}



\begin{frame}{Solution}

Given: 
\begin{align}
    \vec{a}+\vec{b}+\vec{c}=0 \ \text{and}\ \norm{\vec{a}}=3 \ , \ \norm{\vec{b}}=4 \ , \ \norm{\vec{c}}=2
\end{align}

To find
\begin{align}
    \vec{\mu} = \vec{a}^\top\vec{b}+\vec{b}^\top\vec{c}+\vec{c}^\top\vec{a}
\end{align}

\end{frame}

\begin{frame}{Solution}

To find the value of $\vec{\mu}$

\begin{align}
    \norm{\vec{a}+\vec{b}+\vec{c}}^2 &= 0\\
    \brak{\vec{a}+\vec{b}+\vec{c}}^\top\brak{\vec{a}+\vec{b}+\vec{c}}&=0\\
    \vec{a}^\top\vec{a} + \vec{b}^\top\vec{b} +\vec{c}^\top\vec{c} + 2\brak{\vec{a}^\top\vec{b} + \vec{b}^\top\vec{c} + \vec{c}^\top\vec{a}} &= 0
\end{align}

\end{frame}

\begin{frame}{Solution}

By using $\vec{x}^\top\vec{x} = \norm{\vec{x}}^2$ we get
\begin{align}
    \brak{\norm{\vec{a}}^2 + \norm{\vec{b}}^2 + \norm{\vec{c}}^2} + 2\mu = 0
\end{align}

Substituting the values of $\norm{\vec{a}}, \norm{\vec{b}}, \norm{\vec{c}}$ we get
\begin{align}
  \mu = \frac{-29}{2}
\end{align}

$\therefore$ The value of $\mu$ is $\frac{-29}{2}$.

\end{frame}

\end{document}
