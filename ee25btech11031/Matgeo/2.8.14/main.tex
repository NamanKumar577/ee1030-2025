\let\negmedspace\undefined
\let\negthickspace\undefined
\documentclass[journal]{IEEEtran}
\usepackage[a5paper, margin=10mm, onecolumn]{geometry}
\usepackage{lmodern} 
\usepackage{tfrupee} 

\setlength{\headheight}{1cm} 
\setlength{\headsep}{0mm}    

\usepackage{gvv-book}
\usepackage{gvv}
\usepackage{cite}
\usepackage{amsmath,amssymb,amsfonts,amsthm}
\usepackage{algorithmic}
\usepackage{graphicx}
\usepackage{textcomp}
\usepackage{xcolor}
\usepackage{txfonts}
\usepackage{listings}
\usepackage{enumitem}
\usepackage{mathtools}
\usepackage{gensymb}
\usepackage{comment}
\usepackage[breaklinks=true]{hyperref}
\usepackage{tkz-euclide} 
\usepackage{listings}                                      
\def\inputGnumericTable{}                                 
\usepackage[latin1]{inputenc}                                
\usepackage{color}                                            
\usepackage{array}                                            
\usepackage{longtable}
\usepackage{multicol}
\usepackage{calc}                                             
\usepackage{multirow}                                         
\usepackage{hhline}                                           
\usepackage{ifthen}                                           
\usepackage{lscape}
\begin{document}

\bibliographystyle{IEEEtran}
\vspace{3cm}

\title{2.8.14}
\author {EE25BTECH11031 - Sai Sreevallabh}
% \maketitle
% \newpage
% \bigskip
{\let\newpage\relax\maketitle}

\renewcommand{\thefigure}{\theenumi}
\renewcommand{\thetable}{\theenumi}
\setlength{\intextsep}{10pt} % Space between text and floats


\numberwithin{equation}{enumi}
\numberwithin{figure}{enumi}
\renewcommand{\thetable}{\theenumi}

\textbf{Question: }\\

Three vectors $\vec{a}$, $\vec{b}$ and $\vec{c}$ satisfy the condition $\vec{a}+\vec{b}+\vec{c}=0$. Evaluate the quantity $\vec{\mu} = \vec{a}\cdot\vec{b} + \vec{b}\cdot\vec{c} + \vec{c}\cdot\vec{a}$. If $\abs{\vec{a}} = 3$, $\abs{\vec{b}}=4$ and $\abs{\vec{c}}=2$. \\

\textbf{Solution: }\\

Given: 
\begin{align}
    \vec{a}+\vec{b}+\vec{c}=0 \ \text{and}\ \norm{\vec{a}}=3 \ , \ \norm{\vec{b}}=4 \ , \ \norm{\vec{c}}=2
\end{align}

To find
\begin{align}
    \vec{\mu} = \vec{a}^\top\vec{b}+\vec{b}^\top\vec{c}+\vec{c}^\top\vec{a}
\end{align}

To find the value of $\vec{\mu}$

\begin{align}
    \norm{\vec{a}+\vec{b}+\vec{c}}^2 &= 0\\
    \brak{\vec{a}+\vec{b}+\vec{c}}^\top\brak{\vec{a}+\vec{b}+\vec{c}}&=0\\
    \vec{a}^\top\vec{a} + \vec{b}^\top\vec{b} +\vec{c}^\top\vec{c} + 2\brak{\vec{a}^\top\vec{b} + \vec{b}^\top\vec{c} + \vec{c}^\top\vec{a}} &= 0
\end{align}

By using $\vec{x}^\top\vec{x} = \norm{\vec{x}}^2$ we get
\begin{align}
    \brak{\norm{\vec{a}}^2 + \norm{\vec{b}}^2 + \norm{\vec{c}}^2} + 2\mu = 0
\end{align}\\

Substituting the values of $\norm{\vec{a}}, \norm{\vec{b}}, \norm{\vec{c}}$ we get
\begin{align}
  \mu = \frac{-29}{2}
\end{align}

$\therefore$ The value of $\mu$ is $\frac{-29}{2}$.

\end{document}

