\let\negmedspace\undefined
\let\negthickspace\undefined
\documentclass[journal]{IEEEtran}
\usepackage[a5paper, margin=10mm, onecolumn]{geometry}

\usepackage{tfrupee} % Include tfrupee package

\setlength{\headheight}{1cm} % Set the height of the header box
\setlength{\headsep}{0mm}     % Set the distance between the header box and the top of the text

\usepackage{gvv-book}
\usepackage{gvv}
\usepackage{cite}
\usepackage{amsmath,amssymb,amsfonts,amsthm}
\usepackage{algorithmic}
\usepackage{graphicx}
\usepackage{textcomp}
\usepackage{xcolor}
\usepackage{txfonts}
\usepackage{listings}
\usepackage{multicol}
\usepackage{enumitem}
\usepackage{mathtools}
\usepackage{gensymb}
\usepackage{comment}
\usepackage[breaklinks=true]{hyperref}
\usepackage{tkz-euclide} 
\usepackage{listings}
% \usepackage{gvv}                                       
\def\inputGnumericTable{}                                 
\usepackage[latin1]{inputenc}                                
\usepackage{color}                                            
\usepackage{array}                                            
\usepackage{longtable}                                       
\usepackage{calc}                                             
\usepackage{multirow}                                         
\usepackage{hhline}                                           
\usepackage{ifthen}                                           
\usepackage{lscape}
\begin{document}
	
	\bibliographystyle{IEEEtran}
	\vspace{3cm}
	
	\title{12.378}
	\author{EE25BTECH11052 - Shriyansh Kalpesh Chawda}
	% \maketitle
	% \newpage
	% \bigskip
	{\let\newpage\relax\maketitle}
	
	\renewcommand{\thefigure}{\theenumi}
	\renewcommand{\thetable}{\theenumi}
	\setlength{\intextsep}{10pt} 
	
	\numberwithin{equation}{enumi}
	\numberwithin{figure}{enumi}
	\renewcommand{\thetable}{\theenumi}
	\textbf{Question}:\\
	The eigenvalues of the matrix 
	\[
	\myvec{6 & 1 \\ -2 & 3}
	\]
	are
	
	\begin{multicols}{4}
		\begin{enumerate}
			\item[(a)] (3, 6)
			\item[(b)] (1, -2)
			\item[(c)] (5, 4)
			\item[(d)] (1, 6)
		\end{enumerate}
	\end{multicols}

\textbf{Solution}\\
	According to the Cayley-Hamilton theorem,  $\det(\mathbf{A} - \lambda\mathbf{I}) = 0$.
	\begin{align}
		\det(\mathbf{A} - \lambda\mathbf{I}) &= \det\myvec{6-\lambda & 1 \\ -2 & 3-\lambda} \\
		&= (6-\lambda)(3-\lambda) - (1)(-2) \\
		&= 18 - 6\lambda - 3\lambda + \lambda^2 + 2 \\
		&= \lambda^2 - 9\lambda + 20 = 0\\
		&=(\lambda - 5)(\lambda - 4) = 0
	\end{align}
	Thus, the eigenvalues are $\lambda_1 = 5$ and $\lambda_2 = 4$. This corresponds to option (c).
	
\end{document}
