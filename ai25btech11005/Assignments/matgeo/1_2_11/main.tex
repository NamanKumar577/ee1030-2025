\let\negmedspace\undefined
\let\negthickspace\undefined
\documentclass[journal]{IEEEtran}
\usepackage[a5paper, margin=10mm, onecolumn]{geometry}
%\usepackage{lmodern} % Ensure lmodern is loaded for pdflatex
\usepackage{tfrupee} % Include tfrupee package

\setlength{\headheight}{1cm} % Set the height of the header box
\setlength{\headsep}{0mm}     % Set the distance between the header box and the top of the text

\usepackage{gvv-book}
\usepackage{gvv}
\usepackage{cite}
\usepackage{amsmath,amssymb,amsfonts,amsthm}
\usepackage{algorithmic}
\usepackage{graphicx}
\usepackage{textcomp}
\usepackage{xcolor}
\usepackage{txfonts}
\usepackage{listings}
\usepackage{enumitem}
\usepackage{mathtools}
\usepackage{gensymb}
\usepackage{comment}
\usepackage[breaklinks=true]{hyperref}
\usepackage{tkz-euclide} 
\usepackage{listings}
% \usepackage{gvv}                                        
\def\inputGnumericTable{}                                 
\usepackage[latin1]{inputenc}                                
\usepackage{color}                                            
\usepackage{array}                                            
\usepackage{longtable}                                       
\usepackage{calc}                                             
\usepackage{multirow}                                         
\usepackage{hhline}                                           
\usepackage{ifthen}                                           
\usepackage{lscape}
\usepackage{circuitikz}
\tikzstyle{block} = [rectangle, draw, fill=blue!20, 
    text width=4em, text centered, rounded corners, minimum height=3em]
\tikzstyle{sum} = [draw, fill=blue!10, circle, minimum size=1cm, node distance=1.5cm]
\tikzstyle{input} = [coordinate]
\tikzstyle{output} = [coordinate]


\begin{document}

\bibliographystyle{IEEEtran}
\vspace{3cm}

\title{1.2.11}
\author{ai25btech11005}
 \maketitle
% \newpage
% \bigskip
{\let\newpage\relax\maketitle}

\renewcommand{\thefigure}{\theenumi}
\renewcommand{\thetable}{\theenumi}
\setlength{\intextsep}{10pt} % Space between text and floats


\numberwithin{equation}{enumi}
\numberwithin{figure}{enumi}
\renewcommand{\thetable}{\theenumi}
Q. Find the slope of the lines
\begin{enumerate}
  \item Passing through the points \((3,-2)\) and \((-1,4)\).
  \item Passing through the points \((3,-2)\) and \((7,-2)\).
  \item Passing through the points \((3,-2)\) and \((3,4)\).
  \item Making an inclination of \(60^\circ\) with the positive direction of the \(x\)-axis.
\end{enumerate}

\textbf{Solution.}  

We will use direction ratios. For two points \(P(x_1,y_1)\) and \(Q(x_2,y_2)\), a direction vector (column matrix) is
\[
\vec{d}=\begin{pmatrix}x_2-x_1\\[4pt]y_2-y_1\end{pmatrix}=\begin{pmatrix}l\\[4pt]m\end{pmatrix},
\]
so the direction ratios are \((l,m)\) and the slope is 
\[
\frac{m}{l}\quad (l\neq0).
\]

\begin{enumerate}
\item \(P(3,-2),\;Q(-1,4)\).
\[
\vec{d}=\begin{pmatrix}-1-3\\[4pt]4-(-2)\end{pmatrix}=\begin{pmatrix}-4\\[4pt]6\end{pmatrix}.
\]
Direction ratios \((l,m)=(-4,6)\). Thus the slope is
\[
m=\frac{6}{-4}=-\frac{3}{2}.
\]

\item \(P(3,-2),\;Q(7,-2)\).
\[
\vec{d}=\begin{pmatrix}7-3\\[4pt]-2-(-2)\end{pmatrix}=\begin{pmatrix}4\\[4pt]0\end{pmatrix}.
\]
Direction ratios \((l,m)=(4,0)\). Slope \(=\dfrac{0}{4}=0\). (horizontal line)

\item \(P(3,-2),\;Q(3,4)\).
\[
\vec{d}=\begin{pmatrix}3-3\\[4pt]4-(-2)\end{pmatrix}=\begin{pmatrix}0\\[4pt]6\end{pmatrix}.
\]
Direction ratios \((l,m)=(0,6)\). Here \(l=0\), so the slope is undefined (vertical line).

\item Line making inclination \(\theta=60^\circ\) with positive \(x\)-axis.

A unit direction vector for angle \(\theta\) is 
\(\begin{pmatrix}\cos\theta\\[4pt]\sin\theta\end{pmatrix}\).  
Thus direction ratios may be taken as
\[
\begin{pmatrix}l\\[4pt]m\end{pmatrix}=\begin{pmatrix}\cos60^\circ\\[4pt]\sin60^\circ\end{pmatrix}=\begin{pmatrix}\tfrac{1}{2}\\[4pt]\tfrac{\sqrt3}{2}\end{pmatrix},
\]
so the slope is
\[
m=\frac{\sin60^\circ}{\cos60^\circ}=\tan60^\circ=\sqrt{3}.
\]
\end{enumerate}













































\end{document}
